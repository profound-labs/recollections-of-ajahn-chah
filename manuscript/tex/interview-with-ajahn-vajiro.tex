% Title: Interview with Ajahn Vajiro
% by a French television channel

\chapterNote{Conducted by a French television channel.}
\chapter{Interview with Ajahn Vajiro}

\questionBi%
{Comment avez-vous découvert cette tradition des moines de la forêt?}%
{How did you come across the Forest Monk tradition?}

\answer{}
Through an interest in meditation I heard about the Forest Tradition
in 1975. In 1977 I heard that there were Forest Monks in London where I
was living, so I went to see them. They had been invited to stay in
England to live as \emph{bhikkhus}.

\questionBi%
{Vous connaissiez bien Ajan Chah, quel souvenir en gardez-vous?}%
{You knew Ajahn Chah well. What memories do you have of him?}

\answer{}
I cannot claim to have known him well. I did not speak Thai at
all then and never lived with him for a long time.

I met him in 1977 during his first trip outside of Thailand. I was
struck by how at ease he was in whatever situation he found himself in. 
In 1979 he travelled outside Thailand again and I was then part of the
community. He came to check how the community of four monks he had left
in the UK was getting along. At that time someone had offered some
woodland for the forest monks, and a derelict house close to that
woodland had been bought with the funds from selling the property in
London. Ajahn Chah approved of the move, although he could see that the
house was not comfortable. He is reported to have often said that to
begin a monastery is difficult, but easier than to repair and maintain a
monastery, which is more difficult; and finally, that to have good wise
monks living in the monastery is the most difficult. 

I helped with the driving when he was making a tour of the United
Kingdom. We travelled all the way to Edinburgh in a very unsuitable slow
van. He never complained or showed any sign of impatience with the van. 
He did take the opportunity to teach and instruct with economy and
humour. I was drying his bowl, and he came over to where I was and with
the help of one of the other monks translating explained completely all
the stages in taking care of the bowl. And at the end he pointed out, `I
will train you to take care of your bowl, Ajahn Sumedho will teach you
to reach Nibbāna.' In this way he was both teaching me and indirectly
offering something for Ajahn Sumedho, who was within hearing range, 
something to learn from. 

Later when I was already in Thailand, I awaited the occasion to
undertake the full training as a \emph{bhikkhu}. This is called the
ceremony of \emph{upasampadā}. By that time in early 1980, I had been
both a postulant and a novice for longer than almost any foreign to
Thailand person. There were four young novices awaiting the confirmation
of the date. But Ajahn Chah would not give us a date. On a number of
occasions we would go to his monastery from where we were living, all
prepared, all ready, and ask, `When will the ceremony be arranged?' and
he would always just put us off. And then one evening he said, `Go and
prepare the hall, we'll do it tonight.' This was at around five o'clock. 
So we did, and as the evening fell in the tropics in a monastery without
electricity, the ceremony took place. Simple and easy. With no fuss. It
happened to be my birthday. To this day I do not know if Ajahn Chah knew
or, if he did know, whether he thought it at all important. 

\questionBi%
{Revenons maintenant, si vous voulez bien, sur cette tradition des Moines de la Forêt que vous représentez\ldots{} Est-ce que vous pouvez nous rappeler quand elle est née, et dans quelles circonstances\ldots{}?}%
{Please let us come back to this `Forest Monk' Tradition that you represent. Could you remind us when it was born and in what circumstances?}

\answer{}
The Forest Monk tradition is not confined to Thailand and would have
existed in some form from the time of the Lord Buddha.

\questionBi%
{Est-ce qu'on peut définir la particularité de cette tradition?}%
{Can we define the specificities of this tradition?}

\answer{}
The particular branch of the Forest Tradition to which I belong, Wat
Pah Pong, is distinguished by its working as a community. It works
together. It seems to be Ajahn Chah's great offering, the offering of
encouraging people to work together in community. He used the Vinaya, 
the Training in Community, as received from the traditional scriptures, 
and worked out how to allow ordinary people to use that training to
learn to live and work together. 

Often when a great teacher dies the disciples go their own way. You may
have heard that Ajahn Chah was paralyzed and did not speak for about the
last ten years of his life. This was certainly difficult for all of us, 
those who called themselves his disciples. The effect was that we all had
to learn to work together. There were five people nursing him all the
time of his illness. The monks took turns. They had to work together. 

Today the group consists of maybe 1,500 monastics, with about 270
monasteries. Of those monastics, about 150 would not call Thailand their
origin. There are about 20 monasteries not run by Thais which would
look to Wat Pah Pong and that tradition. I think around 17 of
those monasteries are not in Thailand. 

The monasteries vary in size from maybe two monastics, or even one, to
maybe 50. Outside Thailand the largest in number of monastics is
probably Amaravati Buddhist Monastery.\footnote{\href{http://forestsangha.org/monasteries}{www.forestsangha.org/monasteries}}
We all consider ourselves part of this family.

\questionBi%
{Le maître a un rôle très important dans cette tradition\ldots}%
{The teacher has a very important role in this tradition \ldots}

\answer{}
The teacher has an important role, yes. Like the father.

\questionBi%
{Quelles sont les règles qu'un moine de la forêt doit observer?}%
{What are the rules a forest monk should observe?}

\answer{}
There are four which if not observed, automatically, with immediate
effect and without ceremony, disqualify a man from continuing the
training.

\begin{enumerate}
  \item Any sexual intercourse
  \item Theft of something of value
  \item Shortening or causing to be shortened the life of any human.
  \item Lying deliberately to claim that one has attained some special spiritual level.
\end{enumerate}

The particular rule of our tradition and family, which is common to all
Buddhist monks with a connection to the training at the time of the Lord
Buddha, is not to own or control personal money or that which counts as
money.

\questionBi%
{Comment la communauté des moines de la forêt s'est-elle créée puis a-t-elle évolué en Angleterre?}%
{How was the Forest Sangha created in England and how did it develop?}

\answer{}
The Forest Sangha has evolved in the UK through the confidence of and
confidence in the disciples of Ajahn Chah, particularly Ajahn Sumedho. 
He has allowed communities of monastics to grow in the UK. The
confidence has been that if the monastics are living in accordance with
what can be shown to be the teaching of the Lord Buddha, then there will
be enough generosity to support that life. The places where monastics
live in community can be like generators: generators of generosity (they
exist because those who live there are generous in their lives and what
they offer, asking for nothing, and those who support that life are generous
in offering that support of material things); generators of virtue (the
places encourage virtue in those who live there and, through example and
direct teaching, encourage virtue in those who visit); and generators of
an attitude which cultivates reflection or wisdom (they are places where
people meditate, examine their universe from the inside and practise
being enlightened). 

\questionBi%
{Quelles ont été les principales difficultés?}%
{What were the main difficulties?}

\answer{}
The principal difficulty is that of attachment to opinions and views,
and the pain that follows.

\questionBi%
{Vous avez participé, au début des années 80, à l'établissement du monastère d'Amaravati, en Angleterre\ldots{} le premier monastère de forêt en Occident\ldots{} Est-ce que cela n'a pas été trop compliqué? Les réactions ont-elles été favorables?}
{You took part at the beginning of the 80's in the establishment of Amaravati, the first forest monastery in the West. Wasn't that too complicated? Was the public response positive?}

\answer{}
Amaravati was not the first forest monastery, that was Chithurst
Forest Monastery, Cittaviveka.\footnote{\href{http://cittaviveka.org}{www.cittaviveka.org}}

Yes, beginning any monastery is difficult. Not much more difficult in
the West than in the East. A little different. When something a little
out of the ordinary arrives somewhere, there are always a variety of
responses. The main concern seemed to be traffic. Would the place
attract more cars? There were other worries that the strangeness of the
clothing and customs would somehow undermine what was already there. 
Again a clinging to views and opinions. 

\questionBi%
{Quelles sont les grandes différences avec la communauté des moines de la forêt en Thaïlande?}%
{What are the main differences between the Western Sangha and the Thai one?}

\answer{}
The main difference is maybe the timing of the meal. In Thailand it
is long established that the meal is around 08.30 to maybe 09.00, and
that will be the meal for the day. Outside Thailand the meal is usually
a little later, maybe 10, 10.30 or even 11.30. This is because it is
thought that this will make it easier for people to come to the
monastery to be part of that occasion. Of course there are some
differences of dress to accommodate the differences in climate.

\questionBi%
{Les échanges, les liens entre les deux communautés sont-ils aussi forts aujourd'hui?}%
{Are the links between the two communities still as strong today?}

\answer{}
The links nowadays are still strong, almost stronger in the last few
years than they were in the early 80's and mid-80's. Communication is
now a lot less expensive.

