% Title: Questions and Answers II
% Forest Sangha Newsletter 1993 January

\chapterNote{Issue 23, published in January 1993. The second in a series of extracts from a conversation between Luang Por Chah and a lay Buddhist.}
\chapter{Questions \& Answers II}

\question{}
Does one have to practise and gain \emph{samādhi} (concentration) before one
can contemplate the Dhamma?

\answer{Luang Por Chah}
We can say that's correct from one point of view,
but from the aspect of practice, \emph{paññā} has to come first. In
conventional terms, it's \emph{sīla}, \emph{samādhi} and then
\emph{paññā}, but if we are truly practising the Dhamma, then
\emph{paññā} comes first. If \emph{paññā} is there from the beginning, 
it means that we know what is right and what is wrong; and we know the
heart that is calm and the heart that is disturbed and agitated. 

Talking from the scriptural basis, one has to say that the practice of
restraint and composure will give rise to a sense of shame and fear of
any form of wrong-doing that potentially may arise. Once one has
established the fear of that which is wrong and one is no longer acting
or behaving wrongly, then that which is wrong will not be present within
one. When there is no longer anything wrong present within, this
provides the conditions from which calm will arise in its place. That
calm forms a foundation from which \emph{samādhi} will grow and develop over
time. 

When the heart is calm, that knowledge and understanding which arises
from within that calm is called \emph{vipassanā}. This means that from
moment to moment there is a knowing in accordance with the truth, and
within this are contained different properties. If one was to set them
down on paper they would be \emph{sīla}, \emph{samādhi} and
\emph{paññā}. Talking about them, one can bring them together and say
that these three dhammas form one mass and are inseparable. But if one
were to talk about them as different properties, then it would be
correct to say \emph{sīla}, \emph{samādhi} and \emph{paññā}. 

However, if one was acting in a unwholesome way, it would be impossible
for the heart to become calm. So it would be most accurate to see them
as developing together, and it would be right to say that this is the
way that the heart will become calm. Talking about the practice of
\emph{samādhi:} it involves preserving \emph{sīla}, which includes
looking after the sphere of one's bodily actions and speech, in order
not to do anything which is unwholesome or would lead one to remorse or
suffering. This provides the foundation for the practice of calm, and
once one has a foundation in calm, this in turn provides a foundation
which supports the arising of \emph{paññā}. 

In formal teaching they emphasize the importance of \emph{sīla}. 
\emph{Ādikalyānaṃ, majjhekalyānaṃ, pariyosānakalyānaṃ} -- the practice should
be beautiful in the beginning, beautiful in the middle and beautiful in
the end. This is how it is. Have you ever practised \emph{samādhi}? 

\bigskip

\noindent
I am still learning. The day after I went to see Tan Ajahn at Wat
Keu, my aunt brought a book containing some of your teaching for me
to read. That morning at work I started to read some passages which
contained questions and answers to different problems. In it you said
that the most important point was for the heart to watch over and
observe the process of cause and effect that takes place within; just to
watch and maintain the knowing of the different things that come up. 

That afternoon I was practising meditation, and during the sitting the
characteristic that appeared was that I felt as though my body had
disappeared. I was unable to feel the hands or legs and there were no
bodily sensations. I knew that the body was still there, but I couldn't
feel it. In the evening I had the opportunity to go and pay respects to
Tan Ajahn Tate, and I described to him the details of my experience. He
said that these were the characteristics of the heart that appear when
it unifies in \emph{samādhi}, and that I should continue practising. I
had this experience only once; on subsequent occasions I found that
sometimes I was unable to feel only certain areas of the body, such as
the hands, whereas in other areas there was still feeling.

\question{}
Sometimes
during my practice I start to wonder whether just sitting and allowing
the heart to let go of everything is the correct way to practise; or
else should I think over and occupy myself with the different problems
or unanswered questions concerning the Dhamma which I still have? 

\answer{Luang Por Chah}
It's not necessary to keep going over or adding anything
on at this stage. This is what Tan Ajahn Tate was referring to; one must
not repeat or add onto that which is there already. When that particular
kind of knowing is present, it means that the heart is calm and it is
that state of calm which one must observe. Whatever one feels, whether
it feels like there is a body or a self or not, this is not the
important point. It should all come within the field of one's awareness. 
These conditions indicate that the heart is calm and has unified in
\emph{samādhi}. 

When the heart has unified for a long period a few times, then there
will be a change in the conditions and they say that one `withdraws'. 
That state is called \emph{appanā samādhi} (absorption), and having
entered it, the heart will subsequently withdraw. In fact, although it
would not be incorrect to say that the heart withdraws, it doesn't
actually withdraw. Another way is to say that it flips back, or that it
changes, but the style used by most teachers is to say that once the
heart has reached the state of calm, then it will withdraw. However, 
people get caught up in disagreements over the use of language. It can
cause difficulties and one might start to wonder, `How on earth can it
withdraw? This business of withdrawing is just confusing!' It can lead
to much foolishness and misunderstanding just because of the language. 

What one must understand is that the way to practise is to observe these
conditions with \emph{sati-sampajañña} (mindfulness and clear
comprehension). In accordance with the characteristic of impermanence, 
the heart will turn about and withdraw to the level of \emph{upacāra
samādhi} (access concentration). If it withdraws to this level, one can
gain understanding through awareness of sense impressions and mental
states, because at the deeper level (where the mind is fixed with just
one object) there is no understanding. If there is awareness at this
point, that which appears will be \emph{saṅkhāra} (mental formations). 
It will be similar to two people having a conversation and discussing
the Dhamma together.

One who misunderstands this might feel disappointed
that his heart is not really calm, but in fact this dialogue takes place
within the confines of the calm and restraint which have developed. 
These are the characteristics of the heart once it has withdrawn to the
level of \emph{upacāra} -- there will be the ability to know about and
understand different things. 

The heart will stay in this state for a period and then it will turn
inwards again. In other words, it will turn and go back into the deeper
state of calm where it was before; or it is even possible that it might
obtain purer and calmer levels of concentrated energy than were
experienced before. If it does not reach such a level of concentration, 
one should merely note the fact and keep observing until the time when
the heart withdraws again. Once it has withdrawn, different problems
will arise within the heart. 

This is the point where one can have awareness and understanding of
different things. Here is where one should investigate and examine the
different preoccupations and issues which affect the heart, in order to
understand and penetrate them. Once these problems are finished with, 
the heart will gradually move inwards towards the deeper level of
concentration again. The heart will stay there and mature, freed from
any other work or external impingement. There will just be the
one-pointed knowing, and this will prepare and strengthen one's
mindfulness until the time to re-emerge is reached. 

These conditions of entering and leaving will appear in one's heart
during the practice, but this is something that is difficult to talk
about. It is not harmful or damaging to one's practice. After a period
the heart will withdraw and the inner dialogue will start in that place, 
taking the form of \emph{saṅkhāra} (mental formations) conditioning the
heart. If one doesn't know that this activity is \emph{saṅkhāra}, one
might think that it is \emph{paññā}, or that \emph{paññā} is arising. 
One must see that this activity is fashioning and conditioning the
heart, and the most important thing about it is that it is impermanent. 
One must continually keep control and not allow the heart to start
following and believing in all the different creations and stories that
it cooks up. All that is just \emph{saṅkhāra}, it doesn't become
\emph{paññā}. 

The way \emph{paññā} develops is when one listens and knows the heart
as the process of creating and conditioning takes it in different
directions, and one reflects on the instability and uncertainty of this. 
The realization of its impermanence will provide the cause by which one
can let go of things at that point. Once the heart has let go of things
and put them down at that point, it will gradually become more and more
calm and steady. One must keep entering and leaving \emph{samādhi} like
this, and \emph{paññā} will arise at that point. There one will gain
knowledge and understanding. 

As one continues to practise, many different kinds of problems and
difficulties will tend to arise in the heart; but whatever problems the
world or even the universe might bring up, one will be able to deal with
them all. One's wisdom will follow them up and find answers for every
question and doubt. Wherever one meditates, whatever thoughts come up, 
whatever happens, everything will be providing the cause for
\emph{paññā} to arise. This is a process that will take place by itself, 
free from external influence.

\emph{Paññā} will arise like this, but
when it does, one should be careful not to become deluded and see it as
\emph{saṅkhāra}. Whenever one reflects on things and sees them as
impermanent and uncertain, one shouldn't cling or attach to them in any
way. If one keeps developing this state, when \emph{paññā} is present in
the heart, it will take the place of one's normal way of thinking and
reacting, and the heart will become fuller and brighter in the centre of
everything. As this happens one knows and understands all things as they
really are -- one's heart will be able to progress with meditation in
the correct way and without being deluded. That is how it should be. 

