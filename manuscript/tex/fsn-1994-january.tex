% Title: Jack Kornfield
% Forest Sangha Newsletter 1994 January

\chapterNote{Issue 27, published in January 1994.}
\chapter{Recollections by Jack Kornfield}

\begin{quote}
`I was enormously blessed to meet Ajahn Sumedho in 1967 at
an old ruined Cambodian temple on a mountain-top in Sakolnakorn, 
Thailand. With his inspiration I went to see Ajahn Chah at Wat Pah Pong, 
and eventually entered as a monk in 1969. I left to come back to the USA
in 1972, and was re-ordained to live as a monk with Ajahn Chah for a
time in 1982. Like all of us who were with him, I could tell many more
wonderful stories. Most simply, Ajahn Chah was the wisest man I have ever known, and one of
the most delightful, and it has completely
changed my life to have him for a teacher.'
\end{quote}

\noindent
Ajahn Chah had four basic levels of teaching, and each one, although at
times very difficult for the students, was taught with a lot of humour
and a lot of love. Ajahn Chah taught that until we can begin to respect
ourselves and our environment, practice doesn't really develop. And that
dignity, the ground of practice, comes through surrender, through
impeccable discipline. A lot of us in the West understand freedom to
mean freedom to do what we want, but I think you can see that to follow
the wants of the mind isn't terribly free. It's actually rather
troublesome. 

A deeper freedom, taught through Dhamma, is the freedom within form: the
freedom we can find while relating to another human being, the freedom
of being born in a body with its limitations, and the freedom of a tight
monastic form. What Ajahn Chah did was create a situation of dignity and
demand. He really asked a lot from people, probably more than they'd
ever been asked in their whole life -- to give, to pay attention, to be
wholehearted. Sometimes practice is wonderful: the mind gets so clear
that you smell and taste the air in ways that you haven't since
you were a child. But sometimes it's difficult. He said, `That's not the
point; the point is somehow to come to inner freedom.'

We used to sit for long hours at times, and the meditation hall for the
monks was a stone platform -- they don't use cushions in Asia. You have a
square cloth like a handkerchief that you put down on the stone to sit
on. I remember that when I started, because sitting on the floor was so
painful, I would arrive early at the hall and get a place where I could
sit next to one of the pillars and lean against it. After about a week
of being with Ajahn Chah, he gathered the monks together for an evening
talk after the sitting, and he began to talk about how the true practice
of Dhamma was to become independent in any circumstance; not to need to
lean on things. And then he looked at me. 

Sometimes you would sit while he'd talk to someone or receive visitors, 
and you couldn't leave until you were dismissed. And you'd sit and sit, 
and you'd look at your mind, and it would go, `Doesn't he know that we
are sitting here? Doesn't he know I'm thirsty or I want to get up?' And
he'd be talking away -- he knew very well. And you'd sit and sit and
just see all the movement of the mind. We would sit for hours. The
quality of endurance in the monks' life in the forest, where you just
sit and sit and sit, is a very important one. 

He trusted that people came in order to learn and grow, and when it was
hard, that was all right by him. He didn't care if people had a hard
time. He would go up to them when they were having a hard time and he'd
say, `Are you angry? Whose fault is that, mine or yours?' So one really
had to give up a lot, but it wasn't to him or for him -- it was for
oneself. With surrender and dignity one learned to open up and see
clearly. It is essential in our practice to be unflinchingly honest
about ourselves and the world -- just as he was. 

He would sit under his \emph{kuṭī}, and various lay visitors and other
disciples would come, and also some of his monks would be sitting
around, and he would make fun of people. He'd say, `I'd like to
introduce you to my monks. This one, he likes to sleep a lot. And this
one, he is always sick, his health is his thing; he just spends his time
worrying about his health. And this one is a big eater -- he eats more
than two or three other monks. And this is a doubter over there, he
really likes to doubt, really gets into it. And can you imagine, he had
three different wives at the same time. And this one likes to sit a lot, 
all he does is go and sit in his \emph{kuṭī}; I think he is afraid of
people.' And then he'd point to himself and say, `Myself, I like to play
teacher.'

Once, when he came to the USA, there was a man who had been a monk with
him for a long time who had then disrobed and taken ordination as a Zen
priest. So he said, `I can't figure out this guy', (this man was acting
as his translator), `he is not quite a monk and he is not quite a lay
person. He must be some kind of a transvestite.' And throughout the next
ten days he kept introducing this man as Miss whatever his name was --
Frank or John: `This is Miss John. I'd like you to meet my
transvestite translator. He can't quite make up his mind.' He was very
funny, but he was unstintingly honest. He really could make people look
at themselves and their attachments. When I was translating for him, he
said, `Even though I don't speak any English, I know the truth is that
my translator leaves out all the really hard things I say. I tell you
painful things and he leaves out all the things that have a sting in
them, makes them soft and gentle for you. You can't trust him.'

First come dignity and surrender -- really seeing the power of one's
willingness to live in a full way in the Dhamma. And secondly, one has
to learn to see honestly, to be honest about oneself and the people
around one, to see one's limits and not to be caught in the things
outside. When I asked what is the biggest problem with new disciples, he
said, `Views and opinions about everything. They are all so educated. 
They think they know so much. When they come to me, how can they learn
anything? Wisdom is for you to watch and develop. Take from the teacher
what's good, but be aware of your own practice. If I am resting while
you all sit up, does it make you angry? If I say that the sky is red
instead of blue, don't follow me blindly. One of my teachers ate very
fast and made noises as he ate. Yet he told us to eat slowly and
mindfully. I used to watch him and got very upset. I suffered, but he
didn't. I watched the outside. Later, I learned. Some people drive very
fast but carefully, and others drive slowly and have many accidents. 
Don't cling to rules or to form. If you watch others at the most ten
percent of the time and yourself ninety percent, this is proper
practice. First I used to watch my teacher, Ajahn Tongrat, and had many
doubts. People even thought he was mad. He would do strange things and
be very fierce with his disciples. Outside he was angry, but inside
there was nobody, nothing there. He was remarkable. He stayed clear and
mindful until the moment he died. Looking outside of yourself is
comparing, discriminating; you won't find happiness that way. No way
will you find peace if you spend your time looking for the perfect man, 
or the perfect woman or the perfect teacher.'

The Buddha taught us to look at the Dhamma, the Truth, not to look at
other people, to see clearly and to see into ourselves; to know our
limits. Ram Dass asked him about limits. He asked, `Can you teach if
your own work isn't completed, if you're not fully enlightened?' And he
replied, `Be honest with them. Tell them what you know from your heart
and tell people what's possible. Don't pretend to be able to lift big
rocks when you can only lift small ones. Yet it doesn't hurt to tell
people that if you exercise and if you work, it's possible to lift this. 
Just be straightforward and assess what's truly reasonable.' Surrender, 
and dignity in that, and real impeccability: this is the ground. Then
there's clarity, seeing what's true in oneself, seeing one's limits, 
seeing one's attachments. 

Then the third way he taught was by working with things. 

Working is done in two parts: one by overcoming obstacles and
hindrances, and the other by letting go. Overcoming: the first Dhamma
talk I gave was at a large gathering, \emph{Māgha Pūjā} festival
day, and in a hall filled with 500 or 1,000 villagers. We sat up all
night, alternately sitting for one hour and then listening to a talk
given by one of the teachers from his monasteries. He had several
hundred monks there at that time; they all came together from the branch
monasteries for that day. And then in the middle of the night with no
preparation, he said, `Now we'll hear a Dhamma talk from the Western
monk.' I'd never given a Dhamma talk before, much less in Lao, the local
dialect. There was no time, I had to just get up and say what I could
say. He had his chief Western disciple, Ajahn Sumedho, get up and give a
talk. Ajahn Sumedho ended after an hour and Ajahn Chah said, `Talk
more.' So Ajahn Sumedho talked another half-hour; he didn't have much to
say, people were getting bored, he was getting bored, he finished. Ajahn
Chah said, `Now more.' Another half-hour, three-quarters of an hour, it
was getting more and more boring -- he'd run out of things to say. 
People were sleeping; Ajahn Sumedho didn't know what to say, finally
finished, and Ajahn Chah said, `More, a bit more.' Another half-hour
-- it was the most boring talk! And why would he do it? He got Ajahn
Sumedho to learn not to be afraid of being boring. It was wonderful. 

He encouraged people to put themselves in situations where they were
afraid. He would send people who were afraid of ghosts to sit outside at
night in the charnel-ground. I would go sometimes -- because I wasn't
afraid of ghosts, it was a way of showing off -- but for them it was
really scary. Or he had people go away out in the forest and meditate, 
and face the fear of tigers. The spirit of the practice was to really
make yourself work with things to overcome them. He pushed you into what
you disliked. If you liked to be alone in the forest, you were assigned
to a city monastery in Bangkok. And if you liked the city and the easy
life and good food, he'd send you to some impoverished forest monastery
where there were just rice and tree leaves to eat. He was a real rascal. 
He knew all of your trips, and he could find them and he would somehow, 
in a very funny and gentle and yet direct way, really make you look to
see where you were afraid or attached. Fear, boredom, restlessness --
fine, sit with it. Be bored, be restless and die, he would say over and
over again. Die in that restlessness, die in that fear, die in that
boredom. People were sleepy, great: the ascetic practice he'd assign
would be to sit up all night, and if you wouldn't sit, walk, walk more, 
walk backwards if you were really sleepy. Whatever it took, to really go
against it. 

With anger, restlessness, the same. He said, `You are restless. Fine, go
back and sit. Sit more when you are restless, don't sit less.' He said
it's like starving a tiger to death in a cage of mindfulness. It's not
that you need to do anything about the tiger -- the tiger being your
anger or restlessness -- just let it roam around in the cage. But you
make the cage around it with your sitting. He really made people look at
where they were, made them face it. But still, it was done with humour
and it was done with balance. He wouldn't allow people to do fasts, 
except very rarely. He wouldn't even allow people to do long solitary
practice, unless he felt it was really good for them. Some people he'd
make work. `You need to know the strength of the ox-cart', he would say, 
`and not overload it.' He made space for each person to grow at their
own pace. The first part of working was really working to overcome
difficulties. He said, `The way of Dhamma is the way of opposites. If
you like it cold you should have it hot, and if you like it soft, take
it hard.' Whatever it was, to be really willing to let go, to be free. 

The second part of working was by the practice of real mindfulness, of
being aware of things and letting go of them. In terms of form, this
meant to let go of attachments to physical possessions. `Letting go', 
however, also included matters of custom. I remember the villagers came
to complain to him because he'd set up what still exists as a monastery
for training Westerners, and these Westerners were celebrating
Christmas, with a Christmas tree and all. The villagers came and said, 
`Listen, you told us we were going to have a forest monastery for
Buddhist monks by our village, and these Westerners are doing Christmas. 
It doesn't seem right.' So he listened to them and said, `Well, my
understanding is that the teachings of Christianity are the teachings of
loving-kindness, of surrender and compassion, of seeing one's neighbour
as oneself, of sacrifice, of non-attachment -- many of the basic
principles of Buddha-Dhamma. For me, it seems all right that they
celebrate Christmas, especially since it is a holiday of giving and
generosity, of love. But if you insist, we won't celebrate Christmas
there any more.' The villagers were relieved. He said, `We'll have a
celebration, but instead let's call it ChrisBuddhamas.' And that was the
celebration. They were satisfied, and he was satisfied. 

It wasn't as if the way to do it was through some particular form, but
to let go of form, to let go of doubt. He said, `You have to learn to
watch doubts as they arise. Doubting is natural; we all start off with
doubts. What's important is that you don't identify with them or get
caught up in endless circles. Instead, simply watch the whole process of
doubting. See how doubts come and go. Then you will no longer be
victimized by them.' To see them, to know them, to let go. The same with
judgement and fear -- to feel them, to experience them as physical
events, as mental states and yet not be caught. To eventually come to
see all of the energies -- the difficult ones of anger, fear, 
sleepiness, doubt and restlessness; the subtle ones of our attachment to
pride or to stillness, quietness or even insight. Just to see them and
allow them to come and go, and come to a really profound kind of
equanimity. 

He said, `Sitting for hours on end is not necessary. Some people think
that the longer you can sit, the wiser you must be. I've seen chickens
sitting on their nests for days on end. Wisdom comes from being mindful
in all postures. Your practice should begin as you wake up in the
morning and should continue until you fall asleep. Each person has their
own natural pace. Some of you may die at age 50, some at age 65, some at
age 90. So too, your practice will not be identical. Don't worry about
this. What is important is only that you keep watchful, whether you are
working, sitting or going to the bathroom. Try and be mindful and let
things take their natural course. Then your mind will become quieter and
quieter in any surroundings. It will become still, like a clear forest
pool. Then all kinds of wonderful and rare animals will come to drink at
the pool. You will see clearly the nature of all things in the world. 
You'll see many wonderful and strange things come and go, but you will
be still. This is the happiness and understanding of the Buddha.'

