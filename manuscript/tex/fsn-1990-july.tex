% Forest Sangha Newsletter, 1990 July.

\chapterNote{Issue 13, published in July 1990. Further recollections by those who knew him.}
\chapter{Living with Luang Por}

\subsection{Paul Breiter}

\emph{Formerly Ven. Varapañño, writes of his early contact with Ajahn Chah (c. 1970):}

One cold afternoon as we swept the monastery grounds with long-handled
brooms, I thought how nice it would be, what a simple thing it really
was, if we could have a sweet drink of sugary coffee or tea after
working like that, to warm the bones and give us a little energy for
meditation at night. 

I had heard that Western monks in the forest tend to get infatuated with
sweets, and finally the dam burst for me. One morning on
\emph{piṇḍapat}, from the moment I walked out of the gate of the Wat to
the moment I came back about one and a half hours later, I thought
continually about sugar, candy, sweets, chocolate. Finally I sent a
letter asking a lay supporter in Bangkok to send me some palm-sugar
cakes. And I waited. The weeks went by. One day I went to town with a
layman to get medicine. We stopped by the Post Office and my
long-awaited package was there. It was huge, and ants were already at
it. 

When I got back to the Wat, I took the box to my \emph{kuṭī} and opened
it. There were 20-25 pounds of palm and sugarcane cakes. I went wild,
stuffing them down until my stomach ached. Then I thought I should share
them (otherwise I might get very sick!), so I put some aside and took
the rest to Ajahn Chah's \emph{kuṭī}. He had the bell rung, all the
monks and novices came, and everyone enjoyed a rare treat. 

That night I ate more; and the next morning I couldn't control myself. 
The sugar cakes were devouring me; my blessing started to seem like a
curse. So I took the cakes in a plastic bag and decided to go round the
monks' \emph{kuṭīs} and gave them away. 

For a start I fell down my stairs and bruised myself nicely. The wooden
stairs can get slippery in cold weather, and I wasn't being very mindful
in my guilty, distressed state of mind. 

The first \emph{kuṭī} I went to had a light on inside, but I called and
there was no answer. Finally, after I'd called several times and waited, 
the monk timidly asked who it was (I didn't yet understand how strong
fear of ghosts is among those people). I offered him some sugar, and he
asked me why I didn't want to keep it for myself. I tried to explain
about my defiled state of mind. He took one (it was hard to get them to
take much, as it is considered to be in very bad taste to display one's
desire or anger). 

I repeated this with a few others, having little chats along the way. It
was getting late, and although I hadn't unloaded all the sugar cakes, I
headed back to my \emph{kuṭī}. My flashlight batteries were almost dead, 
so I lit matches to try to have a view of the path -- there were lots of
poisonous things creeping and crawling around in the forest. I ran into
some army ants and experienced my first fiery sting. I got back to my
\emph{kuṭī} feeling very foolish. In the morning I took the rest of the
cakes and gave them to one of the senior monks, who I felt would have
the wisdom and self-discipline to be able to handle them. 

But my heart grew heavy. I went to see Ajahn Chah in the afternoon to
confess my sins. I felt like it was all over for me, there was no hope
left. He was talking with an old monk. I made the customary three
prostrations, sat down and waited. When he acknowledged me, I blurted
out, `I'm impure, my mind is soiled, I'm no good\ldots{}' He looked very
concerned. `What is it?' he asked. I told him my story. Naturally he was
amused, and within a few minutes I realized that he had me laughing. I
was very light-hearted; the world was no longer about to end. In fact, I
had forgotten about my burden. This was one of his most magical gifts. 
You could feel so burdened and depressed and hopeless, and after being
around him for a few minutes it all vanished, and you found yourself
laughing. Sometimes you only needed to go and sit down at his
\emph{kuṭī} and be around him as he spoke with others. Even when he was
away I would get a `contact high' of peacefulness as soon I got near his
\emph{kuṭī} to clean up or to sweep leaves. 

He said, `In the afternoon, when water-hauling is finished, you come
here and clean up.' My first reaction was, `He's got a lot of nerve, 
telling me to come and wait on him.' But apart from being one of my
duties, it was a foot in the door and a privilege. Through it, I was to
start seeing that there was a way of life in the monastery which is
rich, structured and harmonious. And at the centre of it all is the
teacher, who is someone to be relied on. 

Finally, he asked why I was so skinny. Immediately, one of the monks who
was there told him that I took a very small ball of rice at meal-time. 
Did I not like the food? I told him I just couldn't digest much of the
sticky rice, so I kept cutting down. I had come to accept it as the way
it was, thinking I was so greedy that eating less and less was a virtue. 
But he was concerned. Did I feel tired? Most of the time I had little
strength, I admitted. `So', he said, `I'm going to put you on a special
diet for a while -- just plain rice gruel and fish sauce to start with. 
You eat a lot of it, and your stomach will stretch out. Then we'll go to
boiled rice, and finally to sticky rice. I'm a doctor', he added. (I
found out later on that he actually was an accomplished herbalist, as
well as having knowledge of all the illnesses to which monks are prone). 
He told me not to push myself too much. If I didn't have any strength, I
didn't have to carry water, etc. 

That was when the magic really began. That was when he was no longer
just Ajahn Chah to me. He became Luang Por, `Venerable Father'. 

\subsection{Ajahn Munindo}

\emph{A visit from Luang Por:}

There was a very difficult period in my training in Thailand, after I
had already been a monk for about four years. As a result of a motorbike
accident I had had before I was ordained, and a number of years of
sitting in bad posture, my knees seized up. The doctors in Bangkok said
it was severe arthritis, but nothing that a small operation couldn't
fix. They said it would take two or three weeks. But after two months
and three operations I was still hardly walking. There had been all
kinds of complications: scar tissue, three lots of general anaesthetic
and the hot season was getting at me; my mind was really in a state. I
was thinking, `My whole life as a monk is ruined. Whoever heard of a
Buddhist monk who can't sit cross-legged?' Every time I saw somebody
sitting cross-legged I'd feel angry. I was feeling terrible, and my mind
was saying, `It shouldn't be like this; the doctor shouldn't have done
it like that; the monks' rules shouldn't be this way \ldots{}.' It was
really painful, physically and mentally. I was in a very unsatisfactory
situation. 

Then I heard that Ajahn Chah was coming down to Bangkok. I thought if I
went to see him he might be able to help in some way. His presence was
always very uplifting. When I visited him I couldn't bow properly; he
looked at me and asked, `What are you up to?' I began to complain. `Oh, 
Luang Por', I said, `It's not supposed to be this way. The doctors said
two weeks and it has been two months \ldots{}' I was really wallowing. 
With a surprised expression on his face he said to me, very powerfully, 
`What do you mean, it shouldn't be this way? If it shouldn't be
this way, it wouldn't be this way!'

That really did something to me. He pointed to exactly what I was doing
that was creating the problem. There was no question about the fact of
the pain; the problem was my denying that fact, and that was something I
was doing. This is not just a theory. When someone offers us the
reflection of exactly what we are doing, we are incredibly grateful, 
even if at that time we feel a bit of a twit. 

\subsection{Ajahn Sumedho}

\emph{An incident from his early days with Ajahn Chah (c. 1967-69):}

In those days I was a very junior monk, and one night Ajahn Chah took us
to a village fete -- I think Satimanto was there at the time. 

Now, we were all very serious practitioners and didn't want any kind of
frivolity or foolishness; so of course going to a village fete was the
last thing we wanted to do, because in these villages they love
loudspeakers. 

Anyway, Ajahn Chah took Satimanto and I to this village fete, and we had
to sit up all night with all the raucous sounds of the loudspeakers
going and monks giving talks all night long. I kept thinking, `Oh, I
want to get back to my cave. Green skin monsters and ghosts are much
better than this.' I noticed that Satimanto (who was incredibly serious) 
was looking angry and critical, and very unhappy. So we sat there
looking miserable, and I thought, `Why does Ajahn Chah bring us to these
things?' Then I began to see for myself. I remember sitting there
thinking, `Here I am getting all upset over this. Is it that bad? What's
really bad is what I'm making out of it, what's really miserable is my
mind. Loudspeakers and noise, distraction and sleepiness -- all that, 
one can really put up with. It's that awful thing in my mind that hates
it, resents it and wants to leave.'

That evening I could really see what misery I could create in my mind
over things that one can bear. I remember that as a very clear insight
of what I thought was miserable and what really is miserable. At first I
was blaming the people and the loudspeakers, and the disruption, the
noise and the discomfort, I thought that was the problem. Then I
realized that it wasn't -- it was my mind that was miserable. 

\subsection{Sister Candasiri}

\emph{Sister Candasiri first met Luang Por Chah while still a
laywoman, during his second visit to England in 1979:}

For me one of the most striking things about Luang Por Chah was the
effect of his presence on those around him. Watching Ajahn Sumedho --
who hitherto had been for me a somewhat awe-inspiring teacher -- sit at
his feet with an attitude of sheer delight, devotion and adoration
lingers in the mind as a memory of extraordinary sweetness. Ajahn Chah
would tease him, `Maybe it's time for you to come back to Thailand!'
Everyone gasped inwardly: `Is he serious?'

Later on a visitor, a professional flautist, began to ask about music.
`What about Bach? Surely there's nothing wrong with that -- much of his
music is very spiritual, not at all worldly.' (It was a question that
interested me greatly). Ajahn Chah looked at her, and when she had
finished he said quietly, `Yes, but the music of the peaceful heart is
much, much more beautiful.'

\subsection{Ajahn Santacitto}

\emph{Recollecting his own first meeting with Ajahn Chah:}

From the very first meeting with Ajahn Chah, I couldn't help but be
aware of how powerful a force was emanating from this person. I had just
arrived at the monastery with a friend, and neither of us spoke much
Thai, so the possibility of talking with and hearing Dhamma from Ajahn
Chah was very limited. I was considering taking ordination as a monk
mainly in order to learn about meditation, rather than from any serious
inclination towards religious practice. 

It happened that just at that time, a group of local villagers came to
ask him to perform a certain traditional ceremony which involved a great
deal of ritual. The laymen bowed down before the Master, then they got
completely covered over with a white cloth, and then holy water was
brought out and candles were dripped into it, while the monks did the
chanting. And young lad that I was, very science-minded, rather
iconoclastic by nature, I found this all rather startling, and wondered
just what I was letting myself in for. Did I really want to become one
of these guys and do this kind of thing? 

So I just started to look around, watching this scene unfold before me, 
until my eye caught Ajahn Chah's, and what I saw on his face was very
unexpected: there was the smile of a mischievous young man, as if he
were saying, `Good fun, isn't it!' This threw me a bit; I could no
longer think of him as being attached to this kind of ritual, and I
began to appreciate his wisdom. But a few minutes later, when the
ceremony was over and everyone got up and out from under the cloth, all
looking very happy and elated, I noticed that the expression on his face
had changed; no sign of that mischievous young lad. And although I
couldn't understand a word of Thai, I couldn't help but feel very deeply
that quality of compassion in the way he took this opportunity of
teaching people who otherwise might not have been open and susceptible. 
It was seeing how, rather than fighting and resisting social customs
with their rites and rituals, he knew how to use them skilfully to help
people. I think this is what hooked me. 

It happened countless times: people would come to the monastery with
their problems, looking for an easy answer, but somehow, whatever the
circumstances, his approach never varied. He met everybody with a
complete openness, with the `eyes of a babe', as it seemed to me, no
matter who they were. One day a very large Chinese businessman came to
visit. He did his rather disrespectful form of bowing, and as he did so
his sports shirt slipped over his back pocket, and out stuck a pistol. 
Carrying a pistol is about the grossest thing you can do when coming to
see an Ajahn in a Thai monastery! That really took me aback, but what
struck me most of all was that when Ajahn Chah looked at him, there was
that same openness, no difference, `eyes like a babe'. There was a
complete openness and willingness to go into the other person's world, 
to be there, to experience it, to share it with them. 

\subsection{Ajahn Sumedho}

\emph{Recalling an incident during Luang Por's visit to Britain in 1977:}

When Ajahn Chah first visited England, he was invited to a certain
woman's home for a vegetarian meal. She obviously had put a lot of
effort into creating the most delicious kinds of food. She was bustling
about offering this food and looking very enthusiastic. Ajahn Chah was
sitting there assessing the situation, and then suddenly he said: `This
is the most delicious and wonderful meal I have ever had!'

That comment was really something, because in Thailand, monks are not
supposed to comment on the food. And yet Luang Por suddenly manifested
this charming character in complimenting a woman who needed to be
complimented, and it made her feel so happy. He had a feeling for the
time and place, for the person he was with, for what would be kind. He
could step out of the designated role and manifest in ways that were
appropriate; he was not actually breaking any rules, but it was out of
character. Now, that shows wisdom and the ability to respond to a
situation -- not to be just rigidly bound within a convention that
blinds you. 

\subsection{Paul Breiter}

On his visit in 1979, he related that once a Westerner (a layman, I
think) came to Wat Pah Pong and asked him if he was an \emph{arahant}.
Ajahn Chah told him, `Your question is a question to be answered. I will
answer it like this: I am like a tree in the forest. Birds come to the
tree, they will sit on its branches and eat its fruit. To the birds, the
fruit may be sweet or sour or whatever. But the tree doesn't know
anything about it. The birds say `sweet' or they say `sour' -- from the
tree's point of view this is just the chattering of the birds.'

On that same evening we also discussed the relative virtues of the
\emph{arahant} and the \emph{bodhisattva}. He ended our discussion by
saying, `Don't be an arahant. Don't be a Buddha. Don't be anything at
all. Being something makes problems. So don't be anything. You don't
have to be something, he doesn't have to be something, I don't have to
be something \ldots{}' He paused, and then said, `Sometimes when I think
about it, I don't want to say anything.'

