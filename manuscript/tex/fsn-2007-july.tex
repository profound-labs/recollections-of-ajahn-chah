
\chapterNote{Issue 80, published in July 2007.}
\chapter{Thirty Years Later}

\subsection{Questions and answers with Ajahn Sumedho}

\question{}
It's 30 years since you came to England with Luang Por Chah. Why did
you leave Thailand?

\answer{}
In 1975 the Americans left Vietnam, Laos and Cambodia --
French Indochina. Those countries became communist, and there was a
`domino theory' everybody seemed to think would happen: that once those
countries fell, the whole of south-east Asia would follow. There was a
widespread fear that Thailand would be next. We had established Wat Pah
Nanachat for Westerners. I was the head monk and we had about 20
Western monks there at the time, and I remember thinking, `What's going
to happen to us if Thailand goes communist?' So that was the catalyst
that started me thinking about the possibilities of establishing a
Buddhist monastery elsewhere. I'd never entertained such an idea, never
wanted to leave -- but because of this notion that Thailand would fall
to the communists, this thought came into my mind. 

Shortly after that I was invited home because my mother was very ill and
they thought she might die, and it seemed to coincide with having that
thought. So when I went back to see my mother and father I thought, 
`Well, if people are interested maybe we could set something up.' I
spent time with my parents in Southern California, and after my mother
seemed to get better I went on with Ven. Varapañño (Paul Breiter) to New
York and stayed with his parents. I went to Buddhist groups in
Massachusetts, where Jack Kornfield and Joseph Goldstein had just opened
the Insight Meditation Society. It was clear that was not to be a
monastic place. So nothing much happened in the States with respect to
people being interested in starting a monastery. 

To get back to Thailand I had to go via London, and that's where I met
George Sharp. He was the chairman of the English Sangha Trust (EST) and
he seemed very interested in me. I stayed at the Hampstead Vihāra, which
was closed; he opened it up for me. During the three days I was there he
came every evening to talk to me. Then he asked if I would consider
living in England, and I said, `Well, I can't really answer that
question, you'll have to ask my teacher, Luang Por Chah, in Thailand.'
And this he did; he came a few months later. Luang Por Chah and I were
invited to England, and we arrived on 6\textsuperscript{th} May 1977.

\question{}
And your idea was that if Thailand fell to the communists, this
would be a way of preserving this monastic tradition? 

\answer{}
Yes. And the thing that impressed me was that the English
Sangha Trust had already been established 20 years before, in 1956, and
though it had tried all kinds of things, it was essentially a trust set
up to support Buddhist monks in England -- so it was for the Sangha.
There was a movement to try to make it more a trust for supporting lay
teachers. But George Sharp had this very strong sense that the original
purpose of the EST was to encourage Buddhist monks to come and live in
England. Several years before he'd met Tan Ajahn Maha Boowa and Ajahn
Paññavaddho when they came to visit London. He consulted with them about
how to bring good monks to start a proper Sangha presence in England, 
and Ajahn Maha Boowa recommended they just wait, not do anything and see
what happened. So George had closed the Hampstead Vihāra until the right
opportunity arose. He wasn't prepared to put just anybody in there. I
think he saw me as a potential incumbent. Ajahn Chah was very successful
in training Westerners, and in inspiring Western men to become monks. 
Wat Pah Nanachat was really quite a work of genius at the time. There'd
been nothing like it. That was Luang Por Chah's idea. 

\question{}
What did he think about the idea of moving out of Thailand?

\answer{}
When I went back to Thailand I told him about it, and of
course he never signified one way or the other in situations like that. 
He seemed interested, but didn't feel a great need to do anything with
it. That's why it was important for George Sharp to visit, so that Luang
Por Chah could meet him. George was very open to any suggestions that
Ajahn Chah had. He had no agenda of his own, but he was interested in
supporting Theravādan monks living under the Vinaya system in England. 
He'd seen so many failures in England over the previous twenty years; 
there were many good intentions to establish something, but things just
seemed to fall apart. They'd send some Englishmen to Thailand for a
couple of years to become ordained, and when they came back they'd be
thrown straight into a teaching situation or something they weren't
prepared for. They had no monastic experience except maybe a short time
in a Bangkok temple. So what impressed George was that by that time I'd
had quite a few years of training within the monastic system of Thailand
and in the Thai Forest tradition, so I wasn't just a neophyte --
although in terms of the way we look at things now, when I came to
England I had only ten \emph{Vassa}. I don't think any ten-\emph{vassa}
monk now would consider doing such an operation! Ajahn Khemadhammo came
a couple of weeks before, and then Ajahn Chah and I came together, 
arriving on 6\textsuperscript{th} May. Later Ajahns Ānando and Viradhammo dropped in, 
because they had gone to visit their families in North America. During
that time I suggested they stay, and Ajahn Chah agreed, so they stayed
on with me and there were four of us. 

\question{}
Did Ajahn Chah make a decision at some point, that yes, OK, it
would work? 

\answer{}
Well, when George Sharp came to see him in Thailand Ajahn Chah
put him through a kind of test. He was looking at George, trying to
figure out what sort of person he was. George had to eat the leftovers
at the end of the line, out of old enamel bowls with chips in them and
sitting on the cement floor near the dogs. George was a rather
sophisticated Londoner, but Ajahn Chah put him in that position and he
seemed to accept it. I didn't hear him complain at all. Later on we had
meetings, and George made a formal invitation and Luang Por Chah
accepted, agreeing that he and I would visit London the next May. 

I was curious, because Luang Por Chah was so highly regarded in Thailand
that I wondered how he would respond to being in a non-Buddhist country. 
There's no question of right procedure in Thailand in terms of monastic
protocol, but you can't expect that in other countries. What impressed
me during the time in England was how Luang Por responded to the
situation. Nothing seemed to bother him. He was interested, he was
curious. He watched people to see how they did things. He wanted to know
why they did it like this or that. He wasn't threatened by anything. He
seemed to just flow with the scene and be able to adapt skilfully to a
culture and climate he'd never experienced before in his life, living in
a country where he couldn't understand what anyone was saying. 

He could relate well to English people, even though he couldn't speak a
word of English; his natural warmth was enough. He was a very
charismatic person in his own right, whether he was in Thailand or in
England, and he seemed to have pretty much the same effect on people, 
whoever they were. 

Every morning we went out on alms-round to Hampstead Heath. People would
come, Thai people -- and Tan Nam and his wife, that's where we met them. 
They've been supporting us all these years. Generally our reception was
excellent. George Sharp's idea was to develop a forest monastery. He
felt that the Hampstead Vihāra was a place that could not develop. It
was associated with a lot of past failures and disappointment, so his
idea was to sell it off in order to find some place in the countryside
that would be suitable for a forest monastery. Luang Por Chah said to
stay at the Hampstead Vihāra first, to see what would happen. And it was
good enough in the beginning. But the aim was always to move out of
there, to sell it off and find a forest. 

\question{}
Did you feel confident that it would work? What were your
feelings at that time, after Luang Por Chah left? 

\answer{}
I didn't know what was going to happen, and I wasn't aware of the
kinds of problems I was moving into, with the state of the English
Sangha Trust. I was quite naïve really. But I appreciated George Sharp's
efforts and intentions, and the legal set-up seemed so good: a trust
fund that had been established for supporting the Sangha. George seemed
to have a vision of this, rather than seeing us as meditation teachers
or just using us to spread Buddhism in Europe. I never got that
impression from him; in fact he made it very clear that if I just came
and practised meditation they'd support that, without even any
expectation of teaching. So right from the beginning it was made clear
that I wasn't going to be pushed around or propelled by people to fulfil
their demands and expectations. It seemed like quite a good place to
start outside of Thailand.

But when Luang Por Chah left -- he was only there for a month -- he made
me promise not to come back for five years. He said, `You can't come
back to Thailand for five years.'

\question{}
So he believed in the project at that point?

\answer{}
He seemed to. He was quite supportive in every way. So I said
I would do that, and I planned to stay.

\subsection{George Sharp}

I think it was in June 1976 when the phone rang and it was Ajahn
Sumedho. He had been given my telephone number by Ajahn Paññavaddho in
Thailand, who suggested he should give me a ring if he needed any
assistance. Principally he rang to say, `Could you give me a place that
is suitable for me to stay in for a few days?' I said: `Okay, I'll send
a taxi for you', which I did, and he arrived in no time. He was there
altogether about three days.

I had work to do, but in the evenings we would talk for hours. He told
me something about the tradition. I was very interested, and in the end
he said: `I invite you to come to Thailand and meet my teacher.' I said
I would, and in November of that year I got on a plane and went. 

I thought Ajahn Chah might agree to having a go at starting a branch in
England, and I suspected he had a great deal of confidence in Ajahn
Sumedho. In fact, on one occasion when Ajahn Sumedho was translating, I
said to Ajahn Chah: `This is really quite a venture and, quite frankly, 
Venerable Sumedho is going to have a very tough time at getting this
started. Now, I don't know anything about Venerable Sumedho. He comes to
me without any reputation whatsoever. But on the other hand, Ajahn Chah, 
you are a great teacher, you have a considerable reputation and with
such a reputation this venture might have a chance of getting off the
ground. What can you tell me to give me confidence in the Venerable
Sumedho?' Ajahn Sumedho had to translate all this. Ajahn Chah said, `I
don't think he'll get married.' That was terrific, because that is what
all the previous \emph{bhikkhus} had been doing at the Hampstead Vihāra.

I came home knowing that Ajahn Chah was coming over and that he was
going to bring four \emph{bhikkhus} with him. So what he was effectively doing was
bringing a Sangha to England. They were going to have a look at
Haverstock Hill, and he was going to make up his mind whether it was
worth a go or not. That is more or less what happened. He simply came in
and took over the place. In the end he apparently said they were to
stay. 

\subsection{Ajahn Sundara}

I started cooking at 8 a.m. in the kitchen of the Hampstead Vihāra, to
serve the main meal at 10.30. \emph{Anāgārikas} Phil (Ajahn Vajiro) and
Jordan (Ajahn Sumano) watched me prepare my favourite dishes and gave me
clues on how to go about in a place that was for me still very strange. 
I was quite intent on my cooking, I wanted it perfect! After presenting
the whole meal to the monks, I felt so nervous and self-conscious that I
just ran downstairs and left! I had no idea of the Buddhist customs of
chanting a blessing, sharing food, etc. 

When I first heard Ajahn Sumedho talk about his life as a forest monk in
Thailand I was stunned, because for a long time I had imagined a way of
life that would include all the qualities he was describing: where
intelligence and simplicity, patience and vitality, humour and
seriousness, being a fool and being wise, could all happily coexist. 
During a later conversation he said, `It is a matter of knowing where
the world is, isn't it?' The penny had dropped: `I am the world!' I had
read and heard this truth many times, but I was truly hearing it for the
first time. That's when I decided to give monastic life a try, not
motivated by the desire to become a nun, but to learn and put into
practice the teaching of the Buddha. I had found my path. 

\subsection{Ajahn Vajiro}

`Forest \emph{bhikkhus} in London', that's what I heard. I was excited by the
news. I bicycled from south of the river, up Haverstock Hill to number
131, a terraced house opposite the Haverstock Arms. The shrine room on
the second floor was as big as could be made from one floor of the
house. When Ajahn Chah was there the room was over-full, cramped and
stuffy. The talks were long and riveting. Tea was served in the basement
afterwards. 

I was particularly struck by the way the \emph{bhikkhus} related to each
other, and especially how they related to Ajahn Chah. There was a
quality of care and attention which I found beautiful. I can remember
thinking, `I'll NEVER bow', and within a few weeks of watching and
listening, asking Ajahn Sumedho to teach me how to bow. 

When I went to live at the Hampstead Vihāra in early 1978, the place was
physically cramped, crowded and chaotic. It was not unusual for six men
to be sleeping in the shared \emph{anāgārikas} and laymen's room on the top
floor. There were two WC's in the main building, one shower, a tiny
kitchen and the small basement room next to the kitchen served as the
\emph{dāna sāla}. What kept us there enduring the physical conditions
was the quality of the Dhamma. The \emph{pūjās} were early in the
morning and included a reflection nearly every day. And with the evening
\emph{pūjās}, talks again were almost daily. 

The main reflection was on uncertainty. There was a confidence that
things would change, and a trust that if the cultivation of
\emph{pāramīs} was sincere, the change would be blessed. 

