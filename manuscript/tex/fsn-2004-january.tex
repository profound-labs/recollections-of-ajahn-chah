% Title: Some Final Words
% Forest Sangha Newsletter

\chapterNote{Issue 67, published in January 2004.}
\chapter{Some Final Words}

\emph{Extract from a talk given by Luang Por Chah to a large gathering
of monks and laypeople at Wat Nong Pah Pong, recently translated by Paul
Breiter.}

In every home and every community, whether we live in the city, the
countryside, the forests or the mountains, we are the same in
experiencing happiness and suffering; but very many of us lack a place
of refuge, a field or garden where we can cultivate positive qualities
of heart. We don't have clear understanding of what this life is about
and what we ought to be doing. From childhood and youth until adulthood, 
we learn to seek enjoyment and take delight in the pleasures of the
senses, and we never think that danger will threaten us as we go about
our lives, forming a family and so on. 

There is also for many of us an inner lack of virtue and Dhamma in our
lives, through not listening to the teachings and not practising Dhamma. 
As a result there is little wisdom in our lives, and everything
regresses and degenerates. The Buddha, our Supreme Teacher, had
loving-kindness (\emph{mettā}) for beings. He led sons and daughters of
good family to ordain, practise and realize the truth. He taught them to
establish and spread the teaching, and to show people how to live with
happiness in their daily lives. He taught the proper ways to earn a
livelihood, to be moderate and thrifty in managing finances and to act
without carelessness in all affairs. 

The Lord Buddha taught that no matter how poor we may be, we should not
let it impoverish our hearts and starve our wisdom. Even if there are
floods inundating our fields, our villages and our homes to the point
where it is beyond our capability to save anything, the Buddha taught us
not to let the floods overcome our hearts. Flooding the heart means that
we lose sight of and have no knowledge of Dhamma. 

Even if water floods our fields again and again over the years, or even
if fire burns down our homes, we will still have our minds. If our minds
have virtue and Dhamma, we can then use our wisdom to help us make a
living and support ourselves. We can acquire land again and make a new
start. 

I really believe that if you listen to the Dhamma, contemplating it and
understanding it, you can make an end of your suffering. You will know
what is right to do, what you need to do, what you need to use and what
you need to spend. You can live your life according to moral precepts
and Dhamma, applying wisdom to worldly matters. Unfortunately, most of
us are far from that. 

We should remember that when the Buddha taught Dhamma and set out the
way of practice, he wasn't trying to make our lives difficult. He wanted
us to improve, to become better and more skilful. It's just that we
don't listen. This is pretty bad. It's like a little child who doesn't
want to take a bath in the middle of winter because it's too cold. He
starts to stink so much that the parents can't even sleep at night, so
they grab hold of him and give him a bath. That makes him mad, and he
cries and curses his father and mother. The parents and the child see
the situation differently. For the child, it's too uncomfortable to take
a bath in the winter. For the parents, the child's smell is unbearable. 
The two views can't be reconciled. 

The Buddha didn't simply want to leave us as we are. He wanted us to be
diligent and work hard in ways that are good and beneficial, and to be
enthusiastic about the right path. Instead of being lazy, we have to
make efforts. 

His teaching is not something that will make us foolish or useless. It
teaches us how to develop and apply wisdom to whatever we are doing: 
working, farming, raising a family and managing our finances. If we live
in the world, we have to pay attention and know the ways of the world, 
otherwise we end up in dire straits. When we have our means of
livelihood, our homes and possessions, our minds can be comfortable and
upright, and we can have the energy of spirit to help and assist each
other. If someone is able to share food and clothing and provide shelter
to those in need, that is an act of loving-kindness. The way I see it, 
giving things in a spirit of loving-kindness is far better than selling
them to make a profit. Those who have \emph{mettā} don't wish for
anything for themselves. They only wish for others to live in happiness. 

When we live according to Dhamma, we feel no distress when looking back
on what we have done. We are only creating good \emph{kamma}. If we are
creating bad \emph{kamma}, then the result later on will be misery. So
we need to listen and contemplate, and we need to figure out where
difficulties come from. Have you ever carried things to the fields on a
pole over your shoulders? When the load is too heavy in front, isn't
that uncomfortable to carry? When it's too heavy behind, isn't that
uncomfortable to carry? Which way is balanced and which way is
unbalanced? When you're doing it well, you can see it. Dhamma is like
that. There is cause and effect -- it is common sense. When the load is
balanced it's easier to carry. With an attitude of moderation our family
relations and our work will be smoother. Even if you aren't rich, you
will still have ease of mind; you won't need to suffer over them. 

As we haven't died yet, now is the time to talk about these things. If
you don't hear Dhamma when you are a human being, there won't be any
other chance. Do you think animals can be taught Dhamma? Animal life is
a lot harder than ours, being born as a toad or a frog, a pig or a dog, 
a cobra or a viper, a squirrel or a rabbit. When people see them they
only think about killing or beating them, or catching or raising them
for food. So we have this opportunity only as humans. As we're still
alive, now is the time to look into this and mend our ways. If things
are difficult, try to bear with the difficulty for the time being and
live in the right way, until one day you can do it. This is the way to
practise Dhamma. 

So I am reminding you all of the need to have a good mind and live your
lives in an ethical way. However you may have been doing things up to
now, you should take a look and examine to see whether what you are
doing is good or not. If you've been following wrong ways, give them up. 
Give up wrong livelihood. Earn your living in a good and decent way that
doesn't harm others and doesn't harm yourself or society. When you
practise right livelihood, then you will live with a comfortable mind. 

We should use our time to create benefit right now, in the present. This
was the Buddha's intention: benefit in this life, benefit in future
lives. In this life, we need to apply ourselves from childhood to study, 
to learn at least enough to be able to earn a living, so that we can
support ourselves and eventually establish a family and not live in
poverty. But we sometimes lack this responsible attitude. We seek
enjoyment instead. Wherever there's a festival, a play or a concert, 
we're on our way there, even when it's getting near harvest time. The
old folks will drag the grandchildren along to hear the famous singer. 
`Where are you off to, Grandmother?' `I'm taking the kids to hear the
concert!' I don't know if Grandma is taking the kids or the kids are
taking her. It doesn't seem to matter how long or difficult a trip it
might be, they go again and again. They say they're taking the
grandchildren, but the truth is that they just want to go themselves. To
them, that's what a good time is. If you invite them to the monastery to
listen to Dhamma, to learn about right and wrong, they'll say, `You go
ahead. I want to stay home and rest\ldots{} I've got a bad
headache\ldots{} my back hurts\ldots{} my knees are sore\ldots{} I
really don't feel well\ldots{}.' But if it's a popular singer or an
exciting play, they'll hurry to round up the kids. Nothing bothers them
then. That's how some folk are. They make such efforts, yet all they do
is bring suffering and difficulty on themselves. They seek out darkness, 
confusion, and intoxication on the path of delusion. 

The Buddha teaches us to create benefit for ourselves in this life, 
ultimate benefit, spiritual welfare. We should do it now, in this very
life. We should seek out the knowledge that helps us do it, so that we
can live our lives well, making good use of our resources, working with
diligence in ways of right livelihood. 

The Buddha taught us to meditate. In meditation we must practise
\emph{samādhi}, which means making the mind still and peaceful. It's
like water in a basin. If we keep putting things in it and stirring it
up, it will always be murky. If the mind is always allowed to be
thinking and worrying over things, we will never see anything clearly. 
If we let the water in the basin settle and become still, then we will
see all sorts of things reflected in it. When the mind is settled and
still, wisdom will be able to see things. The illuminating light of
wisdom surpasses any other kind of light. 

When training the mind in \emph{samādhi}, we initially get the idea it
will be easy. But when we sit, our legs hurt, our back hurts, we feel
tired, we get hot and itchy. Then we start to feel discouraged, thinking
that \emph{samādhi} is as far away from us as the sky from the earth. We
don't know what to do and become overwhelmed by the difficulties. But if
we receive some training, it will get easier little by little. 

It's like a city person looking for mushrooms. He asks, `Where do
mushrooms come from?' Someone tells him, `They grow in the earth.' So he
picks up a basket and goes walking into the countryside, expecting the
mushrooms to be lined up along the side of the road for him. But he
walks and walks, climbing hills and trekking through fields, without
seeing any mushrooms. A village person who has gone picking mushrooms
before would know where to look for them; he would know which part of
which forest to go to. But the city person has had only the experience
of seeing mushrooms on his plate. He heard they grow in the earth and
got the idea that they would be easy to find, but it didn't work out
that way. 

Likewise, you who come here to practise \emph{samādhi} might feel it's
difficult. I had my troubles with it too. I trained with an Ajahn, and
when we were sitting I'd open my eyes to look: `Oh! Is Ajahn ready to
stop yet?' I'd close my eyes again and try to bear it a little longer. I
felt it was going to kill me. I kept opening my eyes, but the Ajahn
looked so comfortable sitting there. One hour, two hours; I would be in
agony, but the Ajahn didn't move. So after a while I got to fear the
sittings. When it was time to practise \emph{samādhi}, I'd feel afraid. 

When we are new to it, training in \emph{samādhi} is difficult. Anything
is difficult when we don't know how to do it. This is our obstacle. But
with training, this can change. That which is good can eventually
overcome and surpass that which is not good. We tend to become
faint-hearted as we struggle -- this is a normal reaction, and we all go
through it. So it's important to train for some time. It's like making a
path through the forest. At first it's rough going, with a lot of
obstructions, but by returning to it again and again, we clear the way. 
After some time, when we have removed the branches and stumps, the
ground becomes firm and smooth from being walked on repeatedly. Then we
have a good path for walking through the forest. This is what it's like
when we train the mind. By keeping at it, the mind becomes illumined. 

So the Buddha wanted us to seek Dhamma. This kind of knowledge is what's
most important. Any form of knowledge or study that does not accord with
the Buddhist way is learning that involves \emph{dukkha}. Our practice
of Dhamma should get us beyond suffering; if we can't fully transcend
suffering, then we should at least be able to transcend it a little, 
now, in the present. 

When problems come to you, recollect Dhamma. Think of what your
spiritual guides have taught you. They have taught you to let go, to
give up, to refrain, to put things down; they have taught you to strive
and fight in a way that will solve your difficulties. The Dhamma that
you come to listen to is for solving problems. The teaching tells you
that you can solve the problems of daily life with Dhamma. After all, we
have been born as human beings; it should be possible for us to live
with happy minds. 

