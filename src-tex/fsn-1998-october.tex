% Title: Ajahn Chah's Birthday
% Forest Sangha Newsletter 1998 October

\chapterNote{Issue 46, published in October 1998.}
\chapter{Ajahn Chah's Birthday}

\begin{quote}\itshape
Ajahn Viradhammo, who was visiting Thailand, passed on a
letter that he had written to the New Zealand Sangha about the
celebration of Ajahn Chah's birthday at Wat Pah Pong some time before
Luang Por passed away.
\end{quote}

\noindent
The birthday celebrations at Wat Nong Pah Pong were a magnificent
tribute to Luang Por. There were over 600 \emph{bhikkhus} and
\emph{sāmaṇeras}, and a sea of white-robed nuns and laypeople around his
\emph{kuṭī} on the afternoon of the 16\textsuperscript{th}. Thānavaro,
you will remember where you sat when Luang Por was brought outside in
his wheelchair. That grassy area was almost entirely filled with the
ochre robe.

We bowed in unison and then Ajahn Mahā Amon led the chanting, `\emph{Mahā There
pamādena} \ldots{}' To my surprise Luang Por's voice answered back (they played
a tape over the public address system) `\emph{Yathā vārivahā} \ldots{}' Luang
Por continued to sit in his chair (he has no choice), and although I couldn't
see his face clearly I'm sure he put tremendous effort forth to acknowledge our
devotion and gratitude. All of this was of course very moving.

After some time we once again bowed in unison and Luang Por was taken
back to his bed for the past six years. In the evening we had chanting
and discourses through the night. The \emph{sāla} was overflowing with
laypeople and \emph{bhikkhus}, with many sitting outside. The
north-east of Thailand is a very special place where so many people
still practise and live their religion with tremendous devotion and
sincerity.

After midnight it started to rain and by dawn there was water all around
 (it has been an exceptionally wet year). In the morning before
\emph{piṇḍapat} there was a \emph{dāna} offering of bowls, \emph{glots},
mosquito netting and white cloth to all of the senior monks of over
eighty branch monasteries. Just to make sure that there were enough sets
of requisites, the laypeople from Bangkok put together 108 sets. The
abundance and volume of Buddhist devotion and generosity are astounding.

After this \emph{Mahā Sangha} offering we had a \emph{piṇḍapat} around
Luang Por's museum. The line of \emph{bhikkhus} stretched from the old
\emph{sāla} to the museum. There was mud everywhere and a seemingly
endless circle of laypeople offering rice into our bowls. When the meal
finally got under way there were six lines of monks and novices outside
the length of the eating hall, and a crammed two lines inside. After the
meal there were a few more formalities, parting words, and soon all the
visitors began to return to their respective monasteries all over
Thailand. This tribute was over and I wished you could both have been
here with me. 

Luang Por's condition is uncertain, although most people say he is
weaker. The most notable difference from last year is in his eyes. The
pupils are rolled upwards and there is no longer any attention in his
eyes. One of the nursing monks said that sometimes he does focus his
eyes and look at what is around him, but this is more and more rare. 
Whatever his physical condition, the power of his practice and teaching
is unmistakable. Equally impressive is the continuing dedication people
have to his way. There is much work to be done, and Luang Por's
impeccability forces one's attention inwards to the source of both
freedom and suffering. 

