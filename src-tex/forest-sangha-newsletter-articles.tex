
\chapter{July 1989}

\section{Gratitude to Ajahn Chah}

June 17th was the 71st birthday of, Venerable Ajahn Chah, spiritual
teacher of over 80 forest monasteries in Thailand, Britain and around
the world. As is customary in the monasteries in England, the day's
practice was offered in gratitude to him and for his well-being. In this
Newsletter we present, through some reflections, an occasion for readers
to recollect what he has made possible for all of us.

\section{Luang Por's Way}

\emph{Venerable Jayasaro was formally abbot at Wat Pah
Nanachat. In 1988 he visited the UK as a translator for Venerable Chao
Khun Paññananda. The following reflections on Ajahn Chah's life are
taken from a talk given at Amaravati Buddhist Centre in June of that
year.}

My own first meeting with Ajahn Chah was on the full moon of December
1978. I had spent the Rains Retreat of that year as an eight-precept lay
person with Ajahn Sumedho at Oakenholt in England. After the retreat I
went out to Thailand. When I arrived at Wat Pah Pong, Venerable Pamutto,
an Australian monk resident there at the time, took me to see Ajahn
Chah. He was sitting under his \emph{kuti} having a drink. He looked at
me and smiled very warmly. He held out the drink he had in his hand, so
I crawled over and took it. As I returned to my place I found there were
tears welling up in my eyes. I was emotionally overcome for quite a
while. Since that day I don't think I have ever wanted to leave the
monastery or do anything except be a disciple of Ajahn Chah.

People often presumed there would be a problem with language for
Westerners who wanted to stay at the monastery, but this was not the
case. Someone once asked Ajahn Chah: `Luang Por, how do you teach all
your Western disciples? Do you speak English or French? Do you speak
Japanese or German?' `No,' replied Ajahn Chah. `Then how do they all
manage?' he asked. `Householder', Ajahn Chah enquired, `at your home do
you have water buffaloes?' `Yes, Luang Por' was the reply. `Do you have
any cows, or dogs, or chickens?' `Yes, Luang Por' `Tell me', Luang Por
asked, `do you speak water-buffalo: do you speak cow?' `No,' the
householder replied. `Well, how do they all manage?'

Language was not so important to Luang Por. He knew how to see through
the exterior trappings of language and culture. He could see how all
minds basically revolve around the same old centres of greed, hatred and
delusion. His method of training was one of pointing directly at the way
our minds work. He was always showing us how craving gives rise to
suffering -- actually allowing us to see the Four Noble Truths directly.
And for him, the way of exposing desires was to frustrate them. In his
vocabulary, the words `to teach' and `to torment' were more or less
interchangeable.

Such training as this can only take place if everyone in the monastery
has great confidence in the teacher. If there is the slightest suspicion
that he might be doing it out of aversion or desire for power, there
wouldn't be any benefit. In Ajahn Chah's case everyone could see that he
had the greatest courage and fortitude, and so could trust that he was
doing it out of compassion.

Primarily he would teach about letting go. But he also taught a lot
about what to do when we can't let go. `We endure', he would say.
Usually people could appreciate intellectually about letting go, but
when faced with obstacles they couldn't do it. The teaching of patient
endurance was a central aspect of the way that he taught. He continually
changed routines around in the monastery so you wouldn't become stuck in
ruts. As a result you kept finding yourself not quite knowing where you
stood. And he would always be there watching, so you couldn't be too
heedless. This is one of the great values of living with a teacher; one
feels the need to be mindful.

In looking into Ajahn Chah's early life, it was inspiring for me to find
just how many problems he had. Biographies of some great masters leave
you with the impression that the monks were perfectly pure from the age
of eight or nine -- that they didn't have to work at their practice. But
for Ajahn Chah practice was very difficult. For one thing, he had a lot
of sensual desire. He also had a great deal of desire for beautiful
requisites, such as his bowl and robes, etc. He made a resolution in
working with these tendencies that he would never ask for anything, even
if it was permitted to do so by the Discipline. He related once how his
robes had been falling to bits; his under-robe was worn paper-thin, so
he had to walk very carefully lest it split. Then one day he heedlessly
squatted down and it tore completely. He didn't have any cloth to patch
it, but remembered the foot-wiping cloths in the Meeting Hall. So he
took them away, washed them and patched his robe with them.

In later times when he had disciples, he excelled in skilful means for
helping them; he had had so many problems himself. In another story, he
related how he made a resolution to really work with sensual desire. He
resolved that for the three-month Rains Retreat he would not look at a
woman. Being very strong-willed, he was able to keep to this. On the
last day of the retreat many people came to the monastery to make
offerings. He thought, `I've done it now for three months, let's see
what happens.' He looked up, and at that moment there was a young woman
right in front of him. He said the impact was like being hit by
lightning. It was then that he realized mere sense restraint, although
essential, was not enough. No matter how restrained one may be regarding
the eyes, ears, nose, tongue, body, and mind, if there wasn't wisdom to
understand the actual nature of desire, then freedom from it was
impossible.

He was always stressing the importance of wisdom, not just restraint,
but mindfulness and contemplation. Throwing oneself into practice with
great gusto and little reflective ability may result in a strong
concentration practice, but one eventually ends up in despair. Monks
practising like this usually come to a point where they decide that they
don't have what it takes to `break through' in this lifetime, and
disrobe. He emphasized that continuous effort was much more important
than making a great effort for a short while, only to let it all slide.
Day in, day out; month in, month out; year, in year out: that is the
real skill of the practice.

What is needed in mindfulness practice, he taught, is a constant
awareness of what one is thinking, doing or saying. It is not a matter
of being on retreat or off retreat, or of being in a monastery or out
wandering on \emph{tudong}; it's a matter of constancy: `What am I doing
now; why am I doing it?' Constantly looking to see what is happening in
the present moment. Is this mind state coarse or refined?' At the
beginning of practice, he said, our mindfulness is intermittent, like
water dripping from a tap. But as we continue, the intervals between the
drips lessen and eventually they become a stream. This stream of
mindfulness is what we are aiming for.

It was noticeable that he did not talk a lot about levels of
enlightenment or the various states of concentration absorption
(\emph{jhāna}). He was aware of how people tend to attach to these terms
and conceive of practice as going from this stage to that. Once someone
asked him if such and such a person was an \emph{arahant} -- was
enlightened. He answered, `If they are then they are, if they're not,
then they're not; you are what you are, and you're not like them. So
just do your own practice.' He was very short with such questions.

When people asked him about his own attainments, he never spoke praising
himself or making any claim whatsoever. When talking about the
foolishness of people, he wouldn't say, `You think like this and you
think like that', or `You do this and you do that.' Rather, he would
always say, `We do this and we do that.' The skill of speaking in such a
personal manner meant that those listening regularly came away feeling
he was talking directly to them. Also, it often happened that people
would come with personal problems they wanted to discuss with him, and
that very same evening he would give a talk covering exactly that
subject.

In setting up his monasteries, he took a lot of his ideas from the great
meditation teacher Venerable Ajahn Mun, but also from other places he
encountered during his years of wandering. Always he laid great emphasis
on a sense of community. In one section of the \emph{Mahāparinibbāna
Sutta}\footnote{** \emph{*}DNikāya* (DN), Sutta 16.} the Buddha speaks
about the welfare of the Sangha being dependent on meeting frequently in
large numbers, in harmony, and on discussing things together. Ajahn Chah
stressed this a lot.

The \emph{Bhikkhu} Discipline (the \emph{Vinaya}) was to Ajahn Chah a
very important tool for training. He had found it so in his own
practice. Often he would give talks on it until one or two o'clock in
the morning; the bell would then ring at three for morning chanting.
Monks were sometimes afraid to go back to their \emph{kutis} lest they
couldn't wake up, so they would just lean against a tree.

Especially in the early days of his teaching things were very difficult.
Even basic requisites like lanterns and torches were rare. In those days
the forest was dark and thick with many wild and dangerous animals. Late
at night you could hear the monks going back to their huts making a loud
noise, stomping and chanting at the same time, On one occasion twenty
torches were given to the monastery, but as soon as the batteries ran
out they all came back into the stores, as there were no new batteries
to replace them.

Sometimes Ajahn Chah was very harsh on those who lived with him. He
admitted himself that he had an advantage over his disciples. He said
that when his mind entered \emph{samādhi} concentration for only 30
minutes, it could be the same as having slept all night. Sometimes he
talked for literally hours, going over and over the same things again
and again, telling the same story hundreds of times. For him, each time
was as if it was the first. He would be sitting there giggling and
chuckling away, and everybody else would be looking at the clock and
wondering when he would let them go.

It seemed that he had a special soft spot for those who suffered a lot;
this often meant the Western monks. There was one English monk,
Venerable Thitabho, to whom he gave a lot of attention; that means he
tormented him terribly. One day there was a large gathering of visitors
to the monastery, and as often happened, Ajahn Chah was praising the
Western monks to the Thais as a way of teaching them. He was saying how
clever the Westerners were, all the things they could do and what good
disciples they were. `All', he said, `except this one,' pointing to
Venerable Thitabho. `He's really stupid.' Another day he asked Venerable
Thitabho, `Do you get angry when I treat you like this?' Venerable
Thitabho replied, `What use would it be? It would be like getting angry
at a mountain.'

Several times people suggested to Ajahn Chah that he was like a Zen
master. `No I'm not', he would say, `I'm like Ajahn Chah.' There was a
Korean monk visiting once who liked to ask him \emph{koans}. Ajahn Chah
was completely baffled; he thought they were jokes. You could see how it
was necessary to know the rules of the game before you could give the
right answers. One day this monk told Ajahn Chah the Zen story about the
flag and the wind, and asked, `Is it the flag that blows or is it the
wind?' Ajahn Chah answered, `It's neither; it's the mind.' The Korean
monk thought that was wonderful and immediately bowed to Ajahn Chah. But
then Ajahn Chah said he'd just read the story in the Thai translation of
Hui Neng.

Many of us tend to confuse complexity with profundity, so Ajahn Chah
liked to show how profundity was in fact simplicity. The truth of
impermanence is the most simple thing in the world, and yet it is the
most profound. He really emphasized that. He said the key to living in
the world with wisdom is a regular recollection of the changing nature
of things. `Nothing is sure,' he would constantly remind us. He was
always using this expression in Thai -- `\emph{Mai nair}!' -- meaning
`uncertain'. He said this teaching, `It's not certain', sums up all the
wisdom of Buddhism. He emphasized that in meditation, `We can't go
beyond the hindrances unless we really understand them.' This means
knowing their impermanence.

Often he talked about `killing the defilements', and this also meant
`seeing their impermanence'. `Killing defilements' is an idiomatic
expression in the meditative Forest Tradition of north-east Thailand. It
means that by seeing with penetrative clarity the actual nature of
defilements, you go beyond them.

While it was considered the `job' of a \emph{bhikkhu} in this tradition
to be dedicated to formal practice, that didn't mean there wasn't work
to do. When work needed doing you did it. And you didn't make a fuss.
Work is not any different from formal practice if one knows the
principles properly. The same principles apply in both cases, as the
same body and mind are active. And in Ajahn Chah's monasteries, when the
monks worked, they really worked. One time he wanted a road built up to
Wat Tum Saeng Pet mountain monastery, and the Highways Department
offered to help. But before long they pulled out, so Ajahn Chah took the
monks up there to do it. Everybody worked from three o'clock in the
afternoon until three o'clock the next morning. A rest was allowed until
just after five, when they would head off down the hill to the village
on almsround. After the meal they could rest again until three, before
starting work once more. But nobody saw Ajahn Chah take a rest; he was
busy receiving people who came to visit. And when it was time for work
he didn't just direct it. He joined in the heavy lifting, carrying rocks
alongside everyone else. That was always very inspiring for the monks to
see: hauling water from the well, sweeping and so on, he was always
there, right up until the time when his health began to fail.

Ajahn Chah wasn't always popular in his province in north-east Thailand,
even though he did bring about many major changes in the lives of the
people. There was a great deal of animism and superstition in their
belief systems. Very few people practised meditation, out of fear that
it would drive them crazy. There was more interest in magical powers and
psychic phenomena than in Buddhism. A lot of killing of animals was done
in the pursuit of merit. Ajahn Chah was often very outspoken on such
issues, so he had many enemies.

Nevertheless, there were always many who loved him, and it was clear
that he never played on that. In fact, if any of his disciples were
getting too close, he would send them away. Sometimes monks became
attached to him, and he promptly sent them off to some other monastery.
Charismatic as he was, he always stressed the importance of the Sangha
-- of community spirit.

I think it was because Ajahn Chah was `nobody in particular' that he
could be anybody he chose. If he felt it was necessary to be fierce, he
could be that. If he felt that somebody would benefit from warmth and
kindness, then he would give them. You had the feeling he would be
whatever was helpful for the person he was with. And he was very clear
about the proper understanding of conventions. Someone once asked about
the relative merits of \emph{arahants} and \emph{bodhisattvas}. He
answered, `Don't be an \emph{arahant}, don't be a \emph{bodhisattva},
don't be anything at all. If you are an \emph{arahant} you will suffer,
if you are \emph{bodhisattva} you will suffer, if you are anything at
you will suffer.' I had the feeling that Ajahn Chah wasn't anything at
all. The quality in him which inspired awe was the light of Dhamma he
reflected; it wasn't exactly him as a person.

So since first meeting Ajahn Chah, I have had an unshakable conviction
that this way is truly possible -- it works -- it is good enough. And
I've found a willingness to acknowledge that if there are any problems,
it's me who is creating them. It's not the form and it's not the
teachings. This appreciation made things a lot easier. It's important
that we are able to learn from all the ups and downs we have in
practice. It's important that we come to know how to be `a refuge unto
ourselves'-- to see clearly for ourselves. When I consider the morass of
selfishness and foolishness my life could have been, and then reflect on
the teachings and benefits I've received, I find I really want to
dedicate my life to being a credit to my teacher. This reflection has
been a great source of strength. This is one form of
\emph{sanghānussati}, `recollection of the Sangha' -- recollection of
the great debt we owe our teachers.

So I trust that you may find this is of some help in your practice.

\setChapterNote{July 1990, further recollections by those who knew him.}
\chapter{Living with Luang Por}

\emph{Paul Breiter (formerly Ven. Varapañño) writes of his early contact with
Ajahn Chah (c. 1970)}

One cold afternoon as we swept the monastery grounds with long-handled
brooms, I thought how nice it would be, what a simple thing it really
was, if we could have a sweet drink of sugary coffee or tea after
working like that, to warm the bones and give us a little energy for
meditation at night.

I had heard that Western monks in the forest tend to get infatuated with
sweets, and finally the dam burst for me. One morning on
\emph{pindapat}, from the moment I walked out of the gate of the Wat to
the moment I came back about one and a half hours later, I thought
continually about sugar, candy, sweets, chocolate. Finally I sent a
letter asking a lay supporter in Bangkok to send me some palm-sugar
cakes. And I waited. The weeks went by. One day I went to town with a
layman to get medicine. We stopped by the Post Office and my
long-awaited package was there. It was huge, and ants were already at
it.

When I got back to the Wat, I took the box to my \emph{kuti} and opened
it. There were 20--25 pounds of palm and sugarcane cakes. I went wild,
stuffing them down until my stomach ached. Then I thought I should share
them (otherwise I might get very sick!), so I put some aside and took
the rest to Ajahn Chah's \emph{kuti}. He had the bell rung, all the
monks and novices came, and everyone enjoyed a rare treat.

That night I ate more; and the next morning I couldn't control myself.
The sugar cakes were devouring me; my blessing started to seem like a
curse. So I took the cakes in a plastic bag and decided to go round the
monks' \emph{kutis} and gave them away.

For a start I fell down my stairs and bruised myself nicely. The wooden
stairs can get slippery in cold weather, and I wasn't being very mindful
in my guilty, distressed state of mind.

The first \emph{kuti} I went to had a light on inside, but I called and
there was no answer. Finally, after I'd called several times and waited,
the monk timidly asked who it was (I didn't yet understand how strong
fear of ghosts is among those people). I offered him some sugar, and he
asked me why I didn't want to keep it for myself. I tried to explain
about my defiled state of mind. He took one (it was hard to get them to
take much, as it is considered to be in very bad taste to display one's
desire or anger).

I repeated this with a few others, having little chats along the way. It
was getting late, and although I hadn't unloaded all the sugar cakes, I
headed back to my \emph{kuti}. My flashlight batteries were almost dead,
so I lit matches to try to have a view of the path -- there were lots of
poisonous things creeping and crawling around in the forest. I ran into
some army ants and experienced my first fiery sting. I got back to my
\emph{kuti} feeling very foolish. In the morning I took the rest of the
cakes and gave them to one of the senior monks, who I felt would have
the wisdom and self-discipline to be able to handle them.

But my heart grew heavy. I went to see Ajahn Chah in the afternoon to
confess my sins. I felt like it was all over for me, there was no hope
left. He was talking with one old monk. I made the customary three
prostrations, sat down and waited. When he acknowledged me, I blurted
out, `I'm impure, my mind is soiled, I'm no good\ldots{}'He looked very
concerned. `What is it?' he asked. I told him my story. Naturally he was
amused, and within a few minutes I realized that he had me laughing. I
was very light-hearted; the world was no longer about to end. In fact, I
had forgotten about my burden. This was one of his most magical gifts.
You could feel so burdened and depressed and hopeless, and after being
around him for a few minutes it all vanished, and you found yourself
laughing. Sometimes you only needed to go and sit down at his
\emph{kuti} and be around him as he spoke with others. Even when he was
away I would get a `contact high' of peacefulness as soon I got near his
\emph{kuti} to clean up or to sweep leaves.

He said, `In the afternoon, when water-hauling is finished, you come
here and clean up.' My first reaction was, `He's got a lot of nerve,
telling me to come and wait on him.' But apart from being one of my
duties, it was a foot in the door and a privilege. Through it, I was to
start seeing that there was a way of life in the monastery which is
rich, structured and harmonious. And at the centre of it all is the
teacher, who is someone to be relied on.

Finally, he asked why I was so skinny. Immediately, one of the monks who
was there told him that I took a very small ball of rice at meal-time.
Did I not like the food? I told him I just couldn't digest much of the
sticky rice, so I kept cutting down. I had come to accept it as the way
it was, thinking I was so greedy that eating less and less was a virtue.
But he was concerned. Did I feel tired? Most of the time I had little
strength, I admitted. `So', he said, `I'm going to put you on a special
diet for a while -- just plain rice gruel and fish sauce to start with.
You eat a lot of it, and your stomach will stretch out. Then we'll go to
boiled rice, and finally to sticky rice. I'm a doctor', he added. (I
found out later on that he actually was an accomplished herbalist, as
well as having knowledge of all the illnesses to which monks are prone).
He told me not to push myself too much. If I didn't have any strength, I
didn't have to carry water, etc.

That was when the magic really began. That was when he was no longer
just Ajahn Chah to me. He became Luang Por, `Venerable Father'.

\textbf{\emph{Ajahn Munindo}} describes a visit from Luang Por.

There was a very difficult period in my training in Thailand, after I
had already been a monk for about four years. As a result of a motorbike
accident I had had before I was ordained, and a number of years of
sitting in bad posture, my knees seized up. The doctors in Bangkok said
it was severe arthritis, but nothing that a small operation couldn't
fix. They said it would take two or three weeks. But after two months
and three operations I was still hardly walking. There had been all
kinds of complications: scar tissue, three lots of general anaesthetic
and the hot season was getting at me; my mind was really in a state. I
was thinking, `My whole life as a monk is ruined. Whoever heard of a
Buddhist monk who can't sit cross-legged?' Every time I saw somebody
sitting cross-legged I'd feel angry. I was feeling terrible, and my mind
was saying, `It shouldn't be like this; the doctor shouldn't have done
it like that; the monks' rules shouldn't be this way \ldots{}.' It was
really painful, physically and mentally. I was in a very unsatisfactory
situation.

Then I heard that Ajahn Chah was coming down to Bangkok. I thought if I
went to see him he might be able to help in some way. His presence was
always very uplifting. When I visited him I couldn't bow properly; he
looked at me and asked, `What are you up to?' I began to complain. `Oh,
Luang Por', I said, `It's not supposed to be this way. The doctors said
two weeks and it has been two months \ldots{}' I was really wallowing.
With a surprised expression on his face he said to me, very powerfully,
`What do you mean, it shouldn't be this way? If it shouldn\emph{'}t be
this way, it wouldn\emph{'}t be this way!'

That really did something to me. He pointed to exactly what I was doing
that was creating the problem. There was no question about the fact of
the pain; the problem was my denying that fact, and that was something I
was doing. This is not just a theory. When someone offers us the
reflection of exactly what we are doing, we are incredibly grateful,
even if at that time we feel a bit of a twit.

\emph{*}Ajahn Sumedho* **recalls an incident from his early days with
Ajahn Chah (c. 1967--69).

In those days I was a very junior monk, and one night Ajahn Chah took us
to a village fete -- I think Satimanto was there at the time.

Now, we were all very serious practitioners and didn't want any kind of
frivolity or foolishness; so of course going to a village fete was the
last thing we wanted to do, because in these villages they love
loudspeakers.

Anyway, Ajahn Chah took Satimanto and I to this village fete, and we had
to sit up all night with all the raucous sounds of the loudspeakers
going and monks giving talks all night long. I kept thinking, `Oh, I
want to get back to my cave. Green skin monsters and ghosts are much
better than this.' I noticed that Satimanto (who was incredibly serious)
was looking angry and critical, and very unhappy. So we sat there
looking miserable, and I thought, `Why does Ajahn Chah bring us to these
things?' Then I began to see for myself. I remember sitting there
thinking, `Here I am getting all upset over this. Is it that bad? What's
really bad is what I'm making out of it, what's really miserable is my
mind. Loudspeakers and noise, distraction and sleepiness -- all that,
one can really put up with. It's that awful thing in my mind that hates
it, resents it and wants to leave.'

That evening I could really see what misery I could create in my mind
over things that one can bear. I remember that as a very clear insight
of what I thought was miserable and what really is miserable. At first I
was blaming the people and the loudspeakers, and the disruption, the
noise and the discomfort, I thought that was the problem. Then I
realized that it wasn't -- it was my mind that was miserable.

\textbf{\emph{Sister Candasiri}} first met Luang Por Chah while still a
laywoman, during his second visit to England in 1979.

For me one of the most striking things about Luang Por Chah was the
effect of his presence on those around him. Watching Ajahn Sumedho --
who hitherto had been for me a somewhat awe-inspiring teacher -- sit at
his feet with an attitude of sheer delight, devotion and adoration
lingers in the mind as a memory of extraordinary sweetness. Ajahn Chah
would tease him, `Maybe it's time for you to come back to Thailand!'
Everyone gasped inwardly -- `Is he serious?'

Later on a visitor, a professional flautist, began to ask about music,
`What about Bach? Surely there's nothing wrong with that -- much of his
music is very spiritual, not at all worldly.' (It was a question that
interested me greatly). Ajahn Chah looked at her, and when she had
finished he said quietly, `Yes, but the music of the peaceful heart is
much, much more beautiful.'

\textbf{\emph{Ajahn Santacitto}} recollects his own first meeting with
Ajahn Chah.

From the very first meeting with Ajahn Chah, I couldn't help but be
aware of how powerful a force was emanating from this person. I had just
arrived at the monastery with a friend, and neither of us spoke much
Thai, so the possibility of talking with and hearing Dhamma from Ajahn
Chah was very limited. I was considering taking ordination as a monk
mainly in order to learn about meditation, rather than from any serious
inclination towards religious practice.

It happened that just at that time, a group of local villagers came to
ask him to perform a certain traditional ceremony which involved a great
deal of ritual. The laymen bowed down before the Master, then they got
completely covered over with a white cloth, and then holy water was
brought out and candles were dripped into it, while the monks did the
chanting. And young lad that I was, very science-minded, rather
iconoclastic by nature, I found this all rather startling, and wondered
just what I was letting myself in for. Did I really want to become one
of these guys and do this kind of thing?

So I just started to look around, watching this scene unfold before me,
until my eye caught Ajahn Chah's, and what I saw on his face was very
unexpected: there was the smile of a mischievous young man, as if he
were saying, `Good fun, isn't it!' This threw me a bit; I could no
longer think of him as being attached to this kind of ritual, and I
began to appreciate his wisdom. But a few minutes later, when the
ceremony was over and everyone got up and out from under the cloth, all
looking very happy and elated, I noticed that the expression on his face
had changed; no sign of that mischievous young lad. And although I
couldn't understand a word of Thai, I couldn't help but feel very deeply
that quality of compassion in the way he took this opportunity of
teaching people who otherwise might not have been open and susceptible.
It was seeing how, rather than fighting and resisting social customs
with their rites and rituals, he knew how to use them skilfully to help
people. I think this is what hooked me.

It happened countless times: people would come to the monastery with
their problems, looking for an easy answer, but somehow, whatever the
circumstances, his approach never varied. He met everybody with a
complete openness, with the `eyes of a babe', as it seemed to me, no
matter who they were. One day a very large Chinese businessman came to
visit. He did his rather disrespectful form of bowing, and as he did so
his sports shirt slipped over his back pocket, and out stuck a pistol.
Carrying a pistol is about the grossest thing you can do when coming to
see an Ajahn in a Thai monastery! That really took me aback, but what
struck me most of all was that when Ajahn Chah looked at him, there was
that same openness, no difference, `eyes like a babe'. There was a
complete openness and willingness to go into the other person's world,
to be there, to experience it, to share it with them.

\emph{*}Ajahn Sumedho* **recalls an incident during Luang Por's visit to
Britain in 1977.

When Ajahn Chah first visited England, he was invited to a certain
woman's home for a vegetarian meal. She obviously had put a lot of
effort into creating the most delicious kinds of food. She was bustling
about offering this food and looking very enthusiastic. Ajahn Chah was
sitting there assessing the situation, and then suddenly he said: `This
is the most delicious and wonderful meal I have ever had!'

That comment was really something, because in Thailand, monks are not
supposed to comment on the food. And yet Luang Por suddenly manifested
this charming character in complimenting a woman who needed to be
complimented, and it made her feel so happy. He had a feeling for the
time and place, for the person he was with, for what would be kind. He
could step out of the designated role and manifest in ways that were
appropriate; he was not actually breaking any rules, but it was out of
character. Now, that shows wisdom and the ability to respond to a
situation -- not to be just rigidly bound within a convention that
blinds you.

\emph{*}Paul Breiter* **

On his visit in 1979, he related that once a Westerner (a layman, I
think) came to Wat Pah Pong and asked him if he was an \emph{arahant}.
Ajahn Chah told him, `Your question is a question to be answered. I will
answer it like this: I am like a tree in the forest. Birds come to the
tree, they will sit on its branches and eat its fruit. To the birds, the
fruit may be sweet or sour or whatever. But the tree doesn't know
anything about it. The birds say `sweet' or they say `sour' -- from the
tree's point of view this is just the chattering of the birds.'

On that same evening we also discussed the relative virtues of the
\emph{arahant} and the \emph{bodhisattva}. He ended our discussion by
saying, `Don't be an arahant. Don't be a Buddha. Don't be anything at
all. Being something makes problems. So don't be anything. You don't
have to be something, he doesn't have to be something, I don't have to
be something \ldots{}' He paused, and then said, `Sometimes when I think
about it, I don't want to say anything.'

\setChapterNote{April 1992}
\chapter{Ajahn Chah Passes Away}

\emph{*}Venerable Thitapañño ****offers an account of the events at Wat
Pah Pong immediately following Luang Por Chah's death. *

On the morning of 16th January, the Sangha in Britain received a brief
message from Wat Pah Nanachat to inform us of the death of Luang Por
Chah. The Venerable Ajahn had been critically ill, paralyzed and
rendered completely incapacitated by brain damage and numerous strokes
over the past ten years. Our winter retreat offered us an ideal
opportunity to pay honour to his example, reflect upon his teachings and
further our practice in the way that he made clear.

It was during a retreat at Wat Keuan* *that Ajahn Sumedho and the
Western Sangha who had gathered there heard that Luang Por Chah had been
admitted to Ubon Hospital. Malfunctioning kidneys and heart
complications had proved to be beyond the medical skills of the monks
nursing him. During the ten years of his illness Luang Por had entered
hospital many times, yet on each occasion he had miraculously recovered.
However, reports soon began to reach us that his body was refusing to
take food and the general state of his health was deteriorating.

Early on the evening of 15th January the doctors at the ICU realized
that Luang Por's condition had deteriorated to the extent that he was
beyond medical assistance. At 10 pm Luang Por was taken by ambulance to
his nursing \emph{kuti} at Wat Pah Pong, in compliance with his previous
request that he might pass away in his own monastery. It was at 5.20 am
on 16th January that the body of Luang Par Chah breathed its last, and
in an atmosphere of peace the life of a great Buddhist master came to
its end.

The attendant monks chanted the reflection that death is the natural
consequence of birth and that in the cessation of conditions is peace,
then prepared Luang Por's body for the funeral services. As the news of
his death spread, people began to arrive to pay their respects. Soon
government officials, as representatives of the King, came to perform
the initial ceremonies necessary for a royal funeral.

Within hours the corpse was moved to the main \emph{sālā}, where it was
laid in an ornately decorated coffin. The coffin was then sealed, and a
picture of Luang Por was placed to the left along with different
requisites such as his bowl and robes. Wreaths from the King, the Queen
and other members of the royal family were placed to the right. In front
of the coffin, extensive flower arrangements created the finishing
touches.

As the news of Luang Por's death spread, his disciples rushed to the Wat
to pay their respects and offer their support with the preparations to
receive visitors to the monastery. It was decided that during the 15
days following Luang Por's death a Dhamma practice session would be
held, as an offering of remembrance and a focal point around which the
many incoming lay and monastic disciples could collect themselves. The
Sangha from Wat Pah Nanachat would come over every day at around 5 pm
and stay until midnight. During this period of 15 days, about 400 monks,
70 nuns and 500 lay people resided at Wat Pah Pong, practising
meditation until midnight, listening to talks on Dhamma themes and
participating in various funeral ceremonies. Most of the Sangha were
living out under the trees of the forest, using their \emph{glot}
(mosquito net umbrellas) as protection from the elements and insects.
The monastery became a \emph{glot} village.

Soon a huge open-air restaurant complex sprung up at the entrance to the
monastery, serving free food and drink to the enormous numbers of people
who began to make their way there from all over Thailand. As the days
passed, I began to feel a sense of awe as people streamed into the
monastery from early morning to late at night: people of all ages --
families, school groups and individuals. In those first few days over
50,000 books were distributed, which gives some indication of the
numbers coming. By the fourteenth and fifteenth days, the number of
people coming was steadily increasing to over 10,000 per day. As the
people entered the monastery, they filed quietly down the road leading
to the \emph{sālā}, waiting for an opportunity to enter and bow in
respect, and then to sit for a short while before making way for the
next group. Meanwhile the monks, nuns and resident laypeople would be
sitting in meditation, chanting or listening to a talk. Luang Por Jun
led the funeral chanting and various senior monks gave talks. Ajahn Maha
Boowa, the renowned forest meditation master, came over from his own
monastery near Udorn to give a Dhamma talk, and commented on the quiet,
harmonious atmosphere of the Wat, in contrast to the confusion and noise
he had experienced at similar funerals.

A visit from the King's sister at this time seemed to presage the
arrival of the King for the fiftieth day ceremonies on 6th March. As
always in Buddhism, however, especially in Thailand, nothing is certain.
The hundredth day after the death of Luang Por will also be a day of
considerable importance.

Because of the arrangements for the hundreds of thousands of people
expected to attend the actual burning of Luang Por's body (at similar
funerals for famous teachers, up to a million people have attended), and
also to find a day suitable for the King, it was decided to hold the
funeral early in 1993.

For each of us Luang Por Chah has a personal meaning, depending on our
contact with him. I will always wonder and be inspired at the sight of
tens of thousands of people coming to Wat Pah Pong, to pay respects to a
person who had not spoken for ten years, and with whom most had never
had the opportunity to speak. They came to bow before the body of a
being whom they recognized as personifying our highest aspiration -- a
life free from the blindness of self-centred action. Freed of this
delusion, the goal of the Buddhist path is fulfilled. For me, the whole
occasion demonstrated the breadth and power of the influence of such a
being.

\setChapterNote{April 1992}
\chapter{The Fiftieth Day Commemoration}

\emph{*}Ven. Nyanaviro ****offers a report of the
commemorative services at Wat Pah Pong fifty days after the death
of Venerable Ajahn Chah. *

Wat Pah Nanachat, 6th March

At 1 pm 300 laypeople and 200 monks and nuns gathered in the new
\emph{sālā}. The floor is polished granite and the walls are partially
marbled. Four huge chandeliers hang from the high ceiling. Large
garlands of flowers hang from the walls and the shrine is covered in
artificial lotuses, which look beautiful. The cry of the wild chickens
breaks into the silence -- they are all over the Wat!

Ajahn Jun gave a *desanā *at 2 pm, mentioning the debt of gratitude we
all owe to Luang Por Chah. He exhorted us to make an effort to keep up
the practices that Luang Por Chah taught. He also talked about the
benefits of keeping good standards regarding *sīla *and the monastic
conventions, and reminded us that the practice was not in the forest or
the Wat, but is the work of the mind in the body. `So all of Buddhism is
right here in this body/mind. Don't let the practice become perfunctory
-- put life into it. Even though Ajahn Chah is dead, the goodness and
virtue that he embodied are still alive.'

At 7 pm about 3,000 lay people and 300 monks and novices gathered in the
new sala for the evening chanting. At 9 pm Luang Por Paññananda gave a
\emph{desanā}. He started by praising Ajahn Chah as one of the great
monks of this era, who taught a pure kind of Buddhism, with nothing
extraneous. Ajahn Chah had trained a Sangha which could continue, most
notably overseas, where monasteries had arisen from his inspiration.
They represented an historic occasion in the development of Buddhism.

Luang Por Paññananda commented that Ajahn Chah had taught people to be
wise. The way the Pah Pong Sangha was handling the proceedings was a
good example: in Thailand some degenerate practices had crept into
funeral services, making them an excuse for a party with gambling and
alcohol. But the purpose of a funeral is for the study of Dhamma, not
for distraction! It's a lesson, a reminder. Even though Ajahn Chah is
dead, the goodness and virtue that he embodied are still alive. We must
maintain that which he gave to us all: we have to be `mediums' for Luang
Por Chah, channelling his goodness and virtue through our hearts. If we
reflect on Luang Por Chah's \emph{mettā}, \emph{sīla} and \emph{paññā}
and internalize them, then it's as if he is in our hearts, far better
than hanging a medallion with his picture on it around our necks.

Luang Por Paññananda concluded by reflecting that the Buddha left the
Dhamma--Vinaya, not an individual, as our teacher, and that his teaching
was one of sustaining compassion, wisdom and purity. So our practice is
to wish all beings well, refraining from harming others or the
environment. Then to have wisdom -- whatever we're doing, inquiring as
to why are we doing it, what our purpose is and what is the most skilful
means. And to dwell in purity -- honouring goodness by making body,
speech and mind good, associating with good people, and frequenting
places of goodness.

Luang Por Paññananda had witnessed a decline in most monasteries after
the teacher died, with schisms occurring between the disciples. So we
should be careful not to get attached to views, or to wealth and gains,
and agree to have regular meetings in order to maintain harmony. Sangha
and laity should support all the things that are in line with the way
Luang Por Chah taught, and refrain from the things he cautioned us
about. We must all help to do this.

The evening continued with different senior monks giving talks. Ajahn
Santacitto was next, and his memory of Thai was excellent. They were
still talking when we left at 4.45 am, but probably finished at dawn
with morning \emph{pūjā}.

\setChapterNote{17th June, 1918 -- 16th January 16, 1992}
\chapter{April 1992, A Noble Life}

Venerable Ajahn Chah was born on 17th June 1918 in a small village near
the town of Ubon Rajathani, north-east Thailand. Between the ages of 9
and 17 he was a \emph{sāmanera} (novice monk), during which time he
received his basic schooling, before returning to lay life to help his
parents on the farm. At the age of 20, however, he decided to resume
monastic life, and on April 26, 1939 he received \emph{upasampada}
(\emph{bhikkhu} ordination).

Ajahn Chah's early monastic life followed a traditional pattern of
studying Buddhist teachings and the Pali scriptural language. In his
fifth year his father fell seriously ill and died, a blunt reminder of
the frailty and precariousness of human life. This caused him to think
deeply about life's real purpose, for although he had studied
extensively and gained some proficiency in Pali, he seemed no nearer to
a personal understanding of the end of suffering. Feelings of
disenchantment set in, and finally (in 1946) he abandoned his studies
and set off on mendicant pilgrimage.

He walked some 400 km to central Thailand, sleeping in forests and
gathering almsfood in the villages on the way. He took up residence in a
monastery where the Vinaya was carefully studied and practised. While
there he was told about Venerable Ajahn Mun Bhuridatta, a most highly
respected meditation master. Keen to meet such an accomplished teacher,
Ajahn Chah set off on foot for the north-east in search of him.

At this time Ajahn Chah was wrestling with a crucial problem. He had
studied the teachings on morality, meditation and wisdom, which the
texts presented in minute and refined detail, but he could not see how
they could all actually be put into practice. Ajahn Mun told him that
although the teachings are indeed extensive, at their heart they are
very simple. With mindfulness established, it is seen that everything
arises in the mind;. right there is the true path of practice. This
succinct and direct teaching was a revelation for Ajahn Chah, and
transformed his approach to practice. The way was clear.

For the next seven years Ajahn Chah practised in the style of the
austere Forest Tradition, wandering through the countryside in quest of
quiet and secluded places for developing meditation. He lived in tiger-
and cobra-infested jungles, and even in charnel-grounds, using
reflections on death to overcome fear and penetrate to the true meaning
of life. In 1954, after years of wandering, he was invited back to his
home village. He settled close by, in a fever-ridden haunted forest
called `Pah Pong'. Despite the hardships of malaria, poor shelter and
sparse food, disciples gathered around him in increasing numbers. The
monastery which is now known as Wat Pah Pong began there, and eventually
branch monasteries were also established elsewhere.

The training in Ajahn Chah's monasteries was quite strict and
forbidding. Ajahn Chah often pushed his monks to their limits, to test
their powers of endurance so that they would develop patience and
resolution. He sometimes initiated long and seemingly pointless work
projects in order to frustrate their attachment to tranquillity. The
emphasis was always on surrender to the way things are, and great stress
was placed upon strict observance of the Vinaya.

In 1977 Ajahn Chah was invited to visit Britain by the English Sangha
Trust, a charity with the aim of establishing a locally-resident
Buddhist Sangha. He took Venerable Sumedho and Venerable Khemadhammo
along, and seeing the serious interest there, left them in London at the
Hampstead Vihara. Another two of Ajahn Chah's Western \emph{bhikkhus},
who were then visiting their families in North America, were invited to
stay in London to make up a small resident Sangha. He returned to
Britain in 1979, at which time the monks were leaving London to begin
Chithurst Buddhist Monastery in Sussex. He then went on to America and
Canada to visit and teach.

After this trip and again in 1981, Ajahn Chah spent the Rains away from
Wat Pah Pong, since his health was failing due to the debilitating
effects of diabetes. As his illness worsened, he would use his body as a
teaching, a living example of the impermanence of all things. He
constantly reminded people to endeavour to find a true refuge within
themselves, since he would not be able to teach for very much longer.

Before the end of the 1981 Rains, he was taken to Bangkok for an
operation; however, it did little to improve his condition. Within a few
months he stopped talking, and gradually he lost control of his limbs
until he was completely paralyzed and bedridden. From then on he was
diligently nursed and attended by his \emph{bhikkhu} disciples, grateful
for the occasion to offer service to the teacher who so patiently and
compassionately showed the Way to so many.

\setChapterNote{Extracts from a conversation between Luang Por Chah and a lay Buddhist.}
\chapter{October 1992, Questions \& Answers}

Q: There are those periods when our hearts happen to be absorbed in
things and become blemished or darkened, but we are still aware of
ourselves, such as when some form of greed, hatred, or delusion comes
up. Although we know that these things are objectionable, we are unable
to prevent them from arising. Could it be said that even as we are aware
of them, we are providing the basis for increased clinging and
attachment, and maybe putting ourselves further back than where we
started from?

Luang Por Chah: That's it! You must keep knowing them at that point;
that's the method of practice.

I mean that simultaneously, we are both aware of them and repelled by
them, but lacking the ability to resist them; they just burst forth.

By then it's already beyond your capability to do anything. At that
point you have to readjust yourself and then continue contemplation.
Don't just give up on them there and then. When you see things arise in
that way you tend to get upset or feel regret, but it is possible to say
that they are uncertain and subject to change. What happens is that you
see these things are wrong, but you are still not ready or able to deal
with them. It's as if they are independent entities, the leftover kammic
tendencies that are still creating and conditioning the state of the
heart. You don't wish to allow the heart to become like that, but it
does, and it indicates that your knowledge and awareness are still
neither sufficient nor fast enough to keep abreast of things.

You must practise and develop mindfulness as much as you can in order to
gain a greater and more penetrating awareness. Whether the heart is
soiled or blemished in some way, it doesn't matter; whatever comes up,
you should contemplate the impermanence and uncertainty of it. By
maintaining this contemplation at each instant that something arises,
after some time you will see the impermanence of all sense objects and
mental states. Because you see them as such, gradually they will lose
their importance, and your clinging and attachment to that which is a
blemish on the heart will continue to diminish. Whenever suffering
arises, you will be able to work through it and readjust yourself, but
you shouldn't give up on this work or set it aside. You must keep up a
continuity of effort and try to make your awareness fast enough to keep
in touch with the changing mental conditions. It could be said that so
far your development of the Path still lacks sufficient energy to
overcome the mental defilements; whenever suffering arises, the heart
becomes clouded over. But one must keep developing that knowledge and
understanding of the clouded heart; this is what you reflect on.

You must really take hold of it and repeatedly contemplate that this
suffering and discontentment are just not sure things. They are
something that is ultimately impermanent, unsatisfactory, and not-self.
Focusing on these three characteristics, whenever these conditions of
suffering arise again, you will know them straightaway, having
experienced them before.

Gradually, little by little, your practice should gain momentum, and as
time passes, whatever sense objects and mental states arise will lose
their value in this way. Your heart will know them for what they are and
accordingly put them down. When you reach the point where you are able
to know things and put them down with ease, they say that the Path has
matured internally and you will have the ability to bear down swiftly
upon the defilements. From then on there will just be the arising and
passing away in this place, the same as waves striking the seashore.
When a wave comes in and finally reaches the shoreline, it just
disintegrates and vanishes; a new wave comes and it happens again, the
wave going no further than the limit of the shoreline. In the same way,
nothing will be able to go beyond the limits established by your own
awareness.

That's the place where you will meet and come to understand
impermanence, unsatisfactoriness and not-self. It is there that things
will vanish -- the three characteristics of impermanence,
unsatisfactoriness and not-self are the same as the seashore, and all
sense objects and mental states that are experienced go in the same way
as the waves. Happiness is uncertain; it's arisen many times before.
Suffering is uncertain; it's arisen many times before. That's the way
they are. In your heart you will know that they are like that, they are
`just that much'. The heart will experience these conditions in this
way, and they will gradually keep losing their value and importance.
This is talking about the characteristics of the heart, the way it is.
It is the same for everybody, even the Buddha and all his disciples were
like this.

If your practice of the Path matures, it will become automatic and it
will no longer be dependent on anything external. When a defilement
arises, you will immediately be aware of it and accordingly be able to
counteract it. However, that stage when they say that the Path is still
neither mature enough nor fast enough to overcome the defilements is
something that everybody has to experience -- it's unavoidable. But it
is at this point that you must use skilful reflection. Don't go
investigating elsewhere or trying to solve the problem at some other
place. Cure it right there. Apply the cure at that place where things
arise and pass away. Happiness arises and then passes away, doesn't it?
Suffering arises and then passes away, doesn't it? You will continuously
be able to see the process of arising and ceasing, and see that which is
good and bad in the heart. These are phenomena that exist and are part
of nature. Don't cling tightly to them or create anything out of them at
all.

If you have this kind of awareness, then even though you will be coming
into contact with things, there will not be any noise. In other words,
you will see the arising and passing away of phenomena in a very natural
and ordinary way. You will just see things arise and then cease. You
will understand the process of arising and ceasing in the light of
impermanence, unsatisfactoriness, and not-self.

The nature of the Dhamma is like this. When you can see things as `just
that much', then they will remain as `just that much.' There will be
none of that clinging or holding on -- as soon as you become aware of
attachment, it will disappear. There will be just the arising and
ceasing, and that is peaceful. That it's peaceful is not because you
don't hear anything; there is hearing, but you understand the nature of
it and don't cling or hold on to anything. This is what is meant by
peaceful -- the heart is still experiencing sense objects, but it
doesn't follow or get caught up in them. A division is made between the
heart, sense objects and the defilement; but if you understand the
process of arising and ceasing, then there is nothing that can really
arise from it -- it will end just there.

\setChapterNote{Luang Por Chah}
\chapter{January 1993, Questions \& Answers II}

The second in a series of extracts from a conversation between Luang Por
Chah and a lay Buddhist.

Q: Does one have to practise and gain samādhi (concentration) before one
can contemplate the Dhamma?

\emph{Luang Por Chah:} We can say that's correct from one point of view,
but from the aspect of practice, \emph{paññā} has to come first. In
conventional terms, it's \emph{sīla}, \emph{samādhi} and then
\emph{paññā}, but if we are truly practising the Dhamma, then
\emph{paññā} comes first. If \emph{paññā} is there from the beginning,
it means that we know what is right and what is wrong; and we know the
heart that is calm and the heart that is disturbed and agitated.

Talking from the scriptural basis, one has to say that the practice of
restraint and composure will give rise to a sense of shame and fear of
any form of wrong-doing that potentially may arise. Once one has
established the fear of that which is wrong and one is no longer acting
or behaving wrongly, then that which is wrong will not be present within
one. When there is no longer anything wrong present within, this
provides the conditions from which calm will arise in its place. That
calm forms a foundation from which *samādhi *will grow and develop over
time.

When the heart is calm, that knowledge and understanding which arises
from within that calm is called \emph{vipassanā}. This means that from
moment to moment there is a knowing in accordance with the truth, and
within this are contained different properties. If one was to set them
down on paper they would be \emph{sīla}, \emph{samādhi} and
\emph{paññā}. Talking about them, one can bring them together and say
that these three dhammas form one mass and are inseparable. But if one
were to talk about them as different properties, then it would be
correct to say \emph{sīla}, \emph{samādhi} and \emph{paññā}.

However, if one was acting in a unwholesome way, it would be impossible
for the heart to become calm. So it would be most accurate to see them
as developing together, and it would be right to say that this is the
way that the heart will become calm. Talking about the practice of
\emph{samādhi:} it involves preserving \emph{sīla}, which includes
looking after the sphere of one's bodily actions and speech, in order
not to do anything which is unwholesome or would lead one to remorse or
suffering. This provides the foundation for the practice of calm, and
once one has a foundation in calm, this in turn provides a foundation
which supports the arising of \emph{paññā}.

In formal teaching they emphasize the importance of \emph{sīla.
Ādikalyānam, majjhekalyānam, pariyosānakalyānam} -- the practice should
be beautiful in the beginning, beautiful in the middle and beautiful in
the end. This is how it is. Have you ever practised \emph{samādhi}?

Q: I am still learning. The day after I went to see Tan Ajahn at Wat
Keu--an my aunt brought a book containing some of your teaching for me
to read. That morning at work I started to read some passages which
contained questions and answers to different problems. In it you said
that the most important point was for the heart to watch over and
observe the process of cause and effect that takes place within; just to
watch and maintain the knowing of the different things that come up.

That afternoon I was practising meditation, and during the sitting the
characteristic that appeared was that I felt as though my body had
disappeared. I was unable to feel the hands or legs and there were no
bodily sensations. I knew that the body was still there, but I couldn't
feel it. In the evening I had the opportunity to go and pay respects to
Tan Ajahn Tate, and I described to him the details of my experience. He
said that these were the characteristics of the heart that appear when
it unifies in \emph{samādhi}, and that I should continue practising. I
had this experience only once; on subsequent occasions I found that
sometimes I was unable to feel only certain areas of the body, such as
the hands, whereas in other areas there was still feeling. Sometimes
during my practice I start to wonder whether just sitting and allowing
the heart to let go of everything is the correct way to practise; or
else should I think over and occupy myself with the different problems
or unanswered questions concerning the Dhamma which I still have?

Luang Por Chah: It's not necessary to keep going over or adding anything
on at this stage. This is what Tan Ajahn Tate was referring to; one must
not repeat or add onto that which is there already. When that particular
kind of knowing is present, it means that the heart is calm and it is
that state of calm which one must observe. Whatever one feels, whether
it feels like there is a body or a self or not, this is not the
important point. It should all come within the field of one's awareness.
These conditions indicate that the heart is calm and has unified in
\emph{samādhi}.

When the heart has unified for a long period for a few times, then there
will be a change in the conditions and they say that one `withdraws'.
That state is called \emph{appanā samādhi} (absorption), and having
entered it, the heart will subsequently withdraw. In fact, although it
would not be incorrect to say that the heart withdraws, it doesn't
actually withdraw. Another way is to say that it flips back, or that it
changes, but the style used by most teachers is to say that once the
heart has reached the state of calm, then it will withdraw. However,
people get caught up in disagreements over the use of language. It can
cause difficulties and one might start to wonder, `How on earth can it
withdraw? This business of withdrawing is just confusing!' It can lead
to much foolishness and misunderstanding just because of the language.

What one must understand is that the way to practise is to observe these
conditions with \emph{sati-sampajañña} (mindfulness and clear
comprehension). In accordance with the characteristic of impermanence,
the heart will turn about and withdraw to the level of \emph{upacāra
samādhi} (access concentration). If it withdraws to this level, one can
gain understanding through awareness of sense impressions and mental
states, because at the deeper level (where the mind is fixed with just
one object) there is no understanding. If there is awareness at this
point, that which appears will be \emph{sankhāra} (mental formations).
It will be similar to two people having a conversation and discussing
the Dhamma together. One who misunderstands this might feel disappointed
that his heart is not really calm, but in fact this dialogue takes place
within the confines of the calm and restraint which have developed.
These are the characteristics of the heart once it has withdrawn to the
level of \emph{upacāra} -- there will be the ability to know about and
understand different things.

The heart will stay in this state for a period and then it will turn
inwards again. In other words, it will turn and go back into the deeper
state of calm where it was before; or it is even possible that it might
obtain purer and calmer levels of concentrated energy than were
experienced before. If it does not reach such a level of concentration,
one should merely note the fact and keep observing until the time when
the heart withdraws again. Once it has withdrawn, different problems
will arise within the heart.

This is the point where one can have awareness and understanding of
different things. Here is where one should investigate and examine the
different preoccupations and issues which affect the heart, in order to
understand and penetrate them. Once these problems are finished with,
the heart will gradually move inwards towards the deeper level of
concentration again. The heart will stay there and mature, freed from
any other work or external impingement. There will just be the
one-pointed knowing, and this will prepare and strengthen one's
mindfulness until the time to re-emerge is reached.

These conditions of entering and leaving will appear in one's heart
during the practice, but this is something that is difficult to talk
about. It is not harmful or damaging to one's practice. After a period
the heart will withdraw and the inner dialogue will start in that place,
taking the form of \emph{sankhāra} (mental formations) conditioning the
heart. If one doesn't know that this activity is \emph{sankhāra}, one
might think that it is \emph{paññā}, or that \emph{paññā} is arising.
One must see that this activity is fashioning and conditioning the
heart, and the most important thing about it is that it is impermanent.
One must continually keep control and not allow the heart to start
following and believing in all the different creations and stories that
it cooks up. All that is just \emph{sankhāra}, it doesn't become
\emph{paññā}.

The way \emph{paññā} develops is when one listens and knows the heart,
as the process of creating and conditioning takes it in different
directions, and one reflects on the instability and uncertainty of this.
The realization of its impermanence will provide the cause by which one
can let go of things at that point. Once the heart has let go of things
and put them down at that point, it will gradually become more and more
calm and steady. One must keep entering and leaving \emph{samādhi} like
this, and \emph{paññā} will arise at that point. There one will gain
knowledge and understanding.

As one continues to practise, many different kinds of problems and
difficulties will tend to arise in the heart; but whatever problems the
world or even the universe might bring up, one will be able to deal with
them all. One's wisdom will follow them up and find answers for every
question and doubt. Wherever one meditates, whatever thoughts come up,
whatever happens, everything will be providing the cause for
\emph{paññā} to arise. This is a process that will take place by itself,
free from external influence. \emph{Paññā} will arise like this, but
when it does, one should be careful not to become deluded and see it as
\emph{sankhāra}. Whenever one reflects on things and sees them as
impermanent and uncertain, one shouldn't cling or attach to them in any
way. If one keeps developing this state, when \emph{paññā} is present in
the heart, it will take the place of one's normal way of thinking and
reacting, and the heart will become fuller and brighter in the centre of
everything. As this happens one knows and understands all things as they
really are -- one's heart will be able to progress with meditation in
the correct way and without being deluded. That is how it should be.

\chapter{January 1994, Jack Kornfield}

\emph{*} ****`I was enormously blessed to meet Ajahn Sumedho in 1967 at
an old ruined Cambodian temple on a mountain-top in Sakolnakorn,
Thailand. With his inspiration I went to see Ajahn Chah at Wat Pah Pong,
and eventually entered as a monk in 1969. I left to come back to the USA
in 1972, and was re-ordained to live as a monk with Ajahn Chah for a
time in 1982. Like all of us who were with him, I could tell many more
wonderful stories. Most simply, Ajahn Chah was the wisest (and one of
the most delightful) men I have ever known, and it has completely
changed my life to have him for a teacher.'*

Ajahn Chah had four basic levels of teaching, and each one, although at
times very difficult for the students, was taught with a lot of humour
and a lot of love. Ajahn Chah taught that until we can begin to respect
ourselves and our environment, practice doesn't really develop. And that
dignity, the ground of practice, comes through surrender, through
impeccable discipline. A lot of us in the West understand freedom to
mean freedom to do what we want, but I think you can see that to follow
the wants of the mind isn't terribly free. It's actually rather
troublesome.

A deeper freedom, taught through Dhamma, is the freedom within form: the
freedom we can find while relating to another human being, the freedom
of being born in a body with its limitations, and the freedom of a tight
monastic form. What Ajahn Chah did was create a situation of dignity and
demand. He really asked a lot from people, probably more than they'd
ever been asked in their whole life -- to give, to pay attention, to be
wholehearted. Sometimes practice is wonderful: the mind gets clear
enough that you smell and taste the air in ways that you haven't since
you were a child. But sometimes it's difficult. He said, `That's not the
point; the point is somehow to come to inner freedom.'

We used to sit for long hours at times, and the meditation hall for the
monks was a stone platform --they don't use cushions in Asia. You have a
square cloth like a handkerchief that you put down on the stone to sit
on. I remember that when I started, because sitting on the floor was so
painful, I would arrive early at the hall and get a place where I could
sit next to one of the pillars and lean against it. After about a week
of being with Ajahn Chah, he gathered the monks together for an evening
talk after the sitting, and he began to talk about how the true practice
of Dhamma was to become independent in any circumstance; to not need to
lean on things. And then he looked at me.

Sometimes you would sit while he'd talk to someone or receive visitors,
and you couldn't leave until you were dismissed. And you'd sit and sit,
and you'd look at your mind, and it would go, `Doesn't he know that we
are sitting here? Doesn't he know I'm thirsty or I want to get up?' And
he'd be talking away -- he knew very well. And you'd sit and sit and
just see all the movement of the mind. We would sit for hours. The
quality of endurance in the monks' life in the forest, where you just
sit and sit and sit, is a very important one.

He trusted that people came in order to learn and grow, and when it was
hard, that was all right by him. He didn't care if people had a hard
time. He would go up to them when they were having a hard time and he'd
say, `Are you angry? Whose fault is that, mine or yours?' So one really
had to give up a lot, but it wasn't to him or for him -- it was for
oneself. With surrender and dignity one learned to open up and see
clearly. It is essential in our practice to be unflinchingly honest
about ourselves and the world -- just as he was.

He would sit under his \emph{kuti,} and various lay visitors and other
disciples would come, and also some of his monks would be sitting
around, and he would make fun of people. He'd say, `I'd like to
introduce you to my monks. This one, he likes to sleep a lot. And this
one, he is always sick, his health is his thing; he just spends his time
worrying about his health. And this one is a big eater -- he eats more
than two or three other monks. And this is a doubter over there, he
really likes to doubt, really gets into it. And can you imagine, he had
three different wives at the same time. And this one likes to sit a lot,
all he does is go and sit in his \emph{kuti}; I think he is afraid of
people.' And then he'd point to himself and say, `Myself, I like to play
teacher.'

Once, when he came to the USA, there was a man who had been a monk with
him for a long time who had then disrobed and taken ordination as a Zen
priest. So he said, `I can't figure out this guy', (this man was acting
as his translator), `he is not quite a monk and he is not quite a lay
person. He must be some kind of a transvestite.' And throughout the next
ten days he kept introducing this man as Miss whatever his name was --
Frank or John -- `This is Miss John. I'd like you to meet my
transvestite translator. He can't quite make up his mind.' He was very
funny, but he was unstintingly honest. He really could make people look
at themselves and their attachments. When I was translating for him, he
said, `Even though I don't speak any English, I know the truth is that
my translator leaves out all the really hard things I say. I tell you
painful things and he leaves out all the things that have a sting in
them, makes them soft and gentle for you. You can't trust him.'

First come dignity and surrender -- really seeing the power of one's
willingness to live in a full way in the Dhamma. And secondly, one has
to learn to see honestly, to be honest about oneself and the people
around one, to see one's limits and not to be caught in the things
outside. When I asked what is the biggest problem with new disciples, he
said, `Views and opinions about everything. They are all so educated.
They think they know so much. When they come to me, how can they learn
anything? Wisdom is for you to watch and develop. Take from the teacher
what's good, but be aware of your own practice. If I am resting while
you all sit up, does it make you angry? If I say that the sky is red
instead of blue, don't follow me blindly. One of my teachers ate very
fast and made noises as he ate. Yet he told us to eat slowly and
mindfully. I used to watch him and got very upset. I suffered, but he
didn't. I watched the outside. Later, I learned. Some people drive very
fast but carefully, and others drive slowly and have many accidents.
Don't cling to rules or to form. If you watch others at the most ten
percent of the time and yourself ninety percent, this is proper
practice. First I used to watch my teacher, Ajahn Tongrat, and had many
doubts. People even thought he was mad. He would do strange things and
be very fierce with his disciples. Outside he was angry, but inside
there was nobody, nothing there. He was remarkable. He stayed clear and
mindful until the moment he died. Looking outside of yourself is
comparing, discriminating; you won't find happiness that way. No way
will you find peace if you spend your time looking for the perfect man,
or the perfect woman or the perfect teacher.'

The Buddha taught us to look at the Dhamma, the Truth, not to look at
other people, to see clearly and to see into ourselves; to know our
limits. Ram Dass asked him about limits. He asked, `Can you teach if
your own work isn't completed, if you're not fully enlightened?' And he
replied, `Be honest with them. Tell them what you know from your heart
and tell people what's possible. Don't pretend to be able to lift big
rocks when you can only lift small ones. Yet it doesn't hurt to tell
people that if you exercise and if you work, it's possible to lift this.
Just be straightforward and assess what's truly reasonable.' Surrender,
and dignity in that, and real impeccability: this is the ground. Then
there's clarity, seeing what's true in oneself, seeing one's limits,
seeing one's attachments.

Then the third way he taught was by working with things.

Working is done in two parts: one by overcoming obstacles and
hindrances, and the other by letting go. Overcoming: the first Dhamma
talk I gave was at a large gathering, \emph{M}ā\emph{gha Pūj}ā festival
day, and in a hall filled with 500 or 1,000 villagers. We sat up all
night, alternately sitting for one hour and then listening to a talk
given by one of the teachers from his monasteries. He had several
hundred monks there at that time; they all came together from the branch
monasteries for that day. And then in the middle of the night with no
preparation, he said, `Now we'll hear a Dhamma talk from the Western
monk.' I'd never given a Dhamma talk before, much less in Lao, the local
dialect. There was no time, I had to just get up and say what I could
say. He had his chief Western disciple, Ajahn Sumedho, get up and give a
talk. Ajahn Sumedho ended after an hour and Ajahn Chah said, `Talk
more.' So Ajahn Sumedho talked another half-hour; he didn't have much to
say, people were getting bored, he was getting bored, he finished. Ajahn
Chah said, `Now more.' Another half-hour, three-quarters of an hour, it
was getting more and more boring -- he'd run out of things to say.
People were sleeping; Ajahn Sumedho didn't know what to say, finally
finished, and Ajahn Chah said, `More, a bit more.' Another half-hour
--it was the most boring talk! And why would he do it? He got Ajahn
Sumedho to learn not to be afraid of being boring. It was wonderful.

He encouraged people to put themselves in situations where they were
afraid. He would send people who were afraid of ghosts to sit outside at
night in the charnel-ground. I would go sometimes -- because I wasn't
afraid of ghosts, it was a way of showing off -- but for them it was
really scary. Or he had people go away out in the forest and meditate,
and face the fear of tigers. The spirit of the practice was to really
make yourself work with things to overcome them. He pushed you into what
you disliked. If you liked to be alone in the forest, you were assigned
to a city monastery in Bangkok. And if you liked the city and the easy
life and good food, he'd send you to some impoverished forest monastery
where there were just rice and tree leaves to eat. He was a real rascal.
He knew all of your trips, and he could find them and he would somehow,
in a very funny and gentle and yet direct way, really make you look to
see where you were afraid or attached. Fear, boredom, restlessness --
fine, sit with it. Be bored, be restless and die, he would say over and
over again. Die in that restlessness, die in that fear, die in that
boredom. People were sleepy, great: the ascetic practice he'd assign
would be to sit up all night, and if you wouldn't sit, walk, walk more,
walk backwards if you were really sleepy. Whatever it took, to really go
against it.

With anger, restlessness, the same. He said, `You are restless. Fine, go
back and sit. Sit more when you are restless, don't sit less.' He said
it's like starving a tiger to death in a cage of mindfulness. It's not
that you need to do anything about the tiger -- the tiger being your
anger or restlessness -- just let it roam around in the cage. But you
make the cage around it with your sitting. He really made people look at
where they were, made them face it. But still, it was done with humour
and it was done with balance. He wouldn't allow people to do fasts,
except very rarely. He wouldn't even allow people to do long solitary
practice, unless he felt it was really good for them. Some people he'd
make work. `You need to know the strength of the ox-cart', he would say,
`and not overload it.' He made space for each person to grow at their
own pace. The first part of working was really working to overcome
difficulties. He said, `The way of Dhamma is the way of opposites. If
you like it cold you should have it hot, and if you like it soft, take
it hard.' Whatever it was, to be really willing to let go, to be free.

The second part of working was by the practice of real mindfulness, of
being aware of things and letting go of them. In terms of form, this
meant to let go of attachments to physical possessions. `Letting go',
however, also included matters of custom. I remember the villagers came
to complain to him because he'd set up what still exists as a monastery
for training Westerners, and these Westerners were celebrating
Christmas, with a Christmas tree and all. The villagers came and said,
`Listen, you told us we were going to have a forest monastery for
Buddhist monks by our village, and these Westerners are doing Christmas.
It doesn't seem right.' So he listened to them and said, `Well, my
understanding is that the teachings of Christianity are the teachings of
loving-kindness, of surrender and compassion, of seeing one's neighbour
as oneself, of sacrifice, of non-attachment -- many of the basic
principles of Buddha-Dhamma. For me, it seems all right that they
celebrate Christmas, especially since it is a holiday of giving and
generosity, of love. But if you insist, we won't celebrate Christmas
there any more.' The villagers were relieved. He said, `We'll have a
celebration, but instead let's call it ChrisBuddhamas.' And that was the
celebration. They were satisfied, and he was satisfied.

It wasn't as if the way to do it was through some particular form, but
to let go of form, to let go of doubt. He said, `You have to learn to
watch doubts as they arise. Doubting is natural; we all start off with
doubts. What's important is that you don't identify with them or get
caught up in endless circles. Instead, simply watch the whole process of
doubting. See how doubts come and go. Then you will no longer be
victimized by them.' To see them, to know them, to let go. The same with
judgement and fear -- to feel them, to experience them as physical
events, as mental states and yet not be caught. To eventually come to
see all of the energies -- the difficult ones of anger, fear,
sleepiness, doubt and restlessness; the subtle ones of our attachment to
pride or to stillness, quietness or even insight. Just to see them and
allow them to come and go, and come to a really profound kind of
equanimity.

He said, `Sitting for hours on end is not necessary. Some people think
that the longer you can sit, the wiser you must be. I've seen chickens
sitting on their nests for days on end. Wisdom comes from being mindful
in all postures. Your practice should begin as you wake up in the
morning and should continue until you fall asleep. Each person has their
own natural pace. Some of you may die at age 50, some at age 65, some at
age 90. So too, your practice will not be identical. Don't worry about
this. What is important is only that you keep watchful, whether you are
working, sitting or going to the bathroom. Try and be mindful and let
things take their natural course. Then your mind will become quieter and
quieter in any surroundings. It will become still, like a clear forest
pool. Then all kinds of wonderful and rare animals will come to drink at
the pool. You will see clearly the nature of all things in the world.
You'll see many wonderful and strange things come and go, but you will
be still. This is the happiness and understanding of the Buddha.'

\chapter{April 1994, Luang Por Chah's Relics}

\emph{In January1994 Ajahn Sumedho and Ajahn Attapemo went to Wat Pah
Pong in Thailand and took part in the final ceremonies to enshrine the
Atthi-dhātu (relics) of Luang Por Chah. ****Ajahn Attapemo}*\emph{*
explains:}

It was all done quite beautifully, stretching over seven days. Each day
there were periods of meditation and Dhamma talks. On the fourth day,
quietly, the abbots from most of the 152 branch monasteries gathered to
take the majority of the relics up to the (Thai for \emph{stūpa}, or
pagoda) built for the cremation last year. A chamber had been made
inside, into which three reliquaries were placed. To add to the
blessing, gold necklaces, bracelets and rings were draped over the
reliquaries. Some ladies even took their rings off their fingers to be
enshrined for posterity. Later that day this chamber was sealed with a
concrete lid and granite cap-stone. More than 30,000 people had gathered
for the occasion. The final ceremony took place on 16th January, exactly
two years after Luang Por Chah's death.

His Majesty King Bhumiphol sent his Chief Privy Officer to lead the
ceremony, and sent a royal invitation to Somdet Buddhajahn to give a
\emph{desanā}. A bronze and glass stupa nine feet high had been made for
the relics, and the Chief Privy Officer took a crystal platter with
thirty selected pieces of the relics and placed this into the glass
section of the stupa. Somdet Buddhajahn and twenty other important monks
invited to honour the occasion led the chanting of* `Jayanto', *along
with 1,200 more monks sitting inside and around the \emph{chedi}.

Along with the relics were the ashes. These were equally divided among
the 152 branch monasteries, including a small packet for every monk and
nun.

Also on that day, Ajahn Liem was officially appointed as Abbot of Wat
Pah Pong.

\setChapterNote{Greg Klein (Ajahn Anando) 3rd November 1946 -- 11th May 1994}
\chapter{July 1994}

Below,\textbf{* Ajahn Sucitto}* remembers Greg Klein, whose ashes were
interred at Cittaviveka on 17th July, when a plaque he had had made was
also laid.

Something he wrote about his time helping to nurse Luang Por Chah in his
terminal illness not only reflects his own interests, but sums up the
life mystery well. *`I like the early morning, the night shift as they
call it, very much, because one can spend time alone with Luang Por.
From 2 a.m. until maybe 5 a.m. is the time when he seems to sleep the
most peacefully. Then a rather busy time follows; depending on what day
of the week it is we might clean part of the room, very quietly, and
then prepare things for waking him at 5.30 to bathe and exercise him.
Then, the weather and his strength permitting, we put him in the chair,
the one that was sent from England with the money offered by people in
the West. It's a really superlative chair, it does everything except put
itself away at night! I had a look at what they had made for Luang Por
before. It was quite good for the materials they had, but the wheelchair
that he has now is in a class by itself. A sense of great respect and
affectionate caring goes into the nursing. Although he has been
bedridden for almost six years, he has no bedsores. The monks commented
that visiting doctors and nurses are quite amazed at the good condition
of his skin. The monks who are nursing him never eat or drink anything
or sleep in the room. There is very little talking; usually you only
talk about the next thing you have to do for his care. If you do talk,
you talk in a quiet manner. *

*So it is not just a room we nurse him in, it is actually a temple. One
of the senior Thai Ajahns asked me how I was feeling about being with
Luang Por. First I expressed my gratitude for the opportunity. He said,
`But how are you feeling?' I said, `Sometimes I feel very joyful, and
sometimes not so joyful.' I realized that this was going to be a Dhamma
discussion. He was using the opportunity to teach me something. He went
on to say there is a lot of misunderstanding about what is happening to
Luang Por. `Actually, it's just the sankhāras, the aggregates, going
through a certain process.' He said, `All we really need to do is just
let it go, let it cease; but if you did that people would criticize,
they would misunderstand and think you were heartless and cruel, and
that you would let him die. So because of that, we nurse him, which is
fine also.' He then went on to say that the reason we perceive things
the way we do is that we are still attached to our views and our
opinions. But they are not right, they still have the stench of self. He
said that Luang Por practised mettā bhāvanā, meditation on
loving-kindness, very much, and this is why people were drawn to him;
but that has a certain responsibility. `For myself,' he said, `I incline
quite naturally towards equanimity, serenity. There is no responsibility
there, it's light.' *

\emph{On the last morning, when I arrived at Luang Por's kuti, he was
lying on his side, and I just spent a long time sitting facing him, very
consciously directing thoughts of loving-kindness and gratitude towards
him, expressing my happiness at having had the great blessing of
spending some time with him, of having heard his teaching, appreciated
it and incorporated it into my life. The morning went by very easily and
rapidly. I was sitting looking at him comfortably asleep, and
considering how best to use this very special time. And the message was:
see it all as anicca, dukkha, anattā -- something impermanent, imperfect
and impersonal. That's what takes one beyond; it's all right.'}

\chapter{January 1997, Timeless Teachings}

\emph{These Dhamma reflections were published to commemorate the fifth
anniversary of Luang Por Chah's death. They come from a collection of
his teachings assembled by ****Paul Breiter}*\emph{* during the
seventies. They are presented as an expression of reverence and
gratitude.}

Everyone knows suffering -- but they don't really understand suffering.
If we really understood suffering, then that would be the end of our
suffering.

Westerners are generally in a hurry, so they have greater extremes of
happiness and suffering. The fact that they have many* kilesa*
(defilements) can be a source of wisdom later on.

To live the lay life and practise Dhamma, one must be in the world but
remain above it. \emph{Sīla}, beginning with the basic five precepts, is
the all-important parent of all good things. It is for removing all
wrong from the mind, removing that which causes distress and agitation.
When these basic things are gone, the mind will always be in a state of
\emph{samādhi.} At first, the basic thing is to make \emph{sīla} really
firm. Practise formal meditation when there is the opportunity.
Sometimes it will be good, sometimes not. Don't worry about it, just
continue. If doubts arise, just realize that they, like everything else
in the mind, are impermanent.

From this base *samādhi *will come, but not yet wisdom. One must watch
the mind at work -- see like and dislike arising from sense contact, and
not attach to them. Don't be anxious for results or quick progress. An
infant crawls at first, then learns to walk, then to run, and when it is
full grown, can travel half-way round the world to Thailand.

\emph{Dāna} (generosity), if given with good intention, can bring
happiness to oneself and others. But until \emph{sīla} is complete
giving is not pure, because we may steal from one person and give to
another.

Seeking pleasure and having fun is never-ending, one is never satisfied.
It's like a water jar with a hole in it. We try to fill it but the water
is continually leaking out. The peace of the religious life has a
definite end, it puts a stop to the cycle of endless seeking. It's like
plugging up the hole in the water jar!

Living in the world, practising meditation, others will look at you like
a gong which isn't struck, not producing any sound. They will consider
you useless, mad, defeated; but actually it is just the opposite.

As for myself, I never questioned the teachers very much, I have always
been a listener. I would listen to what they had to say, whether it was
right or wrong did not matter; then I would just practise. The same as
you who practise here. You should not have all that many questions. If
one has constant mindfulness, then one can examine one's own mental
states -- we don't need anyone else to examine our moods.

Once when I was staying with an Ajahn, I had to sew myself a robe. In
those days there weren't any sewing machines, one had to sew by hand and
it was a very trying experience. The cloth was very thick and the
needles were dull; one kept stabbing oneself with the needle, one's
hands became very sore and blood kept dripping on the cloth. Because the
task was so difficult I was anxious to get it done. I became so absorbed
in the work that I didn't even notice I was sitting in the scorching
sun, dripping with sweat.

The Ajahn came over to me and asked why I was sitting in the sun and not
in the cool shade. I told him that I was really anxious to get the work
done. `Where are you rushing off to?' he asked. `I want to get this job
done so that I can do my sitting and walking meditation,' I told him.
`When is our work ever finished?' he asked. Oh! \ldots{}This finally
brought me around.

`Our worldly work is never finished,' he explained. `You should use such
occasions as this as exercises in mindfulness, and then when you have
worked long enough, just stop. Put it aside and continue your sitting
and walking practice.'

Now I began to understand his teaching. Previously, when I sewed, my
mind also sewed, and even when I put the sewing away my mind still kept
on sewing. When I understood the Ajahn's teaching I could really put the
sewing away. When I sewed, my mind sewed; then when I put the sewing
down, my mind put the sewing down also. When I stopped sewing, my mind
also stopped sewing.

Know the good and the bad in travelling or in living in one place. You
don't find peace on a hill or in a cave; you can travel to the place of
the Buddha's enlightenment without coming any closer to enlightenment.
The important thing is to be aware of yourself, wherever you are,
whatever you're doing. \emph{Viriya}, effort, is not a question of what
you do outwardly, but just the constant inner awareness and restraint.

It is important not to watch others and find fault with them. If they
behave wrongly, there is no need to make yourself suffer. If you point
out to them what is correct and they don't practise accordingly, leave
it at that. When the Buddha studied with various teachers, he realized
that their ways were lacking but he didn't disparage them. He studied
with humility and respect for the teachers, he practised earnestly and
realized their systems were not complete, but as he had not yet become
enlightened, he did not criticize or attempt to teach them. After he
found enlightenment, he recalled those he had studied and practised with
and wanted to share his new-found knowledge with them.

We practise to be free of suffering, but to be free of suffering does
not mean just to have everything as you would like it, have everyone
behave as you would like them to, speaking only that which pleases you.
Don't believe your own thinking on these matters. Generally, the truth
is one thing, our thinking is another thing. We should have wisdom in
excess of thinking, then there is no problem. When thinking exceeds
wisdom, we are in trouble.

\emph{Tanhā} (desire) in practice can be friend or foe. At first it
spurs us to come and practise -- we want to change things, to end
suffering. But if we are always desiring something that hasn't yet
arisen, if we want things to be other than they are, then this just
causes more suffering.

Sometimes we want to force the mind to be quiet, but this effort just
makes it all the more disturbed. Then we stop pushing and \emph{samādhi}
arises. And then in the state of calm and quiet we begin to wonder,
'What's going on? What's the point of it?' \ldots{} and we're back to
agitation again!

The day before the first \emph{Sanghāyanā}, one of the Buddha's
disciples went to tell the Venerable Ananda: `Tomorrow is the Sangha
council, only \emph{arahants} may attend.' Venerable Ananda was at this
time still unenlightened. So he determined, `Tonight I will do it.' He
practised strenuously all night, seeking to become enlightened. But he
just made himself tired. So he decided to let go, to rest a bit as he
wasn't getting anywhere for all his efforts. Having let go, as soon as
he lay down and his head hit the pillow, he became enlightened.

External conditions don't make you suffer, suffering arises from wrong
understanding. Feelings of pleasure and pain, like and dislike, arise
from sense-contact -- you must catch them as they arise, not follow
them, not give rise to craving and attachment, which in turn cause
mental birth and becoming. If you hear people talking it may stir you up
-- you think it destroys your calm, your meditation; but you hear a bird
chirping and you don't think anything of it, you just let it go as
sound, not giving it any meaning or value.

You shouldn't hurry or rush your practice, but must think in terms of a
long time. Right now we have `new' meditation; if we have `old'
meditation, then we can practise in every situation, whether chanting,
working, or sitting in your hut. We don't have to go seeking for special
places to practise. Wanting to practise alone is half right, but also
half wrong. It isn't that I don't favour a lot of formal meditation
(\emph{samādhi}), but one must know when to come out of it -- after
seven days, two weeks, one month, two months -- and then return to
relating to people and situations again. This is where wisdom is gained;
too much \emph{samādhi} practice has no advantage other than that one
may become mad. Many monks wanting to be alone have gone off and just
died alone!

Having the view that formal practice is the complete and only way to
practise, disregarding one's normal life situation, is called being
intoxicated with meditation.

Meditation is giving rise to wisdom in the mind. This we can do
anywhere, any time and in any posture.

\chapter{October 1998, Ajahn Chah's Birthday}

\emph{*}Ajahn Viradhammo, ****who was visiting Thailand, passed on a
letter that he had written to the New Zealand Sangha about the
celebration of Ajahn Chah's birthday at Wat Pah Pong some time before
Luang Por passed away. *

The birthday celebrations at Wat Nong Pah Pong were a magnificent
tribute to Luang Por. There were over 600 \emph{bhikkhus} and
\emph{sāmaneras,} and a sea of white-robed nuns and laypeople around his
\emph{kuti} on the afternoon of the 16th. Thānavaro, you will remember
where you sat when Luang Por was brought outside in his wheelchair. That
grassy area was almost entirely filled with the ochre robe.

We bowed in unison and then Ajahn Maha Amon led the chanting,
\emph{`Mahā There pamaādena} \ldots{}' To my surprise Luang Por's voice
answered back (they played a tape over the public address system)
`*Yathā vārivahā * \ldots{}' Luang Por continued to sit in his chair (he
has no choice), and although I couldn't see his face clearly I'm sure he
put tremendous effort forth to acknowledge our devotion and gratitude.
All of this was of course very moving.

After some time we once again bowed in unison and Luang Por was taken
back to his bed for the past six years. In the evening we had chanting
and discourses through the night. The \emph{sālā} was overflowing with
lay people and \emph{bhikkhus}, with many sitting outside. The
north-east of Thailand is a very special place where so many people
still practise and live their religion with tremendous devotion and
sincerity.

After midnight it started to rain and by dawn there was water all around
(it has been an exceptionally wet year). In the morning before
\emph{pindapat} there was a \emph{dāna} offering of bowls, \emph{grots},
mosquito netting and white cloth to all of the senior monks of over
eighty branch monasteries. Just to make sure that there were enough sets
of requisites, the lay people from Bangkok put together 108 sets. The
abundance and volume of Buddhist devotion and generosity are astounding.

After this \emph{Mahā Sangha} offering we had a \emph{pindapat} around
Luang Por's museum. The line of \emph{bhikkhus} stretched from the old
\emph{sālā} to the museum. There was mud everywhere and a seemingly
endless circle of lay people offering rice into our bowls. When the meal
finally got under way there were six lines of monks and novices outside
the length of the eating hall, and a crammed two lines inside. After the
meal there were a few more formalities, parting words, and soon all the
visitors began to return to their respective monasteries all over
Thailand. This tribute was over and I wished you could both have been
here with me.

Luang Por's condition is uncertain, although most people say he is
weaker. The most notable difference from last year is in his eyes. The
pupils are rolled upwards and there is no longer any attention in his
eyes. One of the nursing monks said that sometimes he does focus his
eyes and look at what is around him, but this is more and more rare.
Whatever his physical condition, the power of his practice and teaching
is unmistakable. Equally impressive is the continuing dedication people
have to his way. There is much work to be done, and Luang Por's
impeccability forces one's attention inwards to the source of both
freedom and suffering.

\chapter{January 2004, Some Final Words}

\emph{Extract from a talk given by Luang Por Chah to a large gathering
of monks and laypeople at Wat Nong Pah Pong, recently translated by Paul
Breiter.}

In every home and every community, whether we live in the city, the
countryside, the forests or the mountains, we are the same in
experiencing happiness and suffering; but very many of us lack a place
of refuge, a field or garden where we can cultivate positive qualities
of heart. We don't have clear understanding of what this life is about
and what we ought to be doing. From childhood and youth until adulthood,
we learn to seek enjoyment and take delight in the pleasures of the
senses, and we never think that danger will threaten us as we go about
our lives, making a family and so on.

There is also for many of us, an inner lack of virtue and Dhamma in our
lives, through not listening to the teachings and not practising Dhamma.
As a result there is little wisdom in our lives, and everything
regresses and degenerates. The Buddha, our Supreme Teacher, had
loving-kindness (\emph{mettā}) for beings. He led sons and daughters of
good family to ordain, practise and realize the truth. He taught them to
establish and spread the teaching, and to show people how to live with
happiness in their daily lives. He taught the proper ways to earn a
livelihood, to be moderate and thrifty in managing finances and to act
without carelessness in all affairs.

The Lord Buddha taught that no matter how poor we may be, we should not
let it impoverish our hearts and starve our wisdom. Even if there are
floods inundating our fields, our villages and our homes to the point
where it is beyond our capability to save anything, the Buddha taught us
not to let the floods overcome our hearts. Flooding the heart means that
we lose sight of and have no knowledge of Dhamma.

Even if water floods our fields again and again over the years, or even
if fire burns down our homes, we will still have our minds. If our minds
have virtue and Dhamma, we can then use our wisdom to help us make a
living and support ourselves. We can acquire land again and make a new
start.

I really believe that if you listen to the Dhamma, contemplating it and
understanding it, you can make an end of your suffering. You will know
what is right to do, what you need to do, what you need to use and what
you need to spend. You can live your life according to moral precepts
and Dhamma, applying wisdom to worldly matters. Unfortunately, most of
us are far from that.

We should remember that when the Buddha taught Dhamma and set out the
way of practice, he wasn't trying to make our lives difficult. He wanted
us to improve, to become better and more skilful. It's just that we
don't listen. This is pretty bad. It's like a little child who doesn't
want to take a bath in the middle of winter because it's too cold. He
starts to stink so much that the parents can't even sleep at night, so
they grab hold of him and give him a bath. That makes him mad, and he
cries and curses his father and mother. The parents and the child see
the situation differently. For the child, it's too uncomfortable to take
a bath in the winter. For the parents, the child's smell is unbearable.
The two views can't be reconciled.

The Buddha didn't simply want to leave us as we are. He wanted us to be
diligent and work hard in ways that are good and beneficial, and to be
enthusiastic about the right path. Instead of being lazy, we have to
make efforts.

His teaching is not something that will make us foolish or useless. It
teaches us how to develop and apply wisdom to whatever we are doing:
working, farming, raising a family and managing our finances. If we live
in the world, we have to pay attention and know the ways of the world,
otherwise we end up in dire straits. When we have our means of
livelihood, our homes and possessions, our minds can be comfortable and
upright, and we can have the energy of spirit to help and assist each
other. If someone is able to share food and clothing and provide shelter
to those in need, that is an act of loving-kindness. The way I see it,
giving things in a spirit of loving-kindness is far better than selling
them to make a profit. Those who have \emph{mettā} don't wish for
anything for themselves. They only wish for others to live in happiness.

When we live according to Dhamma, we feel no distress when looking back
on what we have done. We are only creating good \emph{kamma}. If we are
creating bad \emph{kamma}, then the result later on will be misery. So
we need to listen and contemplate, and we need to figure out where
difficulties come from. Have you ever carried things to the fields on a
pole over your shoulders? When the load is too heavy in front, isn't
that uncomfortable to carry? When it's too heavy behind, isn't that
uncomfortable to carry? Which way is balanced and which way is
unbalanced? When you're doing it well, you can see it. Dhamma is like
that. There is cause and effect -- it is common sense. When the load is
balanced it's easier to carry. With an attitude of moderation our family
relations and our work will be smoother. Even if you aren't rich, you
will still have ease of mind; you won't need to suffer over them.

As we haven't died yet, now is the time to talk about these things. If
you don't hear Dhamma when you are a human being, there won't be any
other chance. Do you think animals can be taught Dhamma? Animal life is
a lot harder than ours, being born as a toad or a frog, a pig or a dog,
a cobra or a viper, a squirrel or a rabbit. When people see them they
only think about killing or beating them, or catching or raising them
for food. So we have this opportunity only as humans. As we're still
alive, now is the time to look into this and mend our ways. If things
are difficult, try to bear with the difficulty for the time being and
live in the right way, until one day you can do it. This is the way to
practise Dhamma.

So, I am reminding you all of the need to have a good mind and live your
lives in an ethical way. However you may have been doing things up to
now, you should take a look and examine to see whether what you are
doing is good or not. If you've been following wrong ways, give them up.
Give up wrong livelihood. Earn your living in a good and decent way that
doesn't harm others and doesn't harm yourself or society. When you
practise right livelihood, then you will live with a comfortable mind.

We should use our time to create benefit right now, in the present. This
was the Buddha's intention: benefit in this life, benefit in future
lives. In this life, we need to apply ourselves from childhood to study,
to learn at least enough to be able to earn a living, so that we can
support ourselves and eventually establish a family and not live in
poverty. But we sometimes lack this responsible attitude. We seek
enjoyment instead. Wherever there's a festival, a play or a concert,
we're on our way there, even when it's getting near harvest time. The
old folks will drag the grandchildren along to hear the famous singer.
`Where are you off to, Grandmother?' `I'm taking the kids to hear the
concert!' I don't know if Grandma is taking the kids or the kids are
taking her. It doesn't seem to matter how long or difficult a trip it
might be, they go again and again. They say they're taking the
grandchildren, but the truth is that they just want to go themselves. To
them, that's what a good time is. If you invite them to the monastery to
listen to Dhamma, to learn about right and wrong, they'll say, `You go
ahead. I want to stay home and rest\ldots{} I've got a bad
headache\ldots{} my back hurts\ldots{} my knees are sore\ldots{} I
really don't feel well\ldots{}.' But if it's a popular singer or an
exciting play, they'll hurry to round up the kids. Nothing bothers them
then. That's how some folk are. They make such efforts, yet all they do
is bring suffering and difficulty on themselves. They seek out darkness,
confusion, and intoxication on the path of delusion.

The Buddha teaches us to create benefit for ourselves in this life,
ultimate benefit, spiritual welfare. We should do it now, in this very
life. We should seek out the knowledge that helps us do it, so that we
can live our lives well, making good use of our resources, working with
diligence in ways of right livelihood.

The Buddha taught us to meditate. In meditation we must practise
\emph{samādhi}, which means making the mind still and peaceful. It's
like water in a basin. If we keep putting things in it and stirring it
up, it will always be murky. If the mind is always allowed to be
thinking and worrying over things, we will never see anything clearly.
If we let the water in the basin settle and become still, then we will
see all sorts of things reflected in it. When the mind is settled and
still, wisdom will be able to see things. The illuminating light of
wisdom surpasses any other kind of light.

When training the mind in \emph{samādhi}, we initially get the idea it
will be easy. But when we sit, our legs hurt, our back hurts, we feel
tired, we get hot and itchy. Then we start to feel discouraged, thinking
that \emph{samādhi} is as far away from us as the sky from the earth. We
don't know what to do and become overwhelmed by the difficulties. But if
we receive some training, it will get easier little by little.

It's like a city person looking for mushrooms. He asks, `Where do
mushrooms come from?' Someone tells him, `They grow in the earth.' So he
picks up a basket and goes walking into the countryside, expecting the
mushrooms to be lined up along the side of the road for him. But he
walks and walks, climbing hills and trekking through fields, without
seeing any mushrooms. A village person who has gone picking mushrooms
before would know where to look for them; he would know which part of
which forest to go to. But the city person has had only the experience
of seeing mushrooms on his plate. He heard they grow in the earth and
got the idea that they would be easy to find, but it didn't work out
that way.

Likewise, you who come here to practise \emph{samādhi} might feel it's
difficult. I had my troubles with it too. I trained with an Ajahn, and
when we were sitting I'd open my eyes to look: `Oh! Is Ajahn ready to
stop yet?' I'd close my eyes again and try to bear it a little longer. I
felt it was going to kill me. I kept opening my eyes, but the Ajahn
looked so comfortable sitting there. One hour, two hours; I would be in
agony, but the Ajahn didn't move. So after a while I got to fear the
sittings. When it was time to practise \emph{samādhi}, I'd feel afraid.

When we are new to it, training in \emph{samādhi} is difficult. Anything
is difficult when we don't know how to do it. This is our obstacle. But
with training, this can change. That which is good can eventually
overcome and surpass that which is not good. We tend to become
faint-hearted as we struggle -- this is a normal reaction, and we all go
through it. So it's important to train for some time. It's like making a
path through the forest. At first it's rough going, with a lot of
obstructions, but by returning to it again and again, we clear the way.
After some time, when we have removed the branches and stumps, the
ground becomes firm and smooth from being walked on repeatedly. Then we
have a good path for walking through the forest. This is what it's like
when we train the mind. By keeping at it, the mind becomes illumined.

So the Buddha wanted us to seek Dhamma. This kind of knowledge is what's
most important. Any form of knowledge or study that does not accord with
the Buddhist way is learning that involves \emph{dukkha}. Our practice
of Dhamma should get us beyond suffering; if we can't fully transcend
suffering, then we should at least be able to transcend it a little,
now, in the present.

When problems come to you, recollect Dhamma. Think of what your
spiritual guides have taught you. They have taught you to let go, to
give up, to refrain, to put things down; they have taught you to strive
and fight in a way that will solve your difficulties. The Dhamma that
you come to listen to is for solving problems. The teaching tells you
that you can solve the problems of daily life with Dhamma. After all, we
have been born as human beings; it should be possible for us to live
with happy minds.

\chapter{July 2007, Thirty years later}

**Questions and answers with Ajahn Sumedho **

*Q: It's 30 years since you came to England with Luang Por Chah. Why did
you leave Thailand? *

\emph{A}: In 1975 the Americans left Vietnam, Laos and Cambodia --
French Indochina. Those countries became communist, and there was a
`domino theory' everybody seemed to think would happen: that once those
countries fell, the whole of south-east Asia would follow. There was a
widespread fear that Thailand would be next. We had established Wat Pah
Nanachat for Westerners. I was the head monk and we had about twenty
Western monks there at the time, and I remember thinking, `What's going
to happen to us if Thailand goes communist?' So that was the catalyst
that started me thinking about the possibilities of establishing a
Buddhist monastery elsewhere. I'd never entertained such an idea, never
wanted to leave -- but because of this notion that Thailand would fall
to the communists, this thought came into my mind.

Shortly after that I was invited home because my mother was very ill and
they thought she might die, and it seemed to coincide with having that
thought. So when I went back to see my mother and father I thought,
`Well, if people are interested maybe we could set something up.' I
spent time with my parents in Southern California, and after my mother
seemed to get better I went on with Ven. Varapañño (Paul Breiter) to New
York and stayed with his parents. I went to Buddhist groups in
Massachusetts, where Jack Kornfield and Joseph Goldstein had just opened
the Insight Meditation Society. It was clear that was not to be a
monastic place. So nothing much happened in the States with respect to
people being interested in starting a monastery.

To get back to Thailand I had to go via London, and that's where I met
George Sharp.~ He was the chairman of the English Sangha Trust (EST) and
he seemed very interested in me. I stayed at the Hampstead Vihara, which
was closed; he opened it up for me. During the three days I was there he
came every evening to talk to me. Then he asked if I would consider
living in England, and I said, `Well, I can't really answer that
question, you'll have to ask my teacher, Luang Por Chah, in Thailand.'
And this he did; he came a few months later. Luang Por Chah and I were
invited to England, and we arrived on May 6th, 1977.

*Q: *And your idea was that, if Thailand fell to the communists, this
would be a way of preserving this monastic tradition?

\emph{A:} Yes. And the thing that impressed me was that the English
Sangha Trust had already been established 20 years before, in 1956, and
though it had tried all kinds of things, it was essentially a trust set
up to support Buddhist monks in England -- so it was for the Sangha.~
There was a movement to try to make it more a trust for supporting lay
teachers. But George Sharp had this very strong sense that the original
purpose of the EST was to encourage Buddhist monks to come and live in
England. Several years before he'd met Tan Ajahn Maha Boowa and Ajahn
Paññavaddho when they came to visit London. He consulted with them about
how to bring good monks to start a proper Sangha presence in England,
and Ajahn Maha Boowa recommended they just wait, not do anything and see
what happened. So George had closed the Hampstead Vihara until the right
opportunity arose. He wasn't prepared to put just anybody in there. I
think he saw me as a potential incumbent. Ajahn Chah was very successful
in training Westerners, and in inspiring Western men to become monks.
Wat Pah Nanachat was really quite a work of genius at the time. There'd
been nothing like it. That was Luang Por Chah's idea.

*Q: *What did he think about the idea of moving out of Thailand?

\emph{A:} When I went back to Thailand I told him about it, and of
course he never signified one way or the other in situations like that.
He seemed interested, but didn't feel a great need to do anything with
it. That's why it was important for George Sharp to visit, so that Luang
Por Chah could meet him. George was very open to any suggestions that
Ajahn Chah had. He had no agenda of his own, but he was interested in
supporting Theravadan monks living under the Vinaya system in England.
He'd seen so many failures in England over the previous twenty years;
there were many good intentions to establish something, but things just
seemed to fall apart. They'd send some Englishmen to Thailand for a
couple of years to become ordained and when they came back they'd be
thrown straight into a teaching situation or something they weren't
prepared for. They had no monastic experience except maybe a short time
in a Bangkok temple. So what impressed George was that by that time I'd
had quite a few years of training within the monastic system of Thailand
and in the Thai Forest tradition, so I wasn't just a neophyte --
although in terms of the way we look at things now, when I came to
England I had only ten \emph{Vassa}. I don't think any ten-\emph{vassa}
monk now would consider doing such an operation! Ajahn Khemadhammo came
a couple of weeks before, and then Ajahn Chah and I came together,
arriving on May 6th. Later Ajahn Anando and Viradhammo dropped in,
because they had gone to visit their families in North America. During
that time I suggested they stay, and Ajahn Chah agreed, so they stayed
on with me and there were four of us.

\emph{Q:} Did Ajahn Chah make a decision at some point, that yes, OK, it
would work?

\emph{A:} Well, when George Sharp came to see him in Thailand Ajahn Chah
put him through a kind of test. He was looking at George, trying to
figure out what sort of person he was. George had to eat the leftovers
at the end of the line, out of old enamel bowls with chips in them and
sitting on the cement floor near the dogs. George was a rather
sophisticated Londoner, but Ajahn Chah put him in that position and he
seemed to accept it. I didn't hear him complain at all. Later on we had
meetings, and George made a formal invitation and Luang Por Chah
accepted, agreeing that he and I would visit London the next May.

I was curious, because Luang Por Chah was so highly regarded in Thailand
that I wondered how he would respond to being in a non-Buddhist country.
There's no question of right procedure in Thailand in terms of monastic
protocol, but you can't expect that in other countries. What impressed
me during the time in England was how Luang Por responded to the
situation. Nothing seemed to bother him. He was interested, he was
curious. He watched people to see how they did things. He wanted to know
why they did it like this or that. He wasn't threatened by anything. He
seemed to just flow with the scene and be able to adapt skilfully to a
culture and climate he'd never experienced before in his life, living in
a country where he couldn't understand what anyone was saying.

He could relate well to English people, even though he couldn't speak a
word of English; his natural warmth was enough. He was a very
charismatic person in his own right, whether he was in Thailand or in
England, and he seemed to have pretty much the same effect on people,
whoever they were.

Every morning we went out on alms-round to Hampstead Heath. People would
come, Thai people -- and Tan Nam and his wife, that's where we met them.
They've been supporting us all these years. Generally our reception was
excellent. George Sharp's idea was to develop a forest monastery. He
felt that the Hampstead Vihara was a place that could not develop. It
was associated with a lot of past failures and disappointment, so his
idea was to sell it off in order to find some place in the countryside
that would be suitable for a forest monastery. Luang Por Chah said to
stay at the Hampstead Vihara first, to see what would happen. And it was
good enough in the beginning. But the aim was always to move out of
there, to sell it off and find a forest.

\emph{Q:} Did you feel confident that it would work? What were your
feelings at that time, after Luang Por Chah left?

*A: *I didn't know what was going to happen, and I wasn't aware of the
kinds of problems I was moving into, with the state of the English
Sangha Trust. I was quite naïve really. But I appreciated George Sharp's
efforts and intentions, and the legal set-up seemed so good: a trust
fund that had been established for supporting the Sangha. George seemed
to have a vision of this, rather than seeing us as meditation teachers
or just using us to spread Buddhism in Europe. I never got that
impression from him; in fact he made it very clear that if I just came
and practised meditation they'd support that, without even any
expectation of teaching. So right from the beginning it was made clear
that I wasn't going to be pushed around or propelled by people to fulfil
their demands and expectations. It seemed like quite a good place to
start outside of Thailand.

But when Luang Por Chah left -- he was only there for a month -- he made
me promise not to come back for five years. He said, `You can't come
back to Thailand for five years.'

Q: So he believed in the project at that point?

A: He seemed to. He was quite supportive in every way. So I said I would
do that, and I planned to stay.

George Sharp

I think it was in June 1976 when the phone rang and it was Ajahn
Sumedho. He had been given my telephone number by Ajahn Paññavaddho in
Thailand, who suggested he should give me a ring if he needed any
assistance. Principally he rang to say: `Could you give me a place that
is suitable for me to stay in for a few days?'~ I said: `Okay, I'll send
a taxi for you', which I did, and he arrived in no time. He was there
altogether about three days. ~

I had work to do, but in the evenings we would talk for hours. He told
me something about the tradition. I was very interested, and in the end
he said: `I invite you to come to Thailand and meet my teacher.' I said
I would, and in November of that year I got on a plane and went.

I thought Ajahn Chah might agree to having a go at starting a branch in
England, and I suspected he had a great deal of confidence in Ajahn
Sumedho. In fact, on one occasion when Ajahn Sumedho was translating, I
said to Ajahn Chah: `This is really quite a venture and, quite frankly,
Venerable Sumedho is going to have a very tough time at getting this
started. Now, I don't know anything about Venerable Sumedho. He comes to
me without any reputation whatsoever. But on the other hand, Ajahn Chah,
you are a great teacher, you have a considerable reputation and with
such a reputation this venture might have a chance of getting off the
ground. What can you tell me to give me confidence in the Venerable
Sumedho?' Ajahn Sumedho had to translate all this. Ajahn Chah said, `I
don't think he'll get married.' That was terrific, because that is what
all the previous \emph{bhikkhu}s had been doing at the Hampstead Vihara.

I came home knowing that Ajahn Chah was coming over and that he was
going to bring four with him. So what he was effectively doing was
bringing a Sangha to England. They were going to have a look at
Haverstock Hill, and he was going to make up his mind whether it was
worth a go or not. That is more or less what happened. He simply came in
and took over the place. In the end he apparently said they were to
stay.

Ajahn Sundara

I started cooking at 8 a.m. in the kitchen of the Hampstead Vihara, to
serve the main meal at 10.30. \emph{Anagārika} Phil (Ajahn Vajiro) and
Jordan (Ajahn Sumano) watched me prepare my favourite dishes and gave me
clues on how to go about in a place that was for me still very strange.
I was quite intent on my cooking, I wanted it perfect! After presenting
the whole meal to the monks, I felt so nervous and self-conscious that I
just ran downstairs and left! I had no idea of the Buddhist customs of
chanting a blessing, sharing food, etc.

When I first heard Ajahn Sumedho talk about his life as a forest monk in
Thailand I was stunned, because for a long time I had imagined a way of
life that would include all the qualities he was describing: where
intelligence and simplicity, patience and vitality, humour and
seriousness, being a fool and being wise, could all happily coexist.
During a later conversation he said, `It is a matter of knowing where
the world is, isn't it?' The penny had dropped: `I am the world!' I had
read and heard this truth many times, but I was truly hearing it for the
first time. That's when I decided to give monastic life a try, not
motivated by the desire to become a nun, but to learn and put into
practice the teaching of the Buddha. I had found my path.

Ajahn Vajiro

`Forest bhikkhus in London,' that's what I heard. I was excited by the
news. I bicycled from south of the river, up Haverstock Hill to number
131, a terraced house opposite the Haverstock Arms. The shrine room on
the second floor was as big as could be made from one floor of the
house. When Ajahn Chah was there the room was over-full, cramped and
stuffy. The talks were long and riveting. Tea was served in the basement
afterwards.

I was particularly struck by the way the \emph{bhikkhus} related to each
other, and especially how they related to Ajahn Chah. There was a
quality of care and attention which I found beautiful. I can remember
thinking, `I'll NEVER bow', and within a few weeks of watching and
listening, asking Ajahn Sumedho to teach me how to bow.

When I went to live at the Hampstead Vihara in early 1978, the place was
physically cramped, crowded and chaotic. It was not unusual for six men
to be sleeping in the shared anagārika and laymen's room on the top
floor. There were two WC's in the main building, one shower, a tiny
kitchen and the small basement room next to the kitchen served as the
\emph{dāna sālā}. What kept us there enduring the physical conditions
was the quality of the Dhamma. The \emph{pūjās} were early in the
morning and included a reflection nearly every day. And with the evening
\emph{pūjās}, talks again were almost daily.

The main reflection was on uncertainty. There was a confidence that
things would change, and a trust that if the cultivation of
\emph{pāramis} was sincere, the change would be blessed.

