% Title: Questons and Answers
% Forest Sangha Newsletter 1992 October

Q: There are those periods when our hearts happen to be absorbed in
things and become blemished or darkened, but we are still aware of
ourselves, such as when some form of greed, hatred, or delusion comes
up. Although we know that these things are objectionable, we are unable
to prevent them from arising. Could it be said that even as we are aware
of them, we are providing the basis for increased clinging and
attachment, and maybe putting ourselves further back than where we
started from? 

\emph{Luang Por Chah:} That's it! You must keep knowing them at that point; 
that's the method of practice. 

I mean that simultaneously, we are both aware of them and repelled by
them, but lacking the ability to resist them; they just burst forth. 

By then it's already beyond your capability to do anything. At that
point you have to readjust yourself and then continue contemplation. 
Don't just give up on them there and then. When you see things arise in
that way you tend to get upset or feel regret, but it is possible to say
that they are uncertain and subject to change. What happens is that you
see these things are wrong, but you are still not ready or able to deal
with them. It's as if they are independent entities, the leftover kammic
tendencies that are still creating and conditioning the state of the
heart. You don't wish to allow the heart to become like that, but it
does, and it indicates that your knowledge and awareness are still
neither sufficient nor fast enough to keep abreast of things. 

You must practise and develop mindfulness as much as you can in order to
gain a greater and more penetrating awareness. Whether the heart is
soiled or blemished in some way, it doesn't matter; whatever comes up, 
you should contemplate the impermanence and uncertainty of it. By
maintaining this contemplation at each instant that something arises, 
after some time you will see the impermanence of all sense objects and
mental states. Because you see them as such, gradually they will lose
their importance, and your clinging and attachment to that which is a
blemish on the heart will continue to diminish. Whenever suffering
arises, you will be able to work through it and readjust yourself, but
you shouldn't give up on this work or set it aside. You must keep up a
continuity of effort and try to make your awareness fast enough to keep
in touch with the changing mental conditions. It could be said that so
far your development of the Path still lacks sufficient energy to
overcome the mental defilements; whenever suffering arises, the heart
becomes clouded over. But one must keep developing that knowledge and
understanding of the clouded heart; this is what you reflect on. 

You must really take hold of it and repeatedly contemplate that this
suffering and discontentment are just not sure things. They are
something that is ultimately impermanent, unsatisfactory, and not-self. 
Focusing on these three characteristics, whenever these conditions of
suffering arise again, you will know them straightaway, having
experienced them before. 

Gradually, little by little, your practice should gain momentum, and as
time passes, whatever sense objects and mental states arise will lose
their value in this way. Your heart will know them for what they are and
accordingly put them down. When you reach the point where you are able
to know things and put them down with ease, they say that the Path has
matured internally and you will have the ability to bear down swiftly
upon the defilements. From then on there will just be the arising and
passing away in this place, the same as waves striking the seashore. 
When a wave comes in and finally reaches the shoreline, it just
disintegrates and vanishes; a new wave comes and it happens again, the
wave going no further than the limit of the shoreline. In the same way, 
nothing will be able to go beyond the limits established by your own
awareness. 

That's the place where you will meet and come to understand
impermanence, unsatisfactoriness and not-self. It is there that things
will vanish -- the three characteristics of impermanence, 
unsatisfactoriness and not-self are the same as the seashore, and all
sense objects and mental states that are experienced go in the same way
as the waves. Happiness is uncertain; it's arisen many times before. 
Suffering is uncertain; it's arisen many times before. That's the way
they are. In your heart you will know that they are like that, they are
`just that much'. The heart will experience these conditions in this
way, and they will gradually keep losing their value and importance. 
This is talking about the characteristics of the heart, the way it is. 
It is the same for everybody, even the Buddha and all his disciples were
like this. 

If your practice of the Path matures, it will become automatic and it
will no longer be dependent on anything external. When a defilement
arises, you will immediately be aware of it and accordingly be able to
counteract it. However, that stage when they say that the Path is still
neither mature enough nor fast enough to overcome the defilements is
something that everybody has to experience -- it's unavoidable. But it
is at this point that you must use skilful reflection. Don't go
investigating elsewhere or trying to solve the problem at some other
place. Cure it right there. Apply the cure at that place where things
arise and pass away. Happiness arises and then passes away, doesn't it? 
Suffering arises and then passes away, doesn't it? You will continuously
be able to see the process of arising and ceasing, and see that which is
good and bad in the heart. These are phenomena that exist and are part
of nature. Don't cling tightly to them or create anything out of them at
all. 

If you have this kind of awareness, then even though you will be coming
into contact with things, there will not be any noise. In other words, 
you will see the arising and passing away of phenomena in a very natural
and ordinary way. You will just see things arise and then cease. You
will understand the process of arising and ceasing in the light of
impermanence, unsatisfactoriness, and not-self. 

The nature of the Dhamma is like this. When you can see things as `just
that much', then they will remain as `just that much.' There will be
none of that clinging or holding on -- as soon as you become aware of
attachment, it will disappear. There will be just the arising and
ceasing, and that is peaceful. That it's peaceful is not because you
don't hear anything; there is hearing, but you understand the nature of
it and don't cling or hold on to anything. This is what is meant by
peaceful -- the heart is still experiencing sense objects, but it
doesn't follow or get caught up in them. A division is made between the
heart, sense objects and the defilement; but if you understand the
process of arising and ceasing, then there is nothing that can really
arise from it -- it will end just there. 
