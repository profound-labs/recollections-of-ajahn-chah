
\begin{quote}\itshape
The first chapter of this book has been adapted from a series of transcribed interviews which were conducted by Ajahn Kongrit Ratanawanno during his time at Amaravati Buddhist Monastery, UK. Ajahn Kongrit's home monastery is Wat Beung Saensook, Thailand. In most cases the interviewee was asked the simple question about what had inspired them most in being with Ajahn Chah.
\end{quote}

\subsection{Ajahn Sumedho}

Luang Por Chah had a great deal of \emph{mettā} (loving-kindness) and I
felt welcomed by the way he received me at Wat Pah Pong -- he seemed to
be interested in me. I felt intuitively that this was a very wise man. 
At the time I couldn't understand Thai very well, but what I saw of how
he lived his life and his general way of being was very pleasing to me. 
His teaching was very direct and he was able to see very quickly where I
was at. 

He didn't want me to spend time reading or studying, just to practise. 
He emphasized everybody's \emph{paṭipat} (practice). When I first came
to him, he told me to put my books away and to just read the
\emph{citta}, my mind. I was happy to do that, because I was weary of
studying Buddhism and wanted to practise it instead of just reading
about it. This is what he was encouraging me to do. 

Though he gave a lot of talks, which I couldn't properly understand for
the first two years, he emphasized \emph{kor wat} (monastic duties), the
way you live in the monastery: paying attention, being mindful with food
and the robes, and with the \emph{kuṭī} (hut) and the monastery. He was
like a mirror that would reflect my state of mind. He always seemed to
be completely present. I'd get carried away with thoughts and emotions
sometimes, but by just being around him, I found that I could suddenly
let go -- I could drop what I was holding onto without even telling him. 
His presence helped me to see what I was doing and what I was attached
to. So I decided that I would live with him as long as I could, since
such monks are hard to find. I stayed with him for ten years at Wat Pah
Pong and at various branch monasteries. 

\subsection{Ajahn Pasanno}

I cannot say there is really any single thing that impressed me most, 
there were many things that impressed me. Certainly Luang Por Chah set
an example for us in the sense that he didn't just teach from theory; 
when he taught he was always present -- he was an example of what was
skilful and beautiful. 

The images that come to mind are of Luang Por himself being a great
teacher and everybody respecting him so much. But I also remember a
senior monk coming to visit Wat Pah Pong, and Luang Por paying respects
to and looking after this monk. Seeing the teaching in action without
his `being the Teacher' really impressed me. It was a very direct
teaching on not-self and a living example of the ease and freedom that
comes from penetrating not-self: neither a theory, nor a Buddhist
philosophy. That was the way he taught us: living the example, rather
than just giving us the philosophy. He had a great ability to teach and
draw people to the Dhamma by using these ordinary life situations. 

I remember one time we were coming back from \emph{piṇḍapat} (alms-round)
and I was walking along behind him. My Thai was not so good, so I
was just being respectful and walking close by him. We came in from the
back of the monastery, and as we were walking through the forest, two
lizards fell from a tree. Luang Por looked, then turned and said, `See
those lizards, they were mating. If they weren't caught in sensuality
they wouldn't have fallen and hurt themselves like this!' It was very
simple, and for a new monk a very funny and direct teaching. 

These are very real situations, very ordinary, and very to the point. 
Luang Por's ability to give examples and point to the things around us
empowered us to see Dhamma ourselves, rather than looking to scripture
or looking to him. To see that Dhamma is all around us and is something
we can see for ourselves was very empowering. It was both direct and had
that human quality. 

This `humanness' of Luang Por Chah was really quite striking. One time
he had some skin problems and I was helping him put ointment on the
inflammation. I would have to take off his \emph{sabong} (under-robe) to
spread the ointment all around his bottom, back and legs. And he asked, 
`Look at my bottom, does it look beautiful?' Then he would say, `It is
not beautiful, nobody would want it like this! Everybody who gets old, 
they all look like this.' Again, this is taking the ordinary and making
it something that allows us to relinquish, to let go. 

There was also his extraordinary generosity: his willingness to give of
himself, to give to people, his compassion. That was always very
touching. He never really put himself first. There was one year I was
living at Wat Pah Pong and acting as his attendant. I had been a monk
for many years by then and my Thai was very good, so I could understand
his teaching and what he was doing. I used to stay with him until
night-time, and put him to bed and massage him. It would be very rare
for him to go to bed before midnight, and sometimes he would be up until
1 or 2am. Yet he was always willing to help people who were interested
in Dhamma; to give, to teach, to train, and never thought about keeping
anything for himself -- complete relinquishment, complete renunciation. 
It was very powerful. 

But it was very difficult to be his \emph{upatthak} (attendant)! It was
really hard work because he never had a schedule, just responding to
situations in an appropriate way. His flexibility came from generosity
and compassion, not from any logical sequence of how things should be. 
That was always very impressive. So there are many different aspects of
living with Luang Por Chah. It's difficult to pin it down to just one. 
If you ask me the same question tomorrow, different things will come to
mind. 

\subsection{Ajahn Ṭhiradhammo}

The most meaningful and impressive aspect to me was that he was a living
embodiment of the Buddha's teachings, which I had only previously read
about and understood conceptually. 

The first meaningful example was when I went to live at Wat Pah Pong. I
thought that if I was living in the monasteries under his guidance, I
should get to know who this great teacher Ajahn Chah was and what his
basic teaching was. I arrived there a month before the Rains Retreat
began, when there was a less formal schedule. Thus in the evening one of
the best learning situations was to sit at Ajahn Chah's hut and listen
to him interact with visitors and resident monks. Since my Thai was
passable I could understand most of what was said. However, as I
listened to Ajahn Chah's advice, counsel and teachings, I began to feel
more confused about who he was and what his teaching was. I noticed that
he gave different teachings to different people, sometimes even giving
contrary advice. To me he thus came across as being inconsistent. So
what was it? Was he just putting on a front, or was he confused? This
presented something of a spiritual dilemma for me. On the one hand Ajahn
Chah was obviously an inspiring teacher, displaying considerable wisdom
and charisma. On the other hand, his teachings were not consistent with
how I thought an `enlightened being' should be. 

Then one evening as I listened to him, it suddenly occurred to me that
there was \emph{no fixed and consistent Ajahn Chah.}
Rather than being a person
with a particular teaching, he was actually just responding with
mindfulness and wisdom to whatever situation arose. His apparent
inconsistency was in effect a specific wise response to what the
particular person or situation required at that time. I had previously
been relating to Ajahn Chah as someone with a stable personality and a
set body of beliefs and views. Now it dawned on me that he was not
holding on to a fixed personality or definite views, but was the living
expression of mindfulness and wisdom. What appeared to be inconsistency
on the conventional level was in truth a relevant and immediate response
to whatever was happening at the time. To me this was a living example
of impersonality. 

Another example which was exceptionally helpful to me personally was
when I was bothered by the phenomenon of people's faces coming up in
meditation. They were not usually frightening, but just bothersome and
distracting. I wasn't sure what this meant or what caused it, and became
preoccupied with trying to understand or do something with it. 
Fortunately I was able to ask Ajahn Chah about this. He called it
`mental phenomena' and said, `Just observe it, and don't be fascinated
by it. Know it and go back to the breathing.' He explained that we can
become attracted by such things because they're new and interesting. He
said that I might either become quite excited about them, thinking I had
psychic powers like precognition, seeing the face of someone who next
day might offer food, or I might think that maybe ghosts were haunting
me. This was the best and most useful advice I had received from any
teacher, and when I could apply it, the faces eventually faded away. And
this principle has been very helpful for me in dealing with many of the
unusual phenomena which arise in spiritual practice. 

\subsection{Ajahn Sucitto}

The first time I saw Luang Por Chah was when he landed in Britain, when
he came through the arrivals at Heathrow Airport. There was a group of
us monks: Ajahn Sumedho, Ānando, Viradhammo, and myself. Ajahn Pabhākaro
was with Luang Por Chah. The first thing that I noticed about him was
that he was quite small, particularly compared with Ajahn Pabhākaro. But
he looked like a very, very big man -- he carried himself like a big
man; not aggressive, but completely confident. He looked like he had a
lot of space inside him. 

Here he was in a foreign country, he'd come from a long plane journey, 
couldn't speak the language, but he looked completely in charge and he
knew exactly where he wanted to be. He was not hurried. He was not
anxious. He was balanced in himself and looked warm and friendly -- not
in charge in a hard way, but at ease within his environment. Whenever we
came to see him, he was receptive; he knew how to receive people. He was
like your favourite uncle, as if you'd just been talking to him and
you'd known him all your life; very easy, very warm and you immediately
felt very relaxed. Normally when you meet somebody who's strange, you
think, `Better make sure everything's all right\ldots{}' But with him
you felt relaxed because there was the presence of \emph{mettā} --
immediately. This was overwhelming in some ways because usually almost
everybody takes a little bit of time before they warm up.

He stayed at the Hampstead Vihāra in London. This was just a small town
house. Compared with the big space of Wat Pah Pong it had very narrow
corridors and small rooms, and it was crowded. Yet he was comfortable
there. He had women sitting quite close by him, but it was no problem. 
People were not doing things properly according to the Thai way of doing
things -- not deliberately, but just not doing things in the proper way. 
And I could sense that some of the monks were quite anxious to make sure
it was all right, but he seemed to stay at ease. 

When people asked him questions, of course he couldn't understand their
words. So Ajahn Sumedho or Ajahn Pabhākaro would translate -- but he
kept his focus on the questioner. If somebody would ask some very
complicated question -- about the Abhidhamma, for example -- he
would respond to the questioner rather than the question, saying things
like, `Thinking too much is not good for you', or, `Sometimes it's like
this and sometimes it's like that.' It was always a very simple answer
that went deeper than the question. It went straight to the heart. He
was never fooled by any of the questions; he always went straight to the
heart. He could feel where people were coming from. 

He was very kind: often humorous, but not dismissive. He never wavered
from being receptive and patient. 

People would be affected by that. People could feel that immediate heart
contact and the effect was amazing. Sometimes the place would be crowded
with people who'd sit there just so they could be there. They didn't
have any questions. They just wanted to be there, just to feel that
heart contact. People are usually nervous, tense and anxious, so to be
in a place where there was somebody like this, offering this ease and
clarity, was a blessing. You couldn't understand what he was saying and
you didn't have anything to ask, but still you wanted to be there. It
would go on for hours. He never seemed to change his pace. He never
hurried; he never hung back. Everything was just flowing. Never
hurrying, never stopping, and never moving back. It was always flowing
along, like still flowing water. That image is what he was like: still
flowing water. 

\subsection{Ajahn Munindo}

During the time I was with or nearby Luang Por Chah, I was aware he was
making a powerful impression on me, but it was only many years later
that I became clearer about just what it was that had been impressed
upon me. At the time of living in Thailand it was perhaps more like an
intuition of the `rightness' of staying there, even though it was
certainly not easy. 

I heard that somebody once asked Luang Por Chah, `How come, out of all
the monks in Thailand, you stand out as different?' Luang Por replied,
`I was willing to be daring. Others wouldn't dare do as I did.' I didn't
hear this exchange directly, it was reported to me later, but it had a
significant effect on my own attitude to practice. It signalled where
the priority lay. Knowing this about his attitude helped me to
understand his teachings better. 

Luang Por Chah wasn't worried about being popular or famous or rich, or
having lots of disciples. If he felt that something was right and should
be done, he would do it. Sometimes that took daring. From the stories of
his experiences in practice it was clear that he had to dare to confront
his own fears and resistances. He had to dare so as not to be
intimidated by the things that normally limit others. He had to dare to
contradict the views of others, even when they were strongly held. 

During the five years I was near him, the thing that continually
inspired me was how totally agile he was. My recollection of how he
handled situations stays with me and serves as a valuable support in
dealing with all that we have to face here in the West. I think I had
some sense of the way he just flowed, without resistance. Whether it was
important dignitaries coming to visit, or a simple villager who was
concerned about a sick water buffalo, or rich supporters from Bangkok, 
he always had the same beautiful ability to `go with it'. Sometimes he
would be surrounded by a large gathering of monks hanging on his every
word, and at other times he might just be sitting on his own with one or
two young monks, chewing betel nut and drinking coca cola. He was always
able to adjust without stress. There were none of the tell-tale signs of
clinging which produce suffering in an individual and generate an
atmosphere of artificiality. He was as natural as I could wish a human
being to be. I don't think I have ever seen anyone so thoroughly normal. 
Luang Por was at home wherever he went, whatever he did. He could be
quiet and sensitive when you went to see him about some personal
struggle, and a few minutes later he would be shouting orders at the
huge crowd of soldiers who had come to help build his new temple. 

This teaching example identified for me how much resistance I still had, 
and that this struggling `for' and `against' life was the source of the
problem. Sometimes we think our difficulties are caused by external
circumstances, but usually the biggest cause is our inner habits of
clinging. Luang Por didn't show any signs of resistance and accordingly
didn't manifest suffering. This state of non-suffering was real for him, 
and it was remarkable how evident it also was outwardly. Because he had
settled the great questions in his own heart, he was a catalyst for
harmony and well-being in the outer world. To have had the good fortune
to witness that was a blessing. 

\subsection{Ajahn Amaro}

One of the most impressive things about Luang Por Chah was the way that
he could display authority without being authoritarian. He was a very
good leader but not someone who had to dominate people. I didn't live
with him for a long time, and maybe the very first time I had an
exchange with him was in about April or May 1978, when I was an
\emph{anāgārika} (postulant) and Luang Por was staying with us at Wat
Pah Nanachat. As an \emph{anāgārika} I was the attendant to Ajahn
Pabhākaro, who was the abbot of the monastery. So it was my job to get
his robes and bowl ready for \emph{piṇḍapat} in the morning. I never
found it easy to get up early in the morning; I still don't. Morning is
not my best time -- I can do it as an act of will, but I have to make
the effort. 

On this particular morning I woke up and saw light coming through the
gaps between the planks of the walls. I thought, `Wow, the moon is
really bright tonight.' Then I looked at the clock and saw that it must
have stopped, and I realized, `That's not the moon; that's the sun.' So
I leapt up, threw my clothes on and raced down the path. When I got to
the back of the \emph{sāla} (main hall), all the other people had
already gone out for \emph{piṇḍapat}, but Ajahn Pabhākaro and Luang Por, 
who were going out on a nearby \emph{piṇḍapat}, still hadn't left. I
thought, `OK, I've still got time. Maybe they didn't notice.' I then
realized it was twenty-five past and they were going to leave at
half-past. So I got their robes, hoping they hadn't noticed I'd arrived
late and had missed the morning chanting and sitting. While I was down
by Ajahn Chah's feet tying up the bottom end of his robes, he said
something in Thai which I couldn't understand. I looked up slightly
anxiously at Ajahn Pabhākaro for translation. Ajahn Chah had a big grin
on his face, an incredibly friendly, loving smile. Then Ajahn Pabhākaro
translated, `Sleep is delicious.' That was the first time in my life
when I did something wrong, but instead of being criticized or punished
was met by an extraordinarily loving attitude. It was at that point that
something in my heart knew Buddhism was really very different from
anything I had encountered previously. 

Luang Por was also very flexible. He had no respect for time. And he
didn't have any respect for logical consistency. He could change his
mind or his approach in a finger-snap. A couple of years later, when
Ajahn Sumedho was starting up Chithurst monastery, I was thinking of
going back to England to visit my family. I got a telegram saying my
father was very ill with a heart attack, so I came down from Roi-Et and
then to Wat Pah Pong to pay respects to Luang Por and ask his advice. I
felt I should leave for England soon, but my question was how I should
go about this. My Thai was pretty poor, and on that occasion Ajahn
Jāgaro was translating. I explained to Luang Por that I only had one
Rains Retreat as a monk and that I was from England; my family lived
quite near Chithurst and my father had just had a heart attack and was
very sick, and what did Luang Por think I should do?

He spoke for about twenty minutes -- it was a long speech and I didn't
really catch much of it. At the end, Ajahn Jāgaro said, `Well, he said
four things. 

`\thinspace ``Go to England and when your visit to your family is finished, go and
pay your respects to Ajahn Sumedho and then come straight back to
Thailand.'

`\thinspace ``Go to England and stay with your family and when your business with
your family is finished, go to stay with Ajahn Sumedho for a year and
then after that year you should come back to Thailand.'

`\thinspace ``Go to England, stay with your family, when your business with your
family is finished, go stay with Ajahn Sumedho and help him out. If it
gets too difficult, you can come back to Thailand if you really want
to.'

`\thinspace ``Go to England, when the business with the family is finished, go
and stay with Ajahn Sumedho and don't come back.''\thinspace '

The whole talk was delivered with exactly the same expression. It wasn't
as if any one option was preferable. As he was speaking, each single
option was an absolutely sincere piece of advice, a directive: `Do this. 
These are your instructions. Follow them to the letter!' And he wasn't
trying to be clever. It was obvious that he was being absolutely
straightforward.

Related to that was his quality of being transparent as a person. 
Someone once asked me to take a message to him, saying that some people
had just arrived at the \emph{sāla} and could he come to meet them. So I went
to his \emph{kuṭī}, where he was sitting on his rattan bench with his
eyes closed. There was no one else around. I went up and knelt in front
of him and he didn't open his eyes. So I waited a few minutes, wondering
what to do, but he still didn't open his eyes. So I said (in Thai), 
`Excuse me, Luang Por' and he opened his eyes. But it was as if there
was absolutely nobody there. He wasn't asleep; his eyes opened, but
there was no expression on his face. It was completely empty. He looked
at me, and I looked at him and said, `Luang Por, Ajahn Chu asked me to
bring a message that some people have come to the \emph{sāla} and would
it be possible for you to come and receive them?'

Again for a moment there was no expression, just this completely
spacious, empty quality on his face. Then out of nowhere, the
personality appeared. He made some remark that I didn't quite catch and
it was as if suddenly the `person' appeared; it was like watching a
being coming into existence.

There was an extraordinary quality in that
moment, seeing a being putting on a mask or a costume, as if to say, 
`OK, I'll be Ajahn Chah. I can play at being Ajahn Chah for these
people.' You could see that assumption of the personality, the body, all
the characteristics of personhood just being taken up as if he was
putting on his robe or taking up a role for the sake of emerging and
contacting other people. It was very powerful, seeing that `something'
coming out of nothing; seeing a being appearing before your eyes. 

\subsection{Ajahn Jayasāro}

I arrived at Wat Pah Pong in December 1978. It was the \emph{uposatha}
 (observance) day. I was already an \emph{anāgārika} but I hadn't shaved
my head. I had been travelling. One of the Western monks, Tan Pamutto, 
took me to his \emph{kuṭī} and shaved my head, and then we went to pay
respects to Luang Por. The moment I saw him I had a very strong feeling
that he would be my teacher, and that I didn't need to go anywhere else. 

Before I left England, Ajahn Sumedho gave me a piece of advice. He said, 
`Don't look for the perfect monastery, it doesn't exist.' Even so I got
a little side-tracked and went to stay with another teacher for a few
days. But then I came back to Wat Pah Pong and thought, `Now I can stop
travelling.'

I felt Luang Por was unlike anybody I had ever met before. I felt he was
the only totally normal person I had ever met -- everyone else was a bit
abnormal compared to him! It felt as if I'd spent my whole life
listening to people singing just a little bit out of tune, and this was
the first time I'd ever heard someone sing in tune. Or as if I'd grown
up in a country that only had plastic flowers, and then one day I
finally saw a real flower -- `Ah, so that's what a real flower is. I've
only ever seen plastic flowers before.' Plastic flowers can be
beautiful, but they're nothing like real flowers.

\emph{Question:} Ajahn Chah couldn't speak English, and you, when you
came, couldn't speak Thai, so how did you learn from him?

\emph{Answer:} The teaching that you receive in a \emph{desanā} (Dhamma
talk) and in other verbal teachings is only one part of what you get
from a teacher. From the very first day, the thing that I received from
Ajahn Chah, and the thing that impressed me most, was this very strong
confidence that he was an enlightened being, and therefore that
enlightenment is real and possible. I had that belief before from books
I'd read, and to a certain extent from other teachers, but it was only
when I met Ajahn Chah that this became really grounded in my being, this
confidence that the path to Nibbāna can still be followed and that it is
possible to realize all the fruits of the Holy Life. So I was impressed
by who Ajahn Chah was, his being, as much as by his teaching. Of course
I was very inspired by his teachings, and there are many teachings that
I treasure and have made great use of in my practice.

When you become a monk, you go through periods of feeling very positive,
and you can also go through periods where you feel discouraged and very
unhappy. I think if you look closely at what sustains you when you feel
down, it's not so much the wise teachings and reflections as the faith
that what you're doing is really meaningful, and that the path of
practice does lead to Nibbāna. I've never had any disrobing doubts since
I became a monk. Other monks who understood or studied the teachings
more than I have disrobed. It didn't help them. But because I had the
presence of Ajahn Chah and afterwards the memory of Ajahn Chah, it
seemed to me there's no alternative, there's nothing else that makes
sense except to be a monk and to follow this path.

I also loved his being and how he expressed himself, his voice. If you
gave Ajahn Chah a newspaper to read out loud, or if he were just to read
names from a telephone directory, I could still listen to him for hours.

\subsection{Ajahn Khemanando}

Most of my own personal experience of Ajahn Chah comes from the period, 
beginning in January 1979, when I came to stay at Wat Pah Pong as a
layman, followed by many months as an \emph{anāgārika} or \emph{pa-khao}
 (postulant). I was a newcomer to Thailand and monastic life, and spoke
or understood very little Thai, being quite dependent on the more senior
Western monks for translations and explanations of what was happening. 
So my impressions from that time were not so much of profound dialogues
or specific instructions on meditation, but more revelations of Ajahn
Chah's character, which would often overturn my own pre-conceptions
about the nature of an enlightened being, whilst also, sometimes
simultaneously, providing evidence that he did indeed function on quite
a different level from the people by whom he was surrounded; apparently
small incidents in which Ajahn Chah would do things that didn't need
explaining, which I was able to observe to gain some food for thought. 

Once I and a fellow \emph{pa-khao}, a New Zealander, were whiling away a
hot, steamy afternoon in idle conversation on the balcony of my
\emph{kuṭī}. At Wat Pah Pong in those days, much of the formal practice
was done as a group activity in the main hall morning and evening, while
your individual \emph{kuṭī} was kind of sacrosanct, where you could
expect to be left to your own devices most of the time. We had adjacent
\emph{kuṭī} in a far corner of the monastery and had become friends, 
offering each other companionship and support in this way, basically
relaxing and goofing off. So you can imagine how surprised and guilty we
felt when Ajahn Chah himself suddenly appeared on the path to the
\emph{kuṭī}, calling out and beckoning with his hand! We thought we were
in for a scolding for not meditating diligently, but Ajahn Chah didn't
seem bothered at all, he wasn't telling us to stop talking, but calling
to us, `Come here, come here!'

It transpired that Ajahn Chah was taking time off from being the
resident sage of Wat Pah Pong, receiving a constant stream of visitors
at his \emph{kuṭī}, and had decided to go hunting for monitor lizards
instead! Having just spotted one in the vicinity, he had come to enlist
our help, patiently miming an explanation of how to fix a string snare
to the end of a bamboo pole. Ajahn Chah was very fond of the forest
chickens, which he would feed with rice in the area around his own
\emph{kuṭī}. He wanted to protect them from their natural enemy, the
large monitor lizards which liked to eat their eggs. 

So there followed what turned out to be a hilarious scene of two rather
clumsy, inexperienced Westerners, goaded on by an enthusiastic Ajahn
Chah, their adopted spiritual guide, thrashing around in the forest
trying to catch a big lizard -- hardly the sort of thing that I had
imagined writing home about! We were quite hopeless, of course, and
eventually gave up without catching anything, but not before having a
good laugh at ourselves. 

What struck me most about this little episode was the contrast between
Ajahn Chah the lizard hunter, displaying a very natural spontaneity and
down-to-earth, almost childlike simplicity and humour, and the
awe-inspiring formality of his role as head of a large important
monastery, which up to that point was all I had ever seen of him. This
had the effect of undermining many of my own preconceptions regarding
what a great enlightened teacher was supposed to be like, and helped me
to see that Ajahn Chah was actually very natural and quite funny. So I
was able to feel less intimidated and more relaxed in being around him. 

I spent the \emph{Vassa} of that year as a \emph{pa-khao} with Ajahn
Chah, when he unexpectedly decided to leave Wat Pah Pong for the
monastery in his home village, Wat Gor Nork, three kilometres away. I
was the most junior of the four foreign disciples who accompanied Ajahn
Chah at that time for what turned out to be a unique Rains Retreat. He
did give some very profound Dhamma talks during this \emph{Vassa}, in
response to specific questions by more senior Western monks who took
advantage of his increased accessibility in such a small place. Most of
this was over my head at the time as my Thai was still pretty minimal, 
and I was for the most part preoccupied with various chores: cleaning
spittoons, etc., such was my lowly position.

Ajahn Chah had come to this little monastery specifically to renovate
it, and soon set about building a new \emph{sāla}. He was often to be
seen supervising the work in progress, strutting around with his big
walking-stick, barking out comments and commands in a most imperious
manner, displaying what appeared to be dissatisfaction, irritation or
even anger. It was really quite intimidating to watch and I was starting
to get a bit put off by it all, when Ajahn Chah seemed to notice that I
was having a few doubts about this performance. He looked across at me
and by way of reassurance pointed to the centre of his chest and said, 
`Nothing here; nothing here!' I realized then that he was actually a
consummate actor and could display behaviour without being at all
affected by it. He was simply doing what was necessary to get the right
response from the village workers, who are culturally conditioned to
respond to that kind of expression of authority. Another time I
witnessed him metamorphose into a really friendly, jovial old uncle or
grandfather in response to a visiting family group -- a most saccharine
performance that at the time struck me as transparently artificial. But
on reflection I could see that it was in fact just right for those
people in that situation, and they departed happy and uplifted.

Through experiences like these I learned to let go of fixed views about
how supposedly enlightened people should or should not act. Ajahn Chah
was very skilful in adapting to circumstances for the sake of inspiring
or teaching others, and this indicated a highly developed mind. But an
unenlightened observer of such outward behaviour cannot see the true
quality of a mind like that. The purity or lack of defilement cannot be
seen directly; all that can be seen is an apparently normal person
displaying normal characteristics and reactions. So we should be very
cautious about jumping to conclusions or passing judgements based on
such superficial observations. As the Buddha pointed out, it is very
difficult for an unenlightened person to know the quality of a wise
person. It needs keen observation over a long period of time -- a very
important point.

Visiting Ajahn Chah back at Wat Pah Pong after that \emph{Vassa}, I
found him directing a contingent of young conscript soldiers who had
come to help clean up the monastery, sweeping, picking up leaves, etc.
There he was, sitting in his wicker chair, waving his stick and
bellowing orders left, right and centre. Seeing it was me who had come
to sit beside him under his \emph{kuṭī}, he made an oblique reference to
the previous encounter at Wat Gor Nork by leaning over and saying with a
little grin, `You can't talk to Westerners like that, can you?'

I was impressed by how much he seemed to understand the character of
Westerners and the problems they had in undertaking the monastic
training. Although he spoke most of the time in the appropriate way for
Thais -- who are conditioned to respond to authority like that -- yet he
was adaptable and quick enough to pick up the ways of dealing with
Westerners, even those who couldn't understand his language. The
villagers were always amazed by how Ajahn Chah, who had very little in
the way of formal education or worldly sophistication, could actually
teach so many Western disciples without even speaking English. Ajahn
Chah would simply point out that they themselves were raising chickens
and buffalo all the time without knowing their language, and were
managing all right!

He was very observant and could quite accurately assess the personality
of approaching newcomers by watching their faces, their postures, the
way they walked, etc. Before they had even sat down or said anything,
Ajahn Chah would make a remark to those present, such as, `This one's
full of doubt!' which subsequent conversation would reveal to be true.

More than anything else, I think it was probably his humour that made
him attractive to Westerners, for whom conceit, views and attachment to
all sorts of worldly knowledge and sophistication could be serious
obstructions. But Ajahn Chah would have means of deflating all that in a
humorous way. It's very difficult to point out somebody's defilements in
an acceptable way that doesn't cause offence or inspire resistance or
rejection. But Westerners generally have a rather sarcastic sense of
humour, and Ajahn Chah would play on that with his own wit and make
people aware of their own faults in a very funny way, which would in
turn endear him to them even more.

Most of the time I was actually with Ajahn Chah, I didn't understand
Thai very well at all, and just as I was getting competent in the
language, he got sick and was incapacitated to the point of being unable
to speak. But although the tapes and books produced in later times made
me aware of what I had missed experiencing personally, I feel no regret
about it, because after being a monk for so many years now, I really
believe that the initiation into spiritual life of those early years
gave me something that has sustained me right up to the present.
Basically, the simple conviction that this is right; it works, is all
you need. This conviction sprang directly from my own experience of
Ajahn Chah's example; this person who seemed to have such cast-iron
integrity, who conveyed complete certainty and a kind of natural
authority that commanded respect. Confronted almost daily by all kinds
of people, problems and questions, he was quite unshaken from this
position of inner certainty and calm. No one could upset him or make him
change this position, and this was most impressive. I had never seen
anyone so constant, and it seemed to be proof that he was operating on
quite a different level from the average person.

So although I can't really claim to have had profound discussions or a
deep, personal connection with Ajahn Chah, just the constancy of his
presence was enough to anchor me to the principles of the training he
taught. And it inspired great confidence to have an example of someone
who had achieved such results from the practice, who embodied the Dhamma
and lived it all the time. Consequently, I never really had doubts about
it or any problems in surrendering myself to it. I had never had a
teacher before or much understanding of what that might imply, and was
also a fairly critical person with a rather cynical bent. But the
example of Ajahn Chah himself made the surrender of opinions and
preferences, the endurance of simplicity and austerity, the tribulations
of diet and climate, etc. a joy to undertake.

Without such an example as a constant reminder, it's very easy to remain
stuck in one's own views and opinions, which is a major obstacle to
success in training. Westerners especially have problems because they
know so much. They know that there are other teachers, other traditions
and books all over the place, and they can just get lost, never really
grasping the point of it all. Ajahn Chah would say, `Don't read books.
Don't write home more than twice a year. You've come here to die!' The
idea of living in the forest and being simple really appealed to me, as
my character naturally disposes me to be that way. It was no great
wrench to take up the forest life.

It's often assumed that living with a teacher means having an in-depth
personal rapport, characterized by weighty discussions of profound
topics pertaining to spiritual life and the highest goals thereof. But
that's not necessarily the case. You never really enter spiritual life
whole-heartedly until you surrender yourself, surrender views and
opinions. Ajahn Chah's genius was in his ability to point this out,
orchestrating an environment or training situation in which people could
become aware of their own defilements and learn not to believe their own
thinking. This is incredibly important. Without the example of someone
who has done it; who lives it, it's really difficult to give up
self-concern. I never had any problems wondering whether I should be
doing this or whether I should go somewhere else. Inspired by Ajahn
Chah's example, I just got on with it. I didn't see any point in going
anywhere else.

Eventually you verify the teaching through your own practice, and you
realize how things change. Your habits change; your character changes.
Your defilements get less. Life gets easier and your mind is more
peaceful. Everything Ajahn Chah has been saying is true!

\subsection{Ajahn Chandapālo}

My experience with Ajahn Chah is very limited because I only saw him one
time before he got really sick, when he could still walk and talk and
function normally. While I was studying in Scotland he was invited to
visit Edinburgh, and he was with Ajahn Sumedho and Ajahn Pabhākaro. He
had just arrived that evening, and he stayed overnight and left the
following day. There was a meeting with a few people. He didn't give a
talk, just questions and answers, and I can't remember what he said, but
it did leave an impression. I remember him as someone totally at ease
and just completely normal. There was nothing really outstanding, he was
just someone who was `right there'. There was no kind of pretence or
play-acting; he was just who he was.

It was just a short meeting, a short meditation. Afterwards he answered
questions and I saw him for an hour or so. I was still quite new to
practice and Buddhism. The only monk I'd met before was Ajahn Sumedho,
so it felt like a very important and fortunate thing to be able to meet
his teacher as well. I felt very much in awe, you know -- Ajahn Chah!

We were waiting for them to arrive and I happened to be just outside in
the corridor when he came. I remember feeling quite shy and embarrassed,
not knowing how to behave. So I just raised my hands in \emph{añjalī} (a
gesture of respect) as he walked past -- he was really short and walking
with a stick. And he stopped and looked up at me, and then carried on.

\emph{Q: Do you remember what year this was in?}

\emph{A:} It must have been 1979. I remember him sitting in a chair and
just looking around, tapping things with his stick. I felt there was
kindness, a good feeling from just being in his presence. It was a long
time ago and very brief, but what stands out is that feeling of the
goodness of his presence, and that he was someone who was very much at
ease.

I've always enjoyed his teachings that have been published in books, 
like `\emph{Bodhinyana}' and `A Taste of Freedom', very inspiring.
There's an apparent simplicity in them, but also the depth and
profundity of his wisdom comes across. And even though you can read them
many times, there's still something that reaches and touches you --
something inspiring. 

\subsection{Ajahn Karuṇiko}

I met Luang Por Chah in England. He came to the Hampstead Vihāra in
1979, when I was still a layman. One of the things I noticed was just
the sense of happiness of Luang Por Chah, his joy and happiness and the
effect that had on my mind. It made me feel very happy to be around him. 
One of the most interesting things about that time was that I had been
meditating for maybe eighteen months and it was very difficult, there
was pain, restlessness. But then Luang Por Chah arrived at the Vihāra, 
and my mind was very calm and peaceful. So I'd go every night because my
meditation was very good when he was there. That was very interesting --
just the power of his presence and my mind went calm. Usually it wasn't
calm, because in my early days of meditation, sitting wasn't easy. But
it was easy to sit when he was there.

Then I think just the power of his \emph{mettā} affected me -- this nice
feeling in the heart. So I really enjoyed just being around him. It was
a very special time, that week he was there; I'd go every night. It was
interesting too that even if he was downstairs in the Vihāra, still my
meditation was good, even when he was not in the room -- incredible. I
noticed that when he left to go back to Thailand. 

He used to tease people; ask people questions and tease them a little
bit. So when I sat there and I was at his feet, just in awe of this
wonderful man, he looked down at me and said, `What do you think it
would be like to sit there for one whole hour without one thought coming
into your mind?', I thought, `Oh, very enlightened!' But he said, `Like
a stone!' and I couldn't answer that! Being around him when he came to
the Hampstead Vihāra when I was a layman was a very wonderful
experience. And that's more or less the only time I really was near to
Luang Por Chah when he was well. 

