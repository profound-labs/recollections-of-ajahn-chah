% Title: Ajahn Chah Passes Away
% Forest Sangha Newsletter 1992 April

On the morning of 16\textsuperscript{th} January, the Sangha in Britain received a brief
message from Wat Pah Nanachat to inform us of the death of Luang Por
Chah. The Venerable Ajahn had been critically ill, paralyzed and
rendered completely incapacitated by brain damage and numerous strokes
over the past ten years. Our winter retreat offered us an ideal
opportunity to pay honour to his example, reflect upon his teachings and
further our practice in the way that he made clear. 

It was during a retreat at Wat Keuan that Ajahn Sumedho and the
Western Sangha who had gathered there heard that Luang Por Chah had been
admitted to Ubon Hospital. Malfunctioning kidneys and heart
complications had proved to be beyond the medical skills of the monks
nursing him. During the ten years of his illness Luang Por had entered
hospital many times, yet on each occasion he had miraculously recovered. 
However, reports soon began to reach us that his body was refusing to
take food and the general state of his health was deteriorating. 

Early on the evening of 15\textsuperscript{th} January the doctors at the ICU realized
that Luang Por's condition had deteriorated to the extent that he was
beyond medical assistance. At 10 p.m. Luang Por was taken by ambulance to
his nursing \emph{kuṭī} at Wat Pah Pong, in compliance with his previous
request that he might pass away in his own monastery. It was at 5.20 a.m.
on 16\textsuperscript{th} January that the body of Luang Par Chah breathed its last, and
in an atmosphere of peace the life of a great Buddhist master came to
its end. 

The attendant monks chanted the reflection that death is the natural
consequence of birth and that in the cessation of conditions is peace, 
then prepared Luang Por's body for the funeral services. As the news of
his death spread, people began to arrive to pay their respects. Soon
government officials, as representatives of the King, came to perform
the initial ceremonies necessary for a royal funeral. 

Within hours the corpse was moved to the main \emph{sāla}, where it was
laid in an ornately decorated coffin. The coffin was then sealed, and a
picture of Luang Por was placed to the left along with different
requisites such as his bowl and robes. Wreaths from the King, the Queen
and other members of the royal family were placed to the right. In front
of the coffin, extensive flower arrangements created the finishing
touches. 

As the news of Luang Por's death spread, his disciples rushed to the Wat
to pay their respects and offer their support with the preparations to
receive visitors to the monastery. It was decided that during the 15
days following Luang Por's death a Dhamma practice session would be
held, as an offering of remembrance and a focal point around which the
many incoming lay and monastic disciples could collect themselves. The
Sangha from Wat Pah Nanachat would come over every day at around 5 p.m.
and stay until midnight. During this period of 15 days, about 400 monks, 
70 nuns and 500 laypeople resided at Wat Pah Pong, practising
meditation until midnight, listening to talks on Dhamma themes and
participating in various funeral ceremonies. Most of the Sangha were
living out under the trees of the forest, using their \emph{glots}
 (mosquito net umbrellas) as protection from the elements and insects. 
The monastery became a \emph{glot} village. 

Soon a huge open-air restaurant complex sprung up at the entrance to the
monastery, serving free food and drink to the enormous numbers of people
who began to make their way there from all over Thailand. As the days
passed, I began to feel a sense of awe as people streamed into the
monastery from early morning to late at night: people of all ages --
families, school groups and individuals. In those first few days over
50,000 books were distributed, which gives some indication of the
numbers coming. By the fourteenth and fifteenth days, the number of
people coming was steadily increasing to over 10,000 per day. As the
people entered the monastery, they filed quietly down the road leading
to the \emph{sāla}, waiting for an opportunity to enter and bow in
respect, and then to sit for a short while before making way for the
next group. Meanwhile the monks, nuns and resident laypeople would be
sitting in meditation, chanting or listening to a talk. Luang Por Jun
led the funeral chanting and various senior monks gave talks. Ajahn Mahā
Boowa, the renowned forest meditation master, came over from his own
monastery near Udorn to give a Dhamma talk, and commented on the quiet, 
harmonious atmosphere of the Wat, in contrast to the confusion and noise
he had experienced at similar funerals. 

A visit from the King's sister at this time seemed to presage the
arrival of the King for the fiftieth day ceremonies on 6\textsuperscript{th} March. As
always in Buddhism, however, especially in Thailand, nothing is certain. 
The hundredth day after the death of Luang Por will also be a day of
considerable importance. 

Because of the arrangements for the hundreds of thousands of people
expected to attend the actual burning of Luang Por's body (at similar
funerals for famous teachers, up to a million people have attended), and
also to find a day suitable for the King, it was decided to hold the
funeral early in 1993. 

For each of us Luang Por Chah has a personal meaning, depending on our
contact with him. I will always wonder and be inspired at the sight of
tens of thousands of people coming to Wat Pah Pong, to pay respects to a
person who had not spoken for ten years, and with whom most had never
had the opportunity to speak. They came to bow before the body of a
being whom they recognized as personifying our highest aspiration -- a
life free from the blindness of self-centred action. Freed of this
delusion, the goal of the Buddhist path is fulfilled. For me, the whole
occasion demonstrated the breadth and power of the influence of such a
being. 
