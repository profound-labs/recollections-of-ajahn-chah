% Title: Gratitude to Ajahn Chah
% Forest Sangha Newsletter, 1989 July.

June 17\textsuperscript{th} was the 71\textsuperscript{st} birthday of, Venerable Ajahn Chah, spiritual
teacher of over 80 forest monasteries in Thailand, Britain and around
the world. As is customary in the monasteries in England, the day's
practice was offered in gratitude to him and for his well-being. In this
Newsletter we present, through some reflections, an occasion for readers
to recollect what he has made possible for all of us.

\section{Luang Por's Way}

\emph{Venerable Jayasāro was formally abbot at Wat Pah
Nanachat. In 1988 he visited the UK as a translator for Venerable Chao
Khun Paññananda. The following reflections on Ajahn Chah's life are
taken from a talk given at Amaravati Buddhist Centre in June of that
year.}

My own first meeting with Ajahn Chah was on the full moon of December
1978. I had spent the Rains Retreat of that year as an eight-precept lay
person with Ajahn Sumedho at Oakenholt in England. After the retreat I
went out to Thailand. When I arrived at Wat Pah Pong, Venerable Pamutto, 
an Australian monk resident there at the time, took me to see Ajahn
Chah. He was sitting under his \emph{kuṭī} having a drink. He looked at
me and smiled very warmly. He held out the drink he had in his hand, so
I crawled over and took it. As I returned to my place I found there were
tears welling up in my eyes. I was emotionally overcome for quite a
while. Since that day I don't think I have ever wanted to leave the
monastery or do anything except be a disciple of Ajahn Chah. 

People often presumed there would be a problem with language for
Westerners who wanted to stay at the monastery, but this was not the
case. Someone once asked Ajahn Chah: `Luang Por, how do you teach all
your Western disciples? Do you speak English or French? Do you speak
Japanese or German?' `No,' replied Ajahn Chah. `Then how do they all
manage?' he asked. `Householder', Ajahn Chah enquired, `at your home do
you have water buffaloes?' `Yes, Luang Por' was the reply. `Do you have
any cows, or dogs, or chickens?' `Yes, Luang Por' `Tell me', Luang Por
asked, `do you speak water-buffalo: do you speak cow?' `No,' the
householder replied. `Well, how do they all manage?'

Language was not so important to Luang Por. He knew how to see through
the exterior trappings of language and culture. He could see how all
minds basically revolve around the same old centres of greed, hatred and
delusion. His method of training was one of pointing directly at the way
our minds work. He was always showing us how craving gives rise to
suffering -- actually allowing us to see the Four Noble Truths directly.
And for him, the way of exposing desires was to frustrate them. In his
vocabulary, the words `to teach' and `to torment' were more or less
interchangeable.

Such training as this can only take place if everyone in the monastery
has great confidence in the teacher. If there is the slightest suspicion
that he might be doing it out of aversion or desire for power, there
wouldn't be any benefit. In Ajahn Chah's case everyone could see that he
had the greatest courage and fortitude, and so could trust that he was
doing it out of compassion.

Primarily he would teach about letting go. But he also taught a lot
about what to do when we can't let go. `We endure', he would say.
Usually people could appreciate intellectually about letting go, but
when faced with obstacles they couldn't do it. The teaching of patient
endurance was a central aspect of the way that he taught. He continually
changed routines around in the monastery so you wouldn't become stuck in
ruts. As a result you kept finding yourself not quite knowing where you
stood. And he would always be there watching, so you couldn't be too
heedless. This is one of the great values of living with a teacher; one
feels the need to be mindful.

In looking into Ajahn Chah's early life, it was inspiring for me to find
just how many problems he had. Biographies of some great masters leave
you with the impression that the monks were perfectly pure from the age
of eight or nine -- that they didn't have to work at their practice. But
for Ajahn Chah practice was very difficult. For one thing, he had a lot
of sensual desire. He also had a great deal of desire for beautiful
requisites, such as his bowl and robes, etc. He made a resolution in
working with these tendencies that he would never ask for anything, even
if it was permitted to do so by the Discipline. He related once how his
robes had been falling to bits; his under-robe was worn paper-thin, so
he had to walk very carefully lest it split. Then one day he heedlessly
squatted down and it tore completely. He didn't have any cloth to patch
it, but remembered the foot-wiping cloths in the Meeting Hall. So he
took them away, washed them and patched his robe with them.

In later times when he had disciples, he excelled in skilful means for
helping them; he had had so many problems himself. In another story, he
related how he made a resolution to really work with sensual desire. He
resolved that for the three-month Rains Retreat he would not look at a
woman. Being very strong-willed, he was able to keep to this. On the
last day of the retreat many people came to the monastery to make
offerings. He thought, `I've done it now for three months, let's see
what happens.' He looked up, and at that moment there was a young woman
right in front of him. He said the impact was like being hit by
lightning. It was then that he realized mere sense restraint, although
essential, was not enough. No matter how restrained one may be regarding
the eyes, ears, nose, tongue, body, and mind, if there wasn't wisdom to
understand the actual nature of desire, then freedom from it was
impossible.

He was always stressing the importance of wisdom, not just restraint, 
but mindfulness and contemplation. Throwing oneself into practice with
great gusto and little reflective ability may result in a strong
concentration practice, but one eventually ends up in despair. Monks
practising like this usually come to a point where they decide that they
don't have what it takes to `break through' in this lifetime, and
disrobe. He emphasized that continuous effort was much more important
than making a great effort for a short while, only to let it all slide. 
Day in, day out; month in, month out; year, in year out: that is the
real skill of the practice. 

What is needed in mindfulness practice, he taught, is a constant
awareness of what one is thinking, doing or saying. It is not a matter
of being on retreat or off retreat, or of being in a monastery or out
wandering on \emph{tudong}; it's a matter of constancy: `What am I doing
now; why am I doing it?' Constantly looking to see what is happening in
the present moment. Is this mind state coarse or refined?' At the
beginning of practice, he said, our mindfulness is intermittent, like
water dripping from a tap. But as we continue, the intervals between the
drips lessen and eventually they become a stream. This stream of
mindfulness is what we are aiming for. 

It was noticeable that he did not talk a lot about levels of
enlightenment or the various states of concentration absorption
 (\emph{jhāna}). He was aware of how people tend to attach to these terms
and conceive of practice as going from this stage to that. Once someone
asked him if such and such a person was an \emph{arahant} -- was
enlightened. He answered, `If they are then they are, if they're not, 
then they're not; you are what you are, and you're not like them. So
just do your own practice.' He was very short with such questions. 

When people asked him about his own attainments, he never spoke praising
himself or making any claim whatsoever. When talking about the
foolishness of people, he wouldn't say, `You think like this and you
think like that', or `You do this and you do that.' Rather, he would
always say, `We do this and we do that.' The skill of speaking in such a
personal manner meant that those listening regularly came away feeling
he was talking directly to them. Also, it often happened that people
would come with personal problems they wanted to discuss with him, and
that very same evening he would give a talk covering exactly that
subject. 

In setting up his monasteries, he took a lot of his ideas from the great
meditation teacher Venerable Ajahn Mun, but also from other places he
encountered during his years of wandering. Always he laid great emphasis
on a sense of community. In one section of the \emph{Mahāparinibbāna
Sutta}\footnote{Dīgha Nikāya 16.} the Buddha speaks
about the welfare of the Sangha being dependent on meeting frequently in
large numbers, in harmony, and on discussing things together. Ajahn Chah
stressed this a lot. 

The \emph{Bhikkhu} Discipline (the \emph{Vinaya}) was to Ajahn Chah a
very important tool for training. He had found it so in his own
practice. Often he would give talks on it until one or two o'clock in
the morning; the bell would then ring at three for morning chanting. 
Monks were sometimes afraid to go back to their \emph{kuṭīs} lest they
couldn't wake up, so they would just lean against a tree. 

Especially in the early days of his teaching things were very difficult. 
Even basic requisites like lanterns and torches were rare. In those days
the forest was dark and thick with many wild and dangerous animals. Late
at night you could hear the monks going back to their huts making a loud
noise, stomping and chanting at the same time, On one occasion twenty
torches were given to the monastery, but as soon as the batteries ran
out they all came back into the stores, as there were no new batteries
to replace them. 

Sometimes Ajahn Chah was very harsh on those who lived with him. He
admitted himself that he had an advantage over his disciples. He said
that when his mind entered \emph{samādhi} concentration for only 30
minutes, it could be the same as having slept all night. Sometimes he
talked for literally hours, going over and over the same things again
and again, telling the same story hundreds of times. For him, each time
was as if it was the first. He would be sitting there giggling and
chuckling away, and everybody else would be looking at the clock and
wondering when he would let them go. 

It seemed that he had a special soft spot for those who suffered a lot; 
this often meant the Western monks. There was one English monk, 
Venerable Thitabho, to whom he gave a lot of attention; that means he
tormented him terribly. One day there was a large gathering of visitors
to the monastery, and as often happened, Ajahn Chah was praising the
Western monks to the Thais as a way of teaching them. He was saying how
clever the Westerners were, all the things they could do and what good
disciples they were. `All', he said, `except this one,' pointing to
Venerable Thitabho. `He's really stupid.' Another day he asked Venerable
Thitabho, `Do you get angry when I treat you like this?' Venerable
Thitabho replied, `What use would it be? It would be like getting angry
at a mountain.'

Several times people suggested to Ajahn Chah that he was like a Zen
master. `No I'm not', he would say, `I'm like Ajahn Chah.' There was a
Korean monk visiting once who liked to ask him \emph{koans}. Ajahn Chah
was completely baffled; he thought they were jokes. You could see how it
was necessary to know the rules of the game before you could give the
right answers. One day this monk told Ajahn Chah the Zen story about the
flag and the wind, and asked, `Is it the flag that blows or is it the
wind?' Ajahn Chah answered, `It's neither; it's the mind.' The Korean
monk thought that was wonderful and immediately bowed to Ajahn Chah. But
then Ajahn Chah said he'd just read the story in the Thai translation of
Hui Neng. 

Many of us tend to confuse complexity with profundity, so Ajahn Chah
liked to show how profundity was in fact simplicity. The truth of
impermanence is the most simple thing in the world, and yet it is the
most profound. He really emphasized that. He said the key to living in
the world with wisdom is a regular recollection of the changing nature
of things. `Nothing is sure,' he would constantly remind us. He was
always using this expression in Thai -- `\emph{Mai nair}!' -- meaning
`uncertain'. He said this teaching, `It's not certain', sums up all the
wisdom of Buddhism. He emphasized that in meditation, `We can't go
beyond the hindrances unless we really understand them.' This means
knowing their impermanence. 

Often he talked about `killing the defilements', and this also meant
`seeing their impermanence'. `Killing defilements' is an idiomatic
expression in the meditative Forest Tradition of north-east Thailand. It
means that by seeing with penetrative clarity the actual nature of
defilements, you go beyond them. 

While it was considered the `job' of a \emph{bhikkhu} in this tradition
to be dedicated to formal practice, that didn't mean there wasn't work
to do. When work needed doing you did it. And you didn't make a fuss. 
Work is not any different from formal practice if one knows the
principles properly. The same principles apply in both cases, as the
same body and mind are active. And in Ajahn Chah's monasteries, when the
monks worked, they really worked. One time he wanted a road built up to
Wat Tum Saeng Pet mountain monastery, and the Highways Department
offered to help. But before long they pulled out, so Ajahn Chah took the
monks up there to do it. Everybody worked from three o'clock in the
afternoon until three o'clock the next morning. A rest was allowed until
just after five, when they would head off down the hill to the village
on alms-round. After the meal they could rest again until three, before
starting work once more. But nobody saw Ajahn Chah take a rest; he was
busy receiving people who came to visit. And when it was time for work
he didn't just direct it. He joined in the heavy lifting, carrying rocks
alongside everyone else. That was always very inspiring for the monks to
see: hauling water from the well, sweeping and so on, he was always
there, right up until the time when his health began to fail. 

Ajahn Chah wasn't always popular in his province in north-east Thailand, 
even though he did bring about many major changes in the lives of the
people. There was a great deal of animism and superstition in their
belief systems. Very few people practised meditation, out of fear that
it would drive them crazy. There was more interest in magical powers and
psychic phenomena than in Buddhism. A lot of killing of animals was done
in the pursuit of merit. Ajahn Chah was often very outspoken on such
issues, so he had many enemies. 

Nevertheless, there were always many who loved him, and it was clear
that he never played on that. In fact, if any of his disciples were
getting too close, he would send them away. Sometimes monks became
attached to him, and he promptly sent them off to some other monastery. 
Charismatic as he was, he always stressed the importance of the Sangha
-- of community spirit. 

I think it was because Ajahn Chah was `nobody in particular' that he
could be anybody he chose. If he felt it was necessary to be fierce, he
could be that. If he felt that somebody would benefit from warmth and
kindness, then he would give them. You had the feeling he would be
whatever was helpful for the person he was with. And he was very clear
about the proper understanding of conventions. Someone once asked about
the relative merits of \emph{arahants} and \emph{bodhisattvas}. He
answered, `Don't be an \emph{arahant}, don't be a \emph{bodhisattva}, 
don't be anything at all. If you are an \emph{arahant} you will suffer, 
if you are \emph{bodhisattva} you will suffer, if you are anything at
you will suffer.' I had the feeling that Ajahn Chah wasn't anything at
all. The quality in him which inspired awe was the light of Dhamma he
reflected; it wasn't exactly him as a person. 

So since first meeting Ajahn Chah, I have had an unshakable conviction
that this way is truly possible -- it works -- it is good enough. And
I've found a willingness to acknowledge that if there are any problems, 
it's me who is creating them. It's not the form and it's not the
teachings. This appreciation made things a lot easier. It's important
that we are able to learn from all the ups and downs we have in
practice. It's important that we come to know how to be `a refuge unto
ourselves'-- to see clearly for ourselves. When I consider the morass of
selfishness and foolishness my life could have been, and then reflect on
the teachings and benefits I've received, I find I really want to
dedicate my life to being a credit to my teacher. This reflection has
been a great source of strength. This is one form of
\emph{Sanghānussati}, `recollection of the Sangha' -- recollection of
the great debt we owe our teachers. 

So I trust that you may find this is of some help in your practice.

