% Title: The Fiftieth Day Commemoration
% Forest Sangha Newsletter 1992 April

\subsection*{Wat Pah Nanachat, 6\textsuperscript{th} March}

At 1 pm 300 laypeople and 200 monks and nuns gathered in the new
\emph{sāla}. The floor is polished granite and the walls are partially
marbled. Four huge chandeliers hang from the high ceiling. Large
garlands of flowers hang from the walls and the shrine is covered in
artificial lotuses, which look beautiful. The cry of the wild chickens
breaks into the silence -- they are all over the Wat! 

Ajahn Jun gave a \emph{desanā} at 2 p.m., mentioning the debt of gratitude we
all owe to Luang Por Chah. He exhorted us to make an effort to keep up
the practices that Luang Por Chah taught. He also talked about the
benefits of keeping good standards regarding \emph{sīla} and the monastic
conventions, and reminded us that the practice was not in the forest or
the Wat, but is the work of the mind in the body. `So all of Buddhism is
right here in this body/mind. Don't let the practice become perfunctory
-- put life into it. Even though Ajahn Chah is dead, the goodness and
virtue that he embodied are still alive.'

At 7 p.m. about 3,000 lay people and 300 monks and novices gathered in the
new sala for the evening chanting. At 9 pm Luang Por Paññananda gave a
\emph{desanā}. He started by praising Ajahn Chah as one of the great
monks of this era, who taught a pure kind of Buddhism, with nothing
extraneous. Ajahn Chah had trained a Sangha which could continue, most
notably overseas, where monasteries had arisen from his inspiration. 
They represented an historic occasion in the development of Buddhism. 

Luang Por Paññananda commented that Ajahn Chah had taught people to be
wise. The way the Pah Pong Sangha was handling the proceedings was a
good example: in Thailand some degenerate practices had crept into
funeral services, making them an excuse for a party with gambling and
alcohol. But the purpose of a funeral is for the study of Dhamma, not
for distraction! It's a lesson, a reminder. Even though Ajahn Chah is
dead, the goodness and virtue that he embodied are still alive. We must
maintain that which he gave to us all: we have to be `mediums' for Luang
Por Chah, channelling his goodness and virtue through our hearts. If we
reflect on Luang Por Chah's \emph{mettā}, \emph{sīla} and \emph{paññā}
and internalize them, then it's as if he is in our hearts, far better
than hanging a medallion with his picture on it around our necks. 

Luang Por Paññananda concluded by reflecting that the Buddha left the
Dhamma-Vinaya, not an individual, as our teacher, and that his teaching
was one of sustaining compassion, wisdom and purity. So our practice is
to wish all beings well, refraining from harming others or the
environment. Then to have wisdom -- whatever we're doing, inquiring as
to why are we doing it, what our purpose is and what is the most skilful
means. And to dwell in purity -- honouring goodness by making body, 
speech and mind good, associating with good people, and frequenting
places of goodness. 

Luang Por Paññananda had witnessed a decline in most monasteries after
the teacher died, with schisms occurring between the disciples. So we
should be careful not to get attached to views, or to wealth and gains, 
and agree to have regular meetings in order to maintain harmony. Sangha
and laity should support all the things that are in line with the way
Luang Por Chah taught, and refrain from the things he cautioned us
about. We must all help to do this. 

The evening continued with different senior monks giving talks. Ajahn
Santacitto was next, and his memory of Thai was excellent. They were
still talking when we left at 4.45 am, but probably finished at dawn
with morning \emph{pūjā}. 

