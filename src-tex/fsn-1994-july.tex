% Title: Greg Klein (Ajahn Anando) 3rd November 1946 -- 11th May 1994
% Forest Sangha Newsletter 1994 July

\emph{Greg Klein (Ajahn Ānando) 3\textsuperscript{rd} November 1946 -- 11\textsuperscript{th} May 1994.}

\emph{Below, Ajahn Sucitto remembers Greg Klein, whose ashes were interred at
Cittaviveka on 17\textsuperscript{th} July, when a plaque he had had
made was also laid.}

Something he wrote about his time helping to nurse Luang Por Chah in his
terminal illness not only reflects his own interests, but sums up the
life mystery well. `I like the early morning, the night shift as they
call it, very much, because one can spend time alone with Luang Por. 
From 2 a.m. until maybe 5 a.m. is the time when he seems to sleep the
most peacefully. Then a rather busy time follows; depending on what day
of the week it is we might clean part of the room, very quietly, and
then prepare things for waking him at 5.30 to bathe and exercise him. 
Then, the weather and his strength permitting, we put him in the chair, 
the one that was sent from England with the money offered by people in
the West. It's a really superlative chair, it does everything except put
itself away at night! I had a look at what they had made for Luang Por
before. It was quite good for the materials they had, but the wheelchair
that he has now is in a class by itself. A sense of great respect and
affectionate caring goes into the nursing. Although he has been
bedridden for almost six years, he has no bedsores. The monks commented
that visiting doctors and nurses are quite amazed at the good condition
of his skin. The monks who are nursing him never eat or drink anything
or sleep in the room. There is very little talking; usually you only
talk about the next thing you have to do for his care. If you do talk, 
you talk in a quiet manner.

So it is not just a room we nurse him in, it is actually a temple. One
of the senior Thai Ajahns asked me how I was feeling about being with
Luang Por. First I expressed my gratitude for the opportunity. He said, 
`But how are you feeling?' I said, `Sometimes I feel very joyful, and
sometimes not so joyful.' I realized that this was going to be a Dhamma
discussion. He was using the opportunity to teach me something. He went
on to say there is a lot of misunderstanding about what is happening to
Luang Por. `Actually, it's just the \emph{saṅkhāras}, the aggregates, going
through a certain process.' He said, `All we really need to do is just
let it go, let it cease; but if you did that people would criticize, 
they would misunderstand and think you were heartless and cruel, and
that you would let him die. So because of that, we nurse him, which is
fine also.' He then went on to say that the reason we perceive things
the way we do is that we are still attached to our views and our
opinions. But they are not right, they still have the stench of self. He
said that Luang Por practised \emph{mettā bhāvanā}, meditation on
loving-kindness, very much, and this is why people were drawn to him; 
but that has a certain responsibility. `For myself,' he said, `I incline
quite naturally towards equanimity, serenity. There is no responsibility
there, it's light.'

On the last morning, when I arrived at Luang Por's \emph{kuṭī}, he was
lying on his side, and I just spent a long time sitting facing him, very
consciously directing thoughts of loving-kindness and gratitude towards
him, expressing my happiness at having had the great blessing of
spending some time with him, of having heard his teaching, appreciated
it and incorporated it into my life. The morning went by very easily and
rapidly. I was sitting looking at him comfortably asleep, and
considering how best to use this very special time. And the message was: 
see it all as \emph{anicca, dukkha, anattā} -- something impermanent, imperfect
and impersonal. That's what takes one beyond; it's all right.

