% Title: A Noble Life
% Forest Sangha Newsletter 1992 April

\chapterNote{Issue 20, published in April 1992.}
\chapter{A Noble Life}

\noindent
\emph{17\textsuperscript{th} June, 1918 -- 16\textsuperscript{th} January 16, 1992}

\bigskip
\noindent
Venerable Ajahn Chah was born on 17\textsuperscript{th} June 1918 in a small village near
the town of Ubon Rajathani, north-east Thailand. Between the ages of 9
and 17 he was a \emph{sāmaṇera} (novice monk), during which time he
received his basic schooling, before returning to lay life to help his
parents on the farm. At the age of 20, however, he decided to resume
monastic life, and on 26\textsuperscript{th} April, 1939 he received \emph{upasampadā}
(\emph{bhikkhu} ordination).

Ajahn Chah's early monastic life followed a traditional pattern of
studying Buddhist teachings and the Pāli scriptural language. In his
fifth year as a monk his father fell seriously ill and died, a blunt reminder of
the frailty and precariousness of human life. This caused him to think
deeply about life's real purpose, for although he had studied
extensively and gained some proficiency in Pāli, he seemed no nearer to
a personal understanding of the end of suffering. Feelings of
disenchantment set in, and finally (in 1946) he abandoned his studies
and set off on mendicant pilgrimage. 

He walked some 400 km to central Thailand, sleeping in forests and
gathering alms-food in the villages on the way. He took up residence in a
monastery where the Vinaya was carefully studied and practised. While
there he was told about Venerable Ajahn Mun Bhuridatta, a most highly
respected meditation master. Keen to meet such an accomplished teacher, 
Ajahn Chah set off on foot for the north-east in search of him. 

At this time Ajahn Chah was wrestling with a crucial problem. He had
studied the teachings on morality, meditation and wisdom, which the
texts presented in minute and refined detail, but he could not see how
they could all actually be put into practice. Ajahn Mun told him that
although the teachings are indeed extensive, at their heart they are
very simple. With mindfulness established, it is seen that everything
arises in the mind; right there is the true path of practice. This
succinct and direct teaching was a revelation for Ajahn Chah, and
transformed his approach to practice. The way was clear. 

For the next seven years Ajahn Chah practised in the style of the
austere Forest Tradition, wandering through the countryside in quest of
quiet and secluded places for developing meditation. He lived in tiger-
and cobra-infested jungles, and even in charnel-grounds, using
reflections on death to overcome fear and penetrate to the true meaning
of life. In 1954, after years of wandering, he was invited back to his
home village. He settled close by, in a fever-ridden haunted forest
called `Pah Pong'. Despite the hardships of malaria, poor shelter and
sparse food, disciples gathered around him in increasing numbers. The
monastery which is now known as Wat Pah Pong began there, and eventually
branch monasteries were also established elsewhere. 

The training in Ajahn Chah's monasteries was quite strict and
forbidding. Ajahn Chah often pushed his monks to their limits, to test
their powers of endurance so that they would develop patience and
resolution. He sometimes initiated long and seemingly pointless work
projects in order to frustrate their attachment to tranquillity. The
emphasis was always on surrender to the way things are, and great stress
was placed upon strict observance of the Vinaya. 

In 1977 Ajahn Chah was invited to visit Britain by the English Sangha
Trust, a charity with the aim of establishing a locally-resident
Buddhist Sangha. He took Venerable Sumedho and Venerable Khemadhammo
along, and seeing the serious interest there, left them in London at the
Hampstead Vihāra. Another two of Ajahn Chah's Western \emph{bhikkhus}, 
who were then visiting their families in North America, were invited to
stay in London to make up a small resident Sangha. He returned to
Britain in 1979, at which time the monks were leaving London to begin
Chithurst Buddhist Monastery in Sussex. He then went on to America and
Canada to visit and teach. 

After this trip and again in 1981, Ajahn Chah spent the Rains away from
Wat Pah Pong, since his health was failing due to the debilitating
effects of diabetes. As his illness worsened, he would use his body as a
teaching, a living example of the impermanence of all things. He
constantly reminded people to endeavour to find a true refuge within
themselves, since he would not be able to teach for very much longer. 

Before the end of the 1981 Rains, he was taken to Bangkok for an
operation; however, it did little to improve his condition. Within a few
months he stopped talking, and gradually he lost control of his limbs
until he was completely paralyzed and bedridden. From then on he was
diligently nursed and attended by his \emph{bhikkhu} disciples, grateful
for the occasion to offer service to the teacher who so patiently and
compassionately showed the Way to so many. 

