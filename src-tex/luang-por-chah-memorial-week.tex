
\begin{quote}\itshape
Every year during the week leading up to the anniversary of Ajahn
Chah's death on 16\textsuperscript{th} January, there is a great
gathering at his monastery in north-east Thailand, when many of his
disciples come together for six days of Dhamma practice. Ajahn
Siripañño, the senior monk at Wat Dtao Dam at the time of writing, provides the
following account from his perspective of living as a monk in Ajahn
Chah's branch monasteries in Thailand.
\end{quote}

\noindent
12\textsuperscript{th} January 2009, and all over Thailand motorbikes,
cars, pick-up trucks, mini-vans and tour buses are making their way to
the north-eastern province of Ubon, heading for a certain monastery --
Wat Pah Pong. Those making the journey are looking to spend a week
imbibing the spirit and teachings of a forest master now long gone,
Ajahn Chah. Most never met him in person, but the books, tapes and
first-hand accounts of his life have inspired them enough to make changes
in their lives, to take up meditation, and now to join the annual
pilgrimage to where it all began and take part in a week of communal
Dhamma practice. 

The name of the event translates as `Dhamma practice festival in honour
of the Teacher'. Actually, the word \emph{ngan} -- here translated as
`festival' -- usually means work. But it can also mean any kind of event
or celebration: birthdays, weddings, funerals, festivals -- any kind of
activity, really. The Ajahn Chah \emph{ngan} combines many things: the
serious spiritual work of keeping precepts, meditating and listening to
Dhamma talks, socializing with old friends, and having fun making new
ones\ldots{}. This is against the backdrop of reaffirming one's
dedication to living in line with the teachings of the Lord Buddha and, 
more recently, Ajahn Chah, or Luang Por (`venerable father') as he is
affectionately known. Of the thousands who arrive from near and far, 
some come to practise and hear the Dhamma, some to give and participate
in large measure or small, and some come just to check out the scene, 
and enjoy the free food available for all. 

Luang Por Chah passed away on 16\textsuperscript{th} January 1992, and every year since
his funeral on that date the following year, a gathering has taken place
at his monastery Wat Pah Pong. The number of participants keeps
increasing. This year saw over a thousand monks and novices and five
thousand laypeople put up mosquito nets (and, more and more these days, 
tents) all over the monastery, doing their best to let go of the outside
world and focus their hearts on a different dimension. With Luang Por's
teachings as the conduit, the practice turns one inwards -- to taste
peace, know truth and find oneself. 

Tan Ajahn Liem, the abbot of Wat Pah Pong (and these days himself
referred to as `Luang Por') is sitting under his \emph{kuṭī} receiving
some monks as they arrive to pay their respects. A man of few words, he
gives the young monks advice and encouragement like a warm father. `It
just got a bit colder, but it's not too bad. Last night was about 15
degrees. It'll take a couple of days for the body to adjust, that's all. 
If you put your sleeping sheet directly on the hay it will be warmer. A
plastic groundsheet will stop your body heat from getting trapped in the
hollow stalks, so you'll be colder. We have plenty of toilets these
days, so you should be comfortable \ldots{} not like before. There's
space to put up your mosquito nets behind the Uposatha Hall. Around the
\emph{chedi} is full of laypeople these days, so it's not so
appropriate. How many of you came? For the next few days you should
surrender to the schedule. This will help eradicate unwholesome states
of mind such as arrogance and conceit, and the need to have things your
own way. Otherwise you will always fall under the sway of defilements
and craving. It takes effort, though -- \emph{viriyena dukkhamacceti}: 
``suffering is overcome through effort''. But if you practise correctly
your hearts will experience the happiness of inner peace.'

He pauses and looks up. `Have you set up your bowls for the meal yet? 
No? Off you go then. It's almost time.'

The monks and novices head for the eating hall, directly behind the main
\emph{sāla} which is now slowly filling with white-clothed laypeople. 
Women far outnumber men. Before the meal every day the Eight Precepts
are given and there is a half-hour Dhamma talk. On this, the first day, 
it is Luang Por Liem, like a welcoming host, who gives the introductory
talk. He stresses that initially we have come out of faith in the Buddha
and Luang Por Chah, but that in order to carry out their teachings we
need to develop true \emph{sati} -- true mindfulness: 

`We are all just part of nature: the body must change and return to its
origins. When we think in this way the mind will tend to seclusion, 
rather than clinging to views and conceit. Dwelling secluded in body and
mind, we are able to see the true nature of reality. And so we won't
fall under the sway of things that can obsess the mind and wrong views
which stain the mind. The body is just a natural resource we can make
use of -- not a being, not a person, animal or individual. If we
understand this the mind will feel cool and happy, not anxious and
confused. If we strive in this way we will attain the goal we are
seeking. We have a good opportunity, so try to do it: renounce and
abandon the things that cause you worry. The Buddha taught us to abandon
all worldly dhammas. We can't even depend on our friends and relatives. 
Ultimately we have to build our own inner refuge.'

He outlines the daily routine, emphasizing the need to be harmonious and
helpful as we will be spending a week living together in such large
numbers. Meditation, too, is taught in brief.

`Breathe in and out. See that it's just nature doing its job, breath
coming in and going out. When we understand that our awareness of this
is an aspect of our mind, we see that even this is a
\emph{saṅkhata dhamma} (a conditioned phenomenon). There is no self in
there. The mind experiences the breath. The mind has no physical matter, 
yet that is where \emph{dukkha} arises. All mental states are
impermanent, so develop the quality of patient endurance with regard to
all mental states, good and bad. Usually we get lost in our moods, and
this keeps us away from the correct path of practice \ldots{}

`Whatever posture you are in, you are grounded on the earth. Keep this
deep awareness (Thai: \emph{poo roo}) in mind all the time. This way you
won't think of the body as a self. It will lead to a pure happiness
arising in the mind. Instead of delighting in those things which deceive
us -- things people run to like insects drawn to a flame -- cultivate
faith in the Buddha's awakening \ldots{} Develop yourself internally
with your mind and externally with your actions. You all know the duties
regarding the lodgings and toilets. They are communal property, not
owned by anyone, including the abbot. People who are mindful keep a
place clean and well maintained.'

Knowing it's almost nine o'clock, he concludes, `Now it's time to
provide our bodies with the sustenance we need to carry us through the
next day and night, so I will end there. I wish to express my gladness
that you have all come, and encourage you to make a firm determination
to practise with integrity this week.'

For the rest of the day, monks and laypeople arrive at Wat Pah Pong in a
constant stream. Luang Por Liem receives incoming Sangha members under
his \emph{kuṭī} all day, and by evening he still has not had a chance to
find his own spot in the forest to put up his mosquito net and lay down
a bed of straw like everyone else. He is just slipping away when a monk
approaches him quickly to say that Ajahn Sumedho has arrived to pay
respects.

He returns to his seat, first putting on his robe, and the large group
of Western \emph{bhikkhus}, including Ajahn Sumedho, bows three times. 
The two old friends chat for a while, inquiring after each other's
health, and Luang Por Liem asks about the various branch monasteries in
England. They have known each other for almost 40 years. Practising
together in the old days, travelling on \emph{tudong} and serving their
teacher -- theirs is a lifelong bond, bound up with much mutual warmth
and respect. All over the monastery similar scenes are taking place: 
monks who have spent time together in the past are now meeting again, 
paying respects and catching up, like childhood friends. 

After about half an hour there is a pause and Luang Por Liem, a little
sheepishly, excuses himself. `It will be getting dark soon, I still
haven't put up my net.' There are smiles all round and the visitors
again bow three times. Luang Por Liem disappears into the twilight of
the forest. 

By the evening of the first day, several hundred monks have arrived and
the number of laypeople is over three thousand. There are free food
distribution tents set up -- over a hundred different stalls and
marquees sponsored by individuals, branch monasteries, government
offices and other groups. For the next week, almost round the clock
there will be all kinds of food and drink available for anyone who wants
them. Luang Por Kampan Ṭhitadhammo mentioned this in the talk he gave on
15\textsuperscript{th} January. 

`It's as if the whole country is coming together here. This is the
result of Luang Por's life. Just look at the food tents. It's like a
wholesome cycle of goodness. People come here to hear the Dhamma. Then
they give food to others. Other people come to eat, but in doing so they
get to listen to the Dhamma. Then they in turn want to give.'

Some locals, unable to sponsor a tent for the whole week, simply drive
their pick-up into the monastery with the back full of some kind of
tasty snack. Parking it just inside the monastery gate, they hand out
their offerings to passers-by. In not too long the food is gone and they
drive off, happy to have been a part of the event and to have taken the
family on such a fun outing. 

The local hospitals provide first-aid tents as well as traditional Thai
massage and reflexology for the Sangha members. Last year there was free
dental treatment and this year eye tests and glasses were offered in a
marquee just opposite Luang Por's \emph{chedi}. Over the years the scope
of the gathering has broadened, as well as the range of participants. 
Lay supporters from Abhayagiri monastery in California won the hearts of
everyone when they prepared and served American snacks from a food tent
they set up a few years ago. Professionals and teachers from Bangkok
come and camp around the \emph{chedi}, as well as members of what the
Thais call `\emph{Hi So}' (from the English `high society') -- slang for
the aristocracy and well-heeled elite, who genuinely want to put down
much of the superficiality and stress of modern life and reconnect with
something more meaningful and peaceful. Some tents may be fancier than
others, but everyone keeps the Eight Precepts and most stick diligently
to the schedule -- sharing together in the pre-dawn chill of morning
chanting, queuing for food and toilets and splashing down with a bucket
of cold water to bathe. It is no small matter for some. 

Every year more schoolchildren come in large groups. All wearing white
-- girls camping in one area, boys in another -- they have all the
playful energy of teenagers everywhere. But a genuine sense of respect
and decorum is also there, as if they know that although it's not as
much fun as a usual school trip, somehow it's important, and it's only a
few days after all. 

It's 2.45 a.m. Way too early. But from the high bell tower to the north
of the eating hall, the repetitive striking shatters the stillness. It's
time for morning chanting. You do have a choice, though; you could try
to find an excuse to stay bundled up in a heap of robes on the cosy bed
of straw. You're still a bit weak from that diarrhoea a few days ago, 
your throat seems to hurt a bit -- wouldn't want to get sick on day two
-- with so many monks, no one else would really notice if you weren't
there. But it's useless. Only the previous day, in a talk to the Sangha,
Ajahn Anek had reminded everyone that in Luang Por's time everyone was
at morning chanting, and not all wrapped up in brown shawls and
blankets, either. Then you had to sit with your right shoulder exposed, 
patiently enduring the cold weather and practising \emph{ānāpānasati}
 (mindfulness of breathing). You imagine Luang Por Chah's presence
standing next to where you are lying curled up, looking down
stony-faced: `Eugh! Is this how you practise?' Spitting out some red
betel-nut juice, he turns around and disappears into the void. You don't
really have a choice. 

By 3.05 the \emph{sāla} is nearly full with monks sitting, as is the eating
hall. With the exception of one monk known for his eccentricity who has
crafted himself a Mexican-style poncho, almost no \emph{bhikkhus} are wrapped
in blankets as they were the previous morning. Ajahn Anek's words have
had the desired effect, and the new generation of monks seems keen to
show its fighting spirit. 

The laypeople, who somehow seem to have more enthusiasm for morning
chanting than do the monks, have gathered \emph{en masse}, and the women
-- \emph{mae awks} as they are known in the local dialect -- fill the
\emph{sāla} and flow back out along a wide concrete road. At 3.15 the old
grandfather clock chimes and one of the senior monks rings a bell: 
`\emph{Gra--ahp}' he says over the microphone, Thai for `It's time to
bow and chant.' `\emph{Yo so Bhagavā} \ldots{}' The monk with the
microphone tries to push the pace and raise the pitch, but the massed
ranks of \emph{mae awks} have the strength of numbers and the chanting
stays slow and low. Some find the whole thing tedious; others are filled
with devotion and inspiration. For 45 minutes these ancient Pāli words
and their modern Thai translation are recited line by line, to a
slightly singsong melody that is written only in the hearts of those who
know it and who learned it themselves by listening and following along
from the time they first came to the monastery. 

From 4 until 4.45 there is a period of meditation. Fighting the cold and
fatigue, for many it's nothing but a struggle not to wrap up, fall
asleep, or both. Others seem to have found an equanimity of body and
mind. Seated on the hard granite floor, they embody the peace and wisdom
of the Buddhas; still and silent, aware and knowing, breath going in, 
breath going out. 

By 5 a.m. the monks are setting up the eating hall, sweeping, mopping, 
and putting out tissues, water and spittoons. Next they prepare their
bowls and put on their robes for alms-round. A senior
monk has the microphone and is going through some of the points of
etiquette for alms-round: wearing one's robes properly, walking with
eyes downcast, not swinging the arms and body about, keeping silent, and
many other minor points of practice. Some newly ordained monks and
novices may still be learning all this. Others will have heard it year
in, year out. Yet somehow it has a freshness every time and an immediate
relevance. These minor training rules and the small points of monastic
etiquette, collectively called \emph{kor wat} in Thai, were given huge
importance by Luang Por Chah as the way to begin training the mind: by
letting go of doing things one's own way and being mindful to do things
the prescribed way. The Buddha laid down these principles over 2,500
years ago, and Luang Por knew their value. 

Wat Pah Pong has about a dozen alms routes that wind through the
surrounding villages. But when a thousand or so \emph{bhikkhus} are in need of
some sustenance, it's the nearby town of Warin and the city of Ubon that
provide much of the additionally-required calories. As dawn approaches, 
the monks head out of the monastery gates, each with an alms bowl and
some with two if they are attending a senior \emph{bhikkhu}. Lining the
road to the left, right and directly in front of the gate is a motley
fleet of assorted vehicles: draughty buses and pick-ups and, for the
lucky ones, warm mini-vans. The monks swarm aboard and wait. At an
unseen signal, suddenly engines rev and wheels roll, and the parade of
vehicles heads for various markets and residential areas. When they
arrive at their destination the monks form lines of up to 50 or more
and walk along pre-designated routes. People of all ages line the way
and make their offerings, doing their bit for the \emph{ngan}. The food
is simple but bountiful, and by the end of the alms-round each monk may
have emptied his full bowl up to a dozen or more times: sticky rice, 
boiled eggs, instant noodles, orange drinks, tinned fish, bananas, 
coconut sweets \ldots{} staples of the modern Isan (north-east Thai)
diet woven into this hallowed Isan custom -- offering food to the monks
at dawn. No amount of economic crisis, it seems, can deprive people of
this simple joy. And no matter how often one has taken part in this act
of giving and receiving, it remains a little mysterious, and quite
magical. 

The buses and pick-ups return with the monks and countless baskets
brimming with food. There are still two hours until the Sangha will eat, 
and as they walk past the food tents the novices and young monks glance
enviously at laypeople nibbling away on breakfast snacks. The more
senior monks keep their eyes down, having by now learned that watching
someone else eat while you are cold and hungry makes neither you nor the
other person feel any better. 

Everyone gathers at 8 a.m. in the main \emph{sāla} for the daily taking
of the Precepts. A \emph{desanā} then follows, inevitably covering
familiar ground: our debt of gratitude to Luang Por; the importance of
\emph{sīla} as the basis of happiness and the stepping-stone to
\emph{samādhi} and \emph{pañña}; meditation and the need to see through
the illusory nature of our thoughts and moods, to go beyond desire by
establishing a peaceful mind and taste that special happiness the
Buddhas praised and that Luang Por experienced for himself, doing
everything he could for us to be able to do so as well. 

`Careful not to take too much food; think of all the people still behind
you \ldots{} A purse has been found with some money and keys. If you
think it's yours, come and claim it, but you have to say what colour it
is and how much money is in there \ldots{} Remember not to store food
in your mosquito nets. Ants will come for it -- and you'll be tempted to
eat after midday\ldots{}.'

After the meal, once the Sangha members have washed and dried their
bowls, Luang Por Liem gives a 15 minute exhortation, with speakers
hooked up in both the monks' and nuns' eating halls, encouraging us all
to reflect on our duties as \emph{samaṇas}, recluses who have gone forth
from the household life into homelessness: from cleaning toilets to
realizing Nibbāna and everything in between. 

By 10.30 the sun is filtering through the tall trees and slowly warming
up the forest -- time for most people to have a quick lie-down before
the 1 p.m. gathering for meditation and more Dhamma instruction. These
days the Sangha gathers in the \emph{Uposatha} Hall, or \emph{bot} (a
Thai short form of the Pāli word \emph{uposatha}), the building where
Sangha rituals such as ordinations take place. The \emph{bot} is
jam-packed with monks and the heat and stuffiness build up. Heads begin
to nod, then droop entirely. At 2 p.m. a senior monk gives a talk. A
frequent refrain in these afternoon talks particularly aimed at the
monks is how tough it was living at Wat Pah Pong in the early days. All
requisites, including food, were scarce. You couldn't even pick your own
food, as it was ladled into your bowl for you. There was rarely a sweet
drink in the afternoon, and chores were physically draining, including
hauling water from a well to fill jars for toilets, bathing and
foot-washing. Then there were sweeping, cleaning and general
maintenance. If something was broken you tried to fix it, and if it
couldn't be repaired you went without. Requesting a new one wasn't an
option. But it's the love and respect for Luang Por Chah that come
across most vividly from these elder-most senior monks, as expressed in
a talk from Ajahn Anek.

`Luang Por wished us well from head to toe. Even if our minds didn't
like what he was teaching us, our actions had to comply. We were like
children bathing in a cesspit. Our loving father comes along and says,
`Children, what are you doing that for?' `It's fun.' `Get out. Now!' And
Dad reaches in and pulls us out, and gets water to clean us. And pulling
us out is no easy job. Some Ajahns leave their disciples to wallow in
the cesspit. But Luang Por never did. With just his instruction he was
able to extract poison from our hearts. It was like taking a bitter
medicine which tasted awful, but we knew it would save our lives
\ldots{}

`Luang Por's teaching spread far and wide: Patiently endure. Endure with
patience. Dare to be patient. Dare to endure. \emph{Khantī paramaṃ tapo
tītikkhā:} patient endurance is the supreme incinerator of defilements.
\emph{Khantī,} or patient endurance, is like a fire that no coal or
electricity could ever produce. We chant \emph{tapo ca brahmacariyañca}
-- the austerities of leading the Holy Life. These are the austerities
that can burn up our defilements.

`One aspect of this is the morning and evening chanting \ldots{}
Please give up your own preferences and be present for these activities. 
If during morning chanting there are no monks, but for the meal there
are loads, it feels a bit strange, doesn't it? Between following your
own preferences, or the opinion of society, or the Dhamma -- which is
better? These days notions of personal liberty have so filled people's
minds that they have no room for Dhamma any more. Luang Por is still
with us in spirit. So I ask everyone to please meet together in harmony, 
so that if Luang Por were here in person he would be happy.'

The Sangha pays respects to the senior monk who has given the talk and
an announcement is made to go to the eating hall for the afternoon drink
\ldots{} `if there is one.' It's a slightly tongue-in-cheek reminder
that we shouldn't take anything for granted. These days, though, there
is always something available. Tea, cocoa, freshly squeezed sugarcane
and orange juice; drinks containing aloe vera chunks and other
afternoon-allowable `medicinal' nibbles: sugarcane lumps, candied ginger
and a kind of bitter-sour laxative fruit known as \emph{samor}. The
laypeople too have had their fill of afternoon Dhamma, and those keeping
the Eight Precepts partake of similar fare. 

Everyone is encouraged to take part in a group walking meditation
circumambulation around the \emph{chedi}, the monument to Luang Por Chah
where his crystallized bones, revered by many as holy relics, are kept. 
All too soon it is almost 6 p.m. and the bell is ringing for evening
chanting. The relentlessness of the schedule is a reflection of Luang
Por's training methods: keep everyone pushing against their own
preferences and desires in order to go beyond them; surrender to the
communal routine and allow the sense of self to dissolve into a group
identity; and beyond that to experience the sense of being nothing other
than nature arising and passing away; to have constant reminders and
teachings so that the Dhamma seeps into one's mind -- and the
transformation from being one who suffers through clinging to one who is
free through letting go can take place. 

The first hour of the evening session is silent meditation. The January
air is crisp and cool, and it is the mosquitoes' feeding time. The
\emph{sāla} is full, and all around it and stretching into the forest
are men and women wrapped in white, some young though most older, simply
sitting, being aware of the in- and the out-breath. Inside the
\emph{chedi} too people are meditating, finding warmth in the enclosed
space and inspiration from being so physically close to Luang Por's
remains. As they sit, groups of people, families, children, stream in
and out and pay respects -- three bows -- before heading off, perhaps to
get some noodles at the food tents, or maybe just going home. Over in
the \emph{sāla} the chanting begins, and the voice of the monk leading
it drifts into the \emph{chedi} from a nearby loudspeaker. Many of the
meditators stay motionless, but most slowly open their eyes, and shift
their posture from cross-legged to kneeling in the traditional Thai way
for chanting. By some kind of unvoiced mutual consent they agree that
the monks' pitch is a little too high and settle for something a few
tones lower -- creating an eerie discord which echoes hauntingly around
the inside of the chamber. 

Outside it's noticeably colder. By the time the evening \emph{desanā}
starts at around 8 p.m. the northern wind has picked up, adding to the
talk the flavour of \emph{khantī} -- patient endurance. This was always
one of Luang Por's favourite themes anyway, one reflects. The monks
giving the week's evening talks are Luang Por Chah's most senior
disciples. They know how to inject lightness and humour into their
teachings; stories of Luang Por abound, as well as humorous anecdotes
from their own lives. The language used is mainly central Thai, but
those monks who are native to the north-east will often switch abruptly
to the local Isan dialect, a language full of puns, wordplay and
innuendo, much to the delight of the local crowd. Dhammapada verses, old
sayings, and nearly-forgotten proverbs are given an airing, complete
with the Ajahn's personal commentary. Isan is not a written language, 
and listening to these old monks one gets a sense of the power of an
oral tradition. Even if none of Luang Por's teachings had been recorded, 
we would still be able to enjoy them today from the minds and through
the voices of the disciples he touched. The Buddha's teachings were not
written down for several centuries, yet they managed to survive in a
similar way. 

You are asleep the second your head hits the straw mattress. One day
merges seamlessly into another -- all too soon that monk in the bell
tower is doing his thing and you find yourself heading back to the
\emph{sāla} for morning chanting. Each day is a little easier, though. 
The floor seems less hard. It's a bit warmer, too. The mind is uplifted, 
buoyed by the company of so many people sharing the space and practising
in the same way. Surely that's what it is -- though maybe it's something
else\ldots{}. 

\subsection*{16\textsuperscript{th} January}

The big day arrives. As if to acknowledge one of
the unique aspects of Luang Por's legacy, the international Sangha, the
morning Dhamma talk will be given by Ajahn Jayasāro, who is English. The
evening programme will feature Dhamma talks to be given throughout the
night, but the first one -- the prime-time slot -- will be from Luang
Por Sumedho. 

It is 17 years to the day since Luang Por passed away. He was
cremated on the same day one year later. The main event of the day is a
mass circumambulation of the \emph{chedi} by the whole assembly. The
numbers will swell to many hundreds more, boosted by people who have
come just for this event. With the whole Sangha and all the laypeople
gathered together like a sea of brown robes followed by a white foamy
wake, the effect is quite magical. Beginning in the main \emph{sāla}, 
the assembly walks in complete silence, everyone holding a small set of
candles, flowers and incense, for the few hundred metres until the
\emph{chedi}, which the procession then circumambulates. As everyone
gathers round the \emph{chedi}, a senior monk reads out a dedication to
Luang Por and everyone follows, reciting line by line. The Sangha leads
the way up the steps and into the \emph{chedi}. Each person places their
little offering, then bows and makes way for someone else. 

In the evening Luang Por Sumedho begins his \emph{desanā}. Before moving
to loftier dhammas, he too entertains the crowd with some warm old
memories. He recalls how Luang Por used to teach the Dhamma for hours on
end, cracking jokes and telling stories which would have everyone in
stitches -- except for one person: Venerable Sumedho, this newly arrived
American monk squirming in pain on the cement floor, unable to
understand a word. They've heard it before, but again it brings smiles. 
These stories though, are not told just to get a few laughs. They
capture the spirit of a bygone era for those of us who never heard Luang
Por Chah teach, and they prepare the minds of the listeners to hear and
be more likely to truly receive the essence of the Dhamma: that all is
uncertain and unstable, and that happiness comes from letting go. 

Which is just the insight you need in order to last through a whole
night of Dhamma talks. This all-night talks routine seems to be a unique
feature of the Wat Pah Pong tradition -- and you have to be seriously
dedicated to hearing Dhamma to even want, let alone be able, to sit on a
hard floor for ten hours. Understanding the language, too, is a distinct
advantage. Most people nip off for a small rest at some point in the
evening; but some seem to sit motionless throughout, in a kind of
`\emph{desanā} trance'. The first couple of speakers talk for about an
hour; after that it's half an hour each. So, altogether 15 or so
Dhamma talks ring throughout the forest on loudspeakers right through
till dawn. A bell is struck to let any speaker who's getting a bit
carried away know that his 30 minutes are up. The style of \emph{desanā}
is usually unstructured, which is typical of the Thai forest tradition. 
Anyone who miscalculates his allotted time can therefore easily wrap up
and make way for the next speaker when he hears the bell. The last
speaker is still going at full speed at 5 a.m. as the monks, for one
last time, begin to set up the eating hall and then stream out through the gates
towards the waiting armada of alms-round road transport. 

On this last morning the Sangha and laity gather in the \emph{sāla} one
final time, to take leave and ask forgiveness of the most senior monks. 
After a week of remembrance dedicated to Luang Por Chah, it seems
fitting that the endnote is an acknowledgement of our present-day
teachers. Luang Por Liem, appointed by Luang Por Chah to be his
successor as abbot of Wat Pah Pong, receives the traditional offerings
of tooth-woods -- wooden toothbrushes made from a bitter vine that the
monks meticulously fashion in advance and bring to the gathering to give
to senior Ajahns as a token of respect. 

After a few words of farewell and one last blessing the 2009 memorial
gathering is over. The last meal is taken and followed by a mass exodus. 
Thousands of mosquito nets are taken down and tents dismantled; vans are
loaded, with as many as 15 people crammed in to the back of a
pick-up truck for journeys of up to several hundred kilometres. Rubbish
is collected and areas swept. In the eating hall the spittoons are dried
one last time, the water bottles bagged up for recycling, the sitting
mats put away. Within a few hours the monastery feels deserted. Only the
resident community of 40 or so monks and the nuns in their own
section remain, doing the final clear up. 

The following day is a Sunday. In the afternoon some visitors, including
a couple from Bangkok, stop by Wat Pah Pong to pay respects, and
hopefully make some offerings to Luang Por Liem. A lone monk sweeps the
concrete road around the \emph{chedi}, and not a trace remains of the
thousands of residents over the previous week or the mass
circumambulation the day before. Not someone who seems too interested in
taking a break after a hard day's night, Luang Por Liem is in town
looking for building materials. He won't be long, though, the group is
told. Sure enough, within half an hour he is back. `I went into town to
get some pipes. We are building more toilets for next year's gathering. 
More and more people seem to come. More people means more waste. It's
natural. If we can see the body as part of nature -- natural elements
and not a self -- then peace will arise in the heart. This peace leads
to true happiness.'

\subsection*{Extracts from a desanā given by Ajahn Jayasāro}

It's been 17 years now since Luang Por left us, although actually
that is not quite true. Luang Por never left us -- we are the ones who
leave him behind. Every time we think, say, or do something of which he
pointed out the danger, we leave him behind. There are so many of his
teachings around: books, tapes etc. His Dhamma is still with us. But we
frequently leave his teachings behind, sometimes turning our backs on
the Dhamma entirely. Luang Por Chah is not with us today, but the
question is, are we still with Luang Por Chah? 

\ldots{} Not having Right Understanding (\emph{sammā diṭṭhi}) is what
will prevent true happiness from arising. We won't see the true nature
of the world: the fact that \emph{dukkha} is everywhere. The good news
is that true happiness can also be found. It is not about suppressing
the happiness that we can experience through the eye, ear, nose, tongue, 
body and mind, but rather asking ourselves if that is true happiness. Is
that what we ultimately need ? Sensory happiness makes us waste our time, 
and diverts our interest away from developing ourselves to find that
true happiness. 

Say you had enough money to go abroad and you flew to some other
country. Then from the airport you went straight to a hotel, checked in
and went to your room, closed the windows and stayed there for two
weeks. You then went back to the airport and flew home. Would that be
unwholesome? No. But it would be a pity, a wasted opportunity. Being
born as a human being, but only being interested in the pleasure of
sights, sounds, smells, tastes, touch and thoughts, is a similar waste. 
It's really like living in a dark room. 

\ldots{} Luang Por Chah taught us that our real task in life is to
cultivate a healthy shame and fear of losing our mindfulness
(\emph{sati}). We must always strive to maintain \emph{sati}. If we have
\emph{sati}, it's like we have an Ajahn with us. We feel warm and safe:
whenever we make our mind steady, wisdom is ready to arise. Without
\emph{sati} we will always be slaves of our environment and simply
follow whatever thoughts and moods arise.

\subsection*{Extracts from an 8 a.m. \emph{desanā} given by Luang Por Bundit}

Every second our thoughts and moods are teaching us. People without
Right Understanding think, `Why is it so hot?' or `Why is it so cold?'
But it's just nature doing its job. We don't have to make such a big
deal out of it. If we don't understand the world, we will always
experience \emph{dukkha}. Disappointments will be difficult to accept
and we'll always be living for our hopes and dreams. 

\ldots{} People used to come to pay respects to Luang Por Chah, and
would complain they didn't have time to practise, that they were too
busy looking after their children and everything else. `Do you have time
to breathe?' he would ask. `Yes.' `Well then, practise like that!'

Take up the five primary meditation objects that preceptors give the
newly-ordained as a theme for contemplation: hair of the head, hair of
the body, nails, teeth and skin. Doing this will help free us from being
slaves to the body and all the usual concerns regarding beautification
and health, and obsession with treatments and therapy. 

Luang Por taught us to abandon everything. He repeated it again and
again. In the old days there were no doubts about the correct practice, 
but now everyone has a different opinion about Luang Por's teachings. 

So learn to choose the pure things in life. If you know those things
which are pure and lead to peace, then you will bear witness to the
truth yourself. No one can do it for you, or verify the fruit of your
practice. It's \emph{paccattaṃ} -- to be experienced individually. 

Well, that is enough for today, I'm sure everyone is very hungry. Learn
to choose Dhamma teachings the way you choose the fish you eat. A fish
has scales, bones, intestines, and flesh. Whether you choose the flesh
is up to you. 

