% Title: Luang Por Chah's Relics
% Forest Sangha Newsletter 1994 April

\chapterNote{Issue 28, published in April 1994.}
\chapter{Luang Por Chah's Relics}

\emph{In January 1994 Ajahn Sumedho and Ajahn Attapemo went to Wat Pah
Pong in Thailand and took part in the final ceremonies to enshrine the
Atthi-dhātu (relics) of Luang Por Chah. Ajahn Attapemo explains:}

It was all done quite beautifully, stretching over seven days. Each day
there were periods of meditation and Dhamma talks. On the fourth day, 
quietly, the abbots from most of the 152 branch monasteries gathered to
take the majority of the relics up to the \emph{chedi} (Thai for stūpa, or
pagoda) built for the cremation last year. A chamber had been made
inside, into which three reliquaries were placed. To add to the
blessing, gold necklaces, bracelets and rings were draped over the
reliquaries. Some ladies even took their rings off their fingers to be
enshrined for posterity. Later that day this chamber was sealed with a
concrete lid and granite cap-stone. More than 30,000 people had gathered
for the occasion. The final ceremony took place on 16\textsuperscript{th} January, exactly
two years after Luang Por Chah's death. 

His Majesty King Bhumiphol sent his Chief Privy Officer to lead the
ceremony, and sent a royal invitation to Somdet Buddhajahn to give a
\emph{desanā}. A bronze and glass stūpa nine feet high had been made for
the relics, and the Chief Privy Officer took a crystal platter with
thirty selected pieces of the relics and placed this inside the glass
section of the stūpa. Somdet Buddhajahn and twenty other important monks
invited to honour the occasion led the chanting of `\emph{Jayanto}', along
with 1,200 more monks sitting inside and around the \emph{chedi}.

Along with the relics were the ashes. These were equally divided among
the 152 branch monasteries, including a small packet for every monk and
nun.

Also on that day, Ajahn Liem was officially appointed as Abbot of Wat
Pah Pong.

